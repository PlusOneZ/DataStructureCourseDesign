% History

\chapter{更新历史}

\noindent\textsf{版本v2.00更新}\marginnote{2020/11/28}
\begin{enumerate}[label=\color{three}\protect\circled{\arabic*}]
    \item 全新升级,增加各类文类可选项。包括\textbf{字体、标题编号、版面、颜色主题、参考文献格式、草稿标记、目录分栏和语言等},加入\textsf{部分}结构。
    \item 对字体设置详细修改,中文黑体及宋体均用 \lstinline{xeCJK} 宏包设置伪粗,并将这些特性用于各类标题的设置中。
    \item 使用 \lstinline{cleveref} 宏包进行引用,已预设与正文字间距且多数情况仅使用 \lstinline{\cref} 命令即可。
    \item 使用 \lstinline{\setlist} 具体设置列表间距与缩进,调整图表的间距与浮动参数。
    \item 使用文档结构命令 \lstinline{\frontmatter}、\lstinline{\mainmatter} 和 \lstinline{\backmatter} 修复目录超链接位置和页眉章节内容显示错误问题,增加并改进图片与表格索引目录。
    \item 加入 \lstinline{mhchem} 与 \lstinline{chemfig} 宏包输出化学式,加入 \lstinline{ulem} 宏包输出可设置样式的下划线。
    \item 由于 \lstinline{subfig} 宏包长期未维护可能存在兼容性问题,改用 \lstinline{subcaption} 排列子图。
\end{enumerate}

\noindent\textsf{版本v1.16更新}\marginnote{2020/10/03}
\begin{enumerate}[label=\color{three}\protect\circled{\arabic*}]
    \item 使用 \lstinline{amsthm} 宏包取代 \lstinline{ntheorem} 设置定理环境。
    \item 使用 \lstinline{siunitx} 宏包输出标准单位。
    \item 增加 \lstinline{tabu} 宏包作为表格设置备用。
    \item 去除 \lstinline{\upcite} 命令,使用 \lstinline{natbib} 的 \lstinline{super} 参数统一上标引用格式。
    \item 规范代码排列缩进,删除多余宏包与代码,改变混乱状态。
    \item 重新定义附录命令 \lstinline{\appendix}。
    \item 去除 \lstinline{titlesec} 宏包,避免与 \lstinline{ctex} 冲突。
    \item 设置 \lstinline{listings} 代码环境。 
\end{enumerate}

\noindent\textsf{版本v1.07更新}\marginnote{2020/04/25}
\begin{enumerate}[label=\color{three}\protect\circled{\arabic*}]
    \item 使用 \lstinline{titletoc} 宏包取代 \lstinline{tocloft} 宏包设置目录更为直接。
    \item 对 \lstinline{\ref} 重新包装来定义图表定理等引用。
    \item 宏包排列顺序调整,引入 \lstinline{subfig} 宏包用于排列子图。
    \item 修改多级列表编号与对齐格式,去掉脚注圆圈。
\end{enumerate}

\noindent\textsf{版本v1.00创建}\marginnote{2020/03/10}
\begin{enumerate}[label=\color{three}\protect\circled{\arabic*}]
    \item 为初始版本,在\href{https://github.com/mtobeiyf/whu-thesis}{武汉大学}本科毕业论文\LaTeX 模板的基础上修改标题字体间距等格式而来,改动较少,编译警告多。
    \item 修改页眉页脚,增加页眉线。略微修改目录样式。
    \item 使用 \lstinline{titlesec} 设置标题格式。
\end{enumerate}