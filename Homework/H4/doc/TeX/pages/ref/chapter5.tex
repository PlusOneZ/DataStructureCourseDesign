% Chapter 5

\chapter{其它格式}

\section{代码}
使用 \lstinline{listings} 宏包可以对代码格式进行高度定制,默认语言为 \lstinline{LaTeX},这里以 \lstinline{Python} 为例
\begin{lstlisting}[language=Python,caption={python程序练习题}]
#!/usr/bin/python
# -*- coding: UTF-8 -*-
    
if __name__ == '__main__':
    ptr = []
    for i in range(5):
        num = int(raw_input('please input a number:\n'))
        ptr.append(num)
    print ptr    
\end{lstlisting}

\section{物理量与国际单位}
\lstinline{siunitx} 宏包大大简化了符号输入,\lstinline{\num} 命令可以输入我们想要的各种方式的数字形式,比如科学计数法 \num{-21x.3e5}。而 \lstinline{\SI} 命令用来输出标准单位,比如 \SI{5}{\mole},更多用法与单位见宏包文档。

\section{化学式}
除了用普通公式方式输入,普通化学式还可以用 \lstinline{mhchem} 宏包实现,有机化学相关的公式推荐使用 \lstinline{chemfig} 完成。普通化学式例如 \ce{(NH4)2S}、\ce{CO3^2-{}_{(aq)}}以及
    \[\ce{x Na(NH4)HPO4 ->[\Delta] (NaPO3)_x + x NH3 ^ + x H2O}\]
一个有机结构式的输入例子:
    