
\chapter{Queue 类}

\section{API}

\subsection{类定义}
以下是本题使用队列的 API,各函数的功能已用注释标出。
\begin{lstlisting}[morendkeywords={Queue}, firstnumber=9, caption=Queue 类定义]{}
template <class Type>
class Queue {
    // 一个只含有基础 API 的循环队列,有自动扩充和缩小的功能
public:
    Queue();
    ~Queue();

    bool enqueue(const Type &d);          // 入队操作
    bool dequeue(Type &d);                // 出队操作
    bool peekFront(Type &d) const;        // 查看队首元素
    bool isEmpty() const;                 // 队列是否为空
    int  length() const;                  // 当前元素个数
    void clear();                         // 清空
    ostream &show(ostream &os) const;     // 打印队列元素

private:
    static constexpr int InitSize = 10;   // 初始大小
    int size;                             // 当前元素个数
    int capacity;                         // 当前容量
    int front;                            // 队首
    int rear;                             // 队尾
    Type * data;                          // 元素数组

    int resize(int n);                    // 重新分配
    void tryExtend();                     // 扩张
    void tryShrink();                     // 缩小
};
\end{lstlisting}
\vspace*{1cm}

{
\lstset{
    basicstyle=\ttfamily\color{CPPDark}, 
    numberstyle=\tiny\color{darkgray},
    keywordstyle=\color[RGB]{40,40,255},
    }
\subsection{公开成员}
\begin{enumerate}
    \item \lstinline{bool enqueue(const Type &d);} \\
          入队。\lstinline{d} 是 \lstinline{Type} 类要入队的一个元素。正常情况返回 \lstinline{true}。
    \item \lstinline{bool dequeue(Type &d)} \\
          出队。\lstinline{d} 储存出队元素的值。如果队列为空返回 \lstinline{false}。
    \item \lstinline{bool peekFront(Type &d) const;} \\
          查看队首但不出队,\lstinline{d} 储存出队元素的值。如果队列为空返回 \lstinline{false}。\\
    \item \lstinline{bool isEmpty() const;} \\
          查看是否为空队,是的话返回 \lstinline{true},否返回 \lstinline{false}。
    \item \lstinline{int length() const;} \\
          返回当前队列中的元素个数。\\
    \item \lstinline{void clear();} \\
          清空队列。清空后调用 \lstinline{length()} 应返回 0,调用 \lstinline{isEmpty()} 返回 \lstinline{true}。\\
    \item \lstinline{std::ostream &show(std::ostream &os) const;} \\
          从队头到队尾依次输出元素到 \lstinline{os}。\\
\end{enumerate}

\subsection{私有成员}
\begin{enumerate}
    \item \lstinline{Type * data;} \\
          存放数据的数组,\lstinline{Type} 是范型的型别。
    \item \lstinline{int size; int capacity; int front; int rear;} \\
          分别是 \lstinline{data} 数组有效元素的个数、数组大小,队列之头的位置,队列之尾的位置。\\
    \item \lstinline{int resize(int n);} \\
          重新分配 \lstinline{data} 数组的大小。\lstinline{n} 比当前 \lstinline{size} 大的时候才会做这个动作。
    \item \lstinline{void tryExtend();} \\
          封装了“检查数组是否已满,若已满就扩大到两倍”的动作。
    \item \lstinline{void tryShrink();} \\
          封装了“检查数组是否过空($1/4$ 容量),若过空就缩小一半”的动作。
\end{enumerate}
}

\section{代码逻辑与实现}

\subsection{逻辑描述}


队列是一种操作受限的线性表,除了线性表中\emph{“只有一个直接前驱和一个直接后缀”}的限制之外,队列限制提供的接口保证:%
\begin{inparaenum}
      \item 只能在队列{\bfseries 队首 (\lstinline{front})} 删除元素;
      \item 只能在队列{\bfseries 队尾 (\lstinline{rear})} 插入元素;
      \item 只能查看队首的元素。
\end{inparaenum}
同线性表一样,队列的实现方式(按物理储存分类)也有两种:\emph{数组实现}和\emph{链表实现}。

链表实现与栈相似,在此不赘述。本项目使用数组储存方式维护队列中的数据,使用的数组(像 \lstinline{Vector} 那样)具有重新分配内存空间的功能,%
因此没有明确的元素个数上限。

数组储存队列的方式没有{\kaishu 栈 (Stack)} 那么直观,问题出在需要在一端添加,另一端取出。如果每次从队首取出元素时都像 \lstinline{Vector} 那样%
移动整个序列中的元素,那么开销将会过大。于是在这里使用了{\kaishu 循环队列 (Circular Queue)} 的设计。在循环队列中,维护了两个分别指向队首和队尾的%
指针 \lstinline{front}、\lstinline{rear},{\kaishu 出队 (Dequeue)} 时将 \lstinline{front} 指向的元素返回,并将 \lstinline{front} 后移一个单位;%
{\kaishu 入队 (Enqueue)} 时将 \lstinline{rear} 指向的数据置为插入元素,并将 \lstinline{rear} 后移一个单位。%

{\kaishu “循环”} 这一概念的主要在于\emph{后移一个单位}的操作中。在有限的数组内,由于出队操作造成队列前部分可能有空间放下新元素,%
因此当一个指针想要向后移动一个单位时,如果已经达到数组的结尾,可以考虑前移到数列开头看看有没有可用的空间,我们称之{\kaishu 循环}。%
对于插入操作,后移和循环时要保证下一位置是一个空位。下图是循环队列操作的一个案例。


\begin{figure}[H]
      \centering
      \tikzstyle{c} = [minimum width = 8mm, minimum height = 10mm]
      \tikzstyle{token} = [draw = white, node distance = 1.2cm]
      \tikzstyle{arr} = [thick, ->]
\begin{tikzpicture}[every node/.style={draw}]
      \matrix (m1) [draw=white]
      {
            \node [c](1){1}; & \node [c](2){2}; & \node [c](3){3}; & \node [c](4){4}; & 
            \node [c](5){5}; & \node [c](6){6}; & \node [c](7){7}; & \node [c](8){ };\\
      };
      
      \node (f1)[token, above of = 1]{front};
      \node (r1)[token, above of = 7]{rear};
      \draw [arr] (f1) -- (1);
      \draw [arr] (r1) -- (7);
      
      \matrix (m2) [draw=white, below of = m1, node distance = 2.3cm]
      {
            \node [c](21){1}; & \node [c](22){2}; & \node [c](23){3}; & \node [c](24){4}; & 
            \node [c](25){5}; & \node [c](26){6}; & \node [c](27){7}; & \node [c](28){8};\\
      };
      
      \node (f2)[token, above of = 21]{front};
      \node (r2)[token, above of = 28]{rear};
      \draw [arr] (f2) -- (21);
      \draw [arr] (r2) -- (28);
      
      \matrix (m3) [draw=white, below of = m2, node distance = 2.3cm]
      {
            \node [c](31){ }; & \node [c](32){ }; & \node [c](33){3}; & \node [c](34){4}; & 
            \node [c](35){5}; & \node [c](36){6}; & \node [c](37){7}; & \node [c](38){8};\\
      };
      
      \node (f3)[token, above of = 33]{front};
      \node (r3)[token, above of = 38]{rear};
      \draw [arr] (f3) -- (33);
      \draw [arr] (r3) -- (38);

      \matrix (m4) [draw=white, below of = m3, node distance = 2.3cm]
      {
            \node [c](41){9}; & \node [c](42){ }; & \node [c](43){3}; & \node [c](44){4}; & 
            \node [c](45){5}; & \node [c](46){6}; & \node [c](47){7}; & \node [c](48){8};\\
      };
      
      \node (f4)[token, above of = 43]{front};
      \node (r4)[token, above of = 41]{rear};
      \draw [arr] (f4) -- (43);
      \draw [arr] (r4) -- (41);


\end{tikzpicture}
\caption{循环队列内部数据示意图\\}
\end{figure}


\subsection{代码实现}

{
\lstset{
      morendkeywords={Queue},
      morendkeywords = {size, capacity, front, rear, data}
}
\begin{enumerate}
      \item 入队
\begin{lstlisting} [firstnumber = 47, caption={Queue::enqueue}]
template<class Type>
bool Queue<Type>::enqueue(const Type &d) {
    tryExtend();                    // 保证有足够的空间
    data[rear] = d;
    rear = (rear+1) % capacity;     // 指针循环
    ++size;
    return true;
} \end{lstlisting}
      \item 出队
\begin{lstlisting} [firstnumber = 47, caption={Queue::dequeue}]
template<class Type>
bool Queue<Type>::dequeue(Type &d) {
    if (isEmpty())                  // 不能为空
        return false;
    d = data[front];
    front = (front+1) % capacity;   // 指针循环
    --size;
    tryShrink();                    // 尝试缩小空间
    return true;
}\end{lstlisting}
      \item 重分配空间。不允许数据丢失,重分配后将数据按顺序整理到开头。
\begin{lstlisting} [firstnumber = 85, caption={Queue::resize}]     
bool Queue<Type>::resize(int n) {
    if (n < size) {
        return false;               // 不允许数据丢失
    }
    if (n == capacity) {
        return true;                // 相等就不用分配
    }
    auto temp = new Type[n];        
    if (!isEmpty()) {               // 整理数据
        int j = 0;
        for (int i = front; i != rear; i = (i+1)%capacity) {
            temp[j++] = data[i];
        }
    }
    capacity = n; front = 0;
    rear = size;
    delete [] data; data = temp;
    return true;
} \end{lstlisting}
      \item 扩张数组:在数组将满的时候调用 \lstinline{resize}
\begin{lstlisting} [firstnumber = 109]
template<class Type>
void Queue<Type>::tryExtend() {
    if (capacity - size <= 1)       // 只剩一个空位
        resize(capacity * 2);       // 扩张到两倍
} \end{lstlisting}
      \item 缩小数组:在只有 $1/4$ 空间被有效利用时,尝试将数组缩减到一半大小。
\begin{lstlisting} [firstnumber = 116]
template<class Type>
void Queue<Type>::tryShrink() {
    if (isEmpty())                  // 空队恢复初始大小
        resize(InitSize);           
    else if (capacity > InitSize && capacity / 4 > size)
        resize(capacity / 2);       // 缩减大小
}
\end{lstlisting}
\end{enumerate}
}

\chapter{Bank 类}

封装了本题逻辑功能的一个类。

\section{API}

\subsection{类定义}
{
\lstset{morendkeywords={Bank}}
\begin{lstlisting}[firstnumber=154, caption=Bank 类定义]{}
class Bank {
public:
    explicit Bank(std::ostream& os) : ostream(os) { };
    ~Bank() = default;
    void addBusiness(const string& num);// 将 num 入队
    void handleBusiness();              // 决定顺序并出队
    void run(int num, std::istream &is);// 完成一题

private:
    Queue<string> windowA;
    Queue<string> windowB;
    std::ostream& ostream;

    enum Window { A, B };
    void handle(Window w);              // 出队的实际操作
    static bool isNumeric(const string& num);
                                        // 检查字符串
};
\end{lstlisting}


\subsection{代码实现}

\lstinline{Bank} 类最主要的功能就是封装了两个队列来完成操作,具体的逻辑与图\ref{yewu}一致,下面展示关键代码。

\begin{itemize}
      \item \lstinline{void Bank::addBusiness(long long num)} \\
            将一个数加入到 \textsl{A} 或 \textsl{B} 窗口的队列中。
\begin{lstlisting}[escapechar=^,]
void Bank::addBusiness(const string& num) {
    if (!isNumeric(num))
        throw BankException("Invalid character!"); ^\label{throw}^
    if ((num[num.length()-1] - '0') % 2) {
        windowA.enqueue(num);
    } else {
        windowB.enqueue(num);
    }
}
\end{lstlisting}
      \item \lstinline{void Bank::handleBusiness()}\\
            挨个处理这些业务直到结束。
\begin{lstlisting}
void Bank::handleBusiness() {
    while (!(windowA.isEmpty() && windowB.isEmpty())) {
        handle(A);
        handle(A);
        handle(B);
    }
    ostream << std::endl;
}
\end{lstlisting}
      \item \lstinline{Bank::handle(Bank::Window w)}\\
            处理业务的实际函数,参数化来实现封装。
\begin{lstlisting}
void Bank::handle(Bank::Window w) {
    string m;
    if (w == A) {
        if (windowA.dequeue(m)) {
            ostream << m << ' ';
        }
    } else {
        if (windowB.dequeue(m)) {
            ostream << m << ' ';
        }
    }
}
\end{lstlisting}
\end{itemize}

}