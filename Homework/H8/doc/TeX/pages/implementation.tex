
\chapter{类结构}
由于使用了哈希表,本项目的代码结构较为复杂,因此加以解释。\lstinline{HashTable} 类是仿照 \emph{C++ STL} 源码实现的,其中使用了%
\lstinline{Vector} 类,所以代码中出现在上面的类型(本应该被 \lstinline{#include} 进来)是下面类型的依赖,而不是更重要的意思。%
下面按照一定的逻辑顺序介绍这几个类:

\begin{itemize}
    \item \lstinline{Graph 和 Vertex} 类和最小生成树函数\\
    实现了图、顶点的封装。实现了Prim算法
    \item \lstinline{HashTable} 类\\
    其中维护了一个 \lstinline{HashNode} 的 \lstinline{Vector},通过其哈希函数(\lstinline{hash function})完成对\lstinline{Vector}%
    数组的索引。主要通过\lstinline{find} 和 \lstinline{hasKey}索引和判断键的存在与否,通过 \lstinline{insert} 方法插入一个键唯一的对象。%
    通过 \lstinline{erase} 方法抹去一个值。
    \item \lstinline{StringHashMap} 类\\
    包装和特化之后的一个哈希键值对集合,通过字符串进行索引得到对应值,范型参数为值的类型。可以通过\lstinline{find} 和 %
    \lstinline{hasKey}索引和判断键的存在与否,通过 \lstinline{insert} 方法插入一个键唯一的对象。%
    通过 \lstinline{erase} 方法抹去一个值。并且重载了下标运算符 \lstinline{[]},将字符串作为下标可以索引对应的值。
    \item \lstinline{Vector} 类\\
    \lstinline{HashTable} 类用于存放\lstinline{HashNode}的数组,相同哈希值的对象会以链表的方式存放在这个数组的单元里。%
    在\lstinline{HashTable} 类的实现中主要用到了下标运算、\lstinline{reserve(n)} 方法:开辟 $n$ 长度的数组。\lstinline{resize(n)}%
    方法:重新分配 $n$ 长度的内存。
    \item \lstinline{PowerSystem} 类\\
    封装了各种功能以及输入输出交互、算法。
\end{itemize}

该部分主要详细介绍\lstinline{Graph} 链表类和 \lstinline{HashTable} 哈希表类的实现,其余部分简单概括被使用的接口。


\chapter{Graph 类和 Vertex 类}

\section{API}

\subsection{类定义}
以下是本题使用顶点的 API。
\begin{lstlisting}[morendkeywords={Vertex}, firstnumber=810, caption=Vertex 类定义]{}
template<class Tag>
class Vertex {
public:
    static constexpr int INF = std::numeric_limits<int>::max();
    // 无穷大
    Vertex<Tag>() = default;
    explicit Vertex<Tag>(const Tag &t) : tag(t) {};

    Vertex<Tag> &find(const Tag &t) { // 找对应顶点
        for (int i = 0; i < adjacents.length(); ++i) {
            if (adjacents[i].tag == t) {
                return adjacents[i];
            }
        }
        throw GraphException("Adjacency not found");
    }

    int weightTo(const Tag &t) { // 到节点的权重
        for (int i = 0; i < adjacents.length(); ++i) {
            if (adjacents[i]->tag == t) {
                return weights[i];
            }
        }
        return INF;
    }

    int weightTo(const Vertex &v) {
        return weightTo(v.tag);
    }

    size_t index = 0;// 索引
    Tag tag;         // 名字
    Vector<Vertex *> adjacents;
    Vector<int> weights;
};
\end{lstlisting}

以下是本题使用图的 API。
\begin{lstlisting}[morendkeywords={Graph}, firstnumber=847, caption=Graph 类定义]{}
template<class VTag>
class Graph {
    template <class V> friend
    Graph<V> minSpanningTree(Graph<V> &graph, const V& start);
public:
    static constexpr int INF = std::numeric_limits<int>::max();
    // 无穷大
    Graph() = default;
    // 完善图的函数
    inline void addEdge(const VTag &u, const VTag &v, int weight);
    inline void addEdge(size_t u, size_t v, int weight);
    inline void addVertex(const VTag& tag);

    bool hasTag(const VTag& t) { return tagMap.hasKey(t); }
    unsigned nodes() const { return numVertex; }
    void reset() { vertexes.fill(Vertex<VTag>()); }
    inline std::ostream &show(std::ostream &os);
    inline std::ostream &showByVertexes(std::ostream &os);

private:
    size_t numVertex = 0;   // 顶点数
    Vector<Vertex<VTag> *> vertexes;
    HashMap<VTag, size_t> tagMap;
};
\end{lstlisting}



{
\lstset{
    basicstyle=\ttfamily\color{CPPDark}, 
    numberstyle=\tiny\color{darkgray},
    keywordstyle=\color[RGB]{40,40,255},
    }
\subsection{公开成员}
\begin{enumerate}
    \item \lstinline{void addEdge(size_t u, size_t v, int weight);}
          添加两个顶点(下标索引)之间的一条边。
    \item \lstinline{void addEdge(const VTag &u, const VTag &v, int weight);}
          添加两个顶点(顶点名索引)之间的一条边。
    \item \lstinline{void addVertex(const VTag& tag);} \\
          添加一个顶点。
\end{enumerate}

\subsection{私有成员}
\begin{enumerate}
    \item \lstinline{size_t VerTex::index = 0;} \\
          与 Graph 中该顶点被存在 vertexes 数组中的位置相对应。
    \item \lstinline{Vector<Vertex<VTag> *> Graph::vertexes;} \\
          图的顶点,下标是唯一指定可以用来索引的值。
    \item \lstinline{HashMap<VTag, size_t> Graph::tagMap;} \\
          用来将顶点名字与下标一一对应的map。
\end{enumerate}
}

\section{代码逻辑与实现}

{
\lstset{
      morendkeywords={LinkedList},
      morendkeywords = {head, tail, elemNum}
}
\begin{enumerate}
      \item 增加边
\begin{lstlisting} [firstnumber = 881, caption={Graph::addEdge}]
template<class VTag>
void Graph<VTag>::addEdge(size_t u, size_t v, int weight) {
    if (u == v) // self looping not allowed
        return;
    try {
        for (int i = 0; i < vertexes[u]->adjacents.length(); ++i) {
            if (vertexes[u]->adjacents[i]->index == v)
                cout << "Multiple edges between "
                     "two vertexes, making Multi-graph.\n";
        }
        vertexes[u]->adjacents.pushBack(vertexes[v]);
        vertexes[u]->weights.pushBack(weight);
        vertexes[v]->adjacents.pushBack(vertexes[u]);
        vertexes[v]->weights.pushBack(weight);
    } catch (IllegalVectorAccessing &e) {
        throw GraphException("Vertexes out of range");
    }
} \end{lstlisting}
      \item 增加边
\begin{lstlisting} [firstnumber = 829, caption={LinkedList::removeTarget}]
template<class VTag>
void Graph<VTag>::addVertex(const VTag &tag) {
    if (tagMap.hasKey(tag)) // 不允许同名
        throw GraphException("Same tag not allowed");
    auto v = new Vertex<VTag>(tag);
    v->index = vertexes.length();
    tagMap.insert(tag, vertexes.length());
    vertexes.pushBack(v);
    numVertex++;
} \end{lstlisting}
      
\end{enumerate}
}

\chapter{HashTable 类}

仿照了 C++ STL 的哈希表实现。

\section{API}

\subsection{类定义}
{
\lstset{morendkeywords={HashTable}}
\begin{lstlisting}[firstnumber=435, caption=HashTable 类定义]{}
template<class ValueType, class Key, class HashFunction,
        class ExtractKey, class EqualKey>
class HashTable {
public:
    explicit HashTable(size_t n);

    inline size_t maxBucketNum() const;      // 最大容量
    inline size_t bucketCount() const;       // 当前容量
    bool insertUnique(const ValueType& obj); // 插入值
    void clear();                            // 清空
    ValueType & find(const Key&);            // 用键查找
    ValueType & findOrInsert(const ValueType& obj);     // 查找或插入值
    bool hasKey(const Key& k);          // 查键记录
    inline bool empty() const;          // 是否为空
    size_t erase(const Key& key);       // 抹去一个键值
    void resize(size_t hint);           // 重分配容量

private:
    typedef HashNode<ValueType> node;   // 节点
    typedef HashFunction hasher;        // 哈希函数类
    typedef EqualKey equal_key;         // 判断等值类

    hasher hash;            // 哈希函数对象
    ExtractKey getKey;      // 获取键函数对象
    equal_key equals;       // 判断等值函数对象

    Vector<node *> buckets; // 存值的容器
    size_t num_elements;    // 元素个数
    static constexpr int num_primes = 28;    // 见解释
    static const unsigned long prime_list[num_primes];
        // 找质数、调用哈希函数
    inline unsigned long next_prime(unsigned long n);
    inline size_t findBucketKey(const Key &key, size_t size) const;
    inline size_t findBucket(const ValueType &obj) const;
    inline size_t findBucket(const ValueType &obj, size_t size) const;
};
\end{lstlisting}


\lstinline{HashTable} 有五个模板参数:\lstinline{ValueType} 为存放值类型,\lstinline{Key} 为键类型\footnote{这里的键、值关系并非%
键值对的关系,\lstinline{HashTable} 的值通过\lstinline{ExtractKey}类可以取出键,他们是包含关系而不是关联。},%
\lstinline{HashFunction} 可算出 \lstinline{Key}对象对应的哈希值, \lstinline{ExtractKey} 取出 \lstinline{ValueType} 中的键,%
\lstinline{EqualKey} 可以比较两键是否相等。

\section{代码逻辑与实现}

\subsection{逻辑描述}

哈希表是一个查找和添加元素平均开销均为 $O(1)$ 的数据结构,其特点是数据的存储方式为将数据分散在数组里,并以数组的下标来表征元素的特征。元素的值%
与下标之间的关系满足函数关系,也就是对于任意的元素取值都能找到哈希表中的唯一下标与之对应。由于数据的分散特征,哈希表又称{散列表}。在哈希表中查找一个%
数据可称作“索引”,数据一定要有{\kaishu 可哈希性(可散列性)},即该数据的{\kaishu 键(Key)}可以通过一个函数获得一个唯一个值,%
相同的键的函数值也相同。由于一般的哈希函数算出的值都是非负整数,很难保证不同键的哈希值不相同(如果这样就能构成一一对应关系),%
因此要实现哈希表还要解决{\kaishu 冲突}问题.

C++ STL 的哈希表采用了“开链”方式组织解决冲突。“开链”之意为在表的单元中存放链表的指针,并将应该放在该表中的元素用链表延伸出去。如图 \ref{hash} 所示,%
一个单元中有元素就存放首节点的指针,以单链表延伸;如果没有元素,就要存放空指针。这样,一个哈希表所有键冲突元素都可以存在一个单元中,并且查找某个哈希值对应的%
键值是否存在表中时,只需要单向遍历这个单元的链表就行。理想情况下,键与哈希值是均匀分布的,每个单元只存储一个键;最坏条件下,哈希表退化成链表,查找开销为 $O(n)$。

以下是流程图:

{
    \begin{figure}[H]
        \centering
        \begin{tikzpicture}
            \node (start) [startstop]  {开始};
            \node [io, below=.6cm of start] (input) {输入键 \lstinline{key}};
            \node [process, below=.6cm of input] (hash) {算出 \lstinline{key} 的哈希值 \lstinline{hash}};
            \node [process, below=.6cm of hash] (get) {\lstinline{hash} 下标位置的节点指针 \lstinline{p}};
            \node [condition, below=.6cm of get] (cond1) {\lstinline{p} 不为空};
            \node [condition, below=.6cm of cond1] (cond2) {\lstinline{p->key} != \lstinline{key}};
            \node [process, below=.6cm of cond2] (find) {返回\lstinline{p}节点的值};
            \node [process, below=.6cm of cond2, xshift = 4cm] (notFound) {返回空值};
            \node [startstop, below=.6cm of find] (end) {结束};

            \draw [arr] (start) -- (input);
            \draw [arr] (input) -- (hash);
            \draw [arr] (hash) -- (get);
            \draw [arr] (get) -- (cond1);
            \draw [arr] (cond1) -- node[anchor=east] {是} (cond2);
            \draw [arr] (cond2) -- node[anchor=east] {否} (find);
            \draw [arr] (find) -- (end);
            \draw [arr] (notFound.south) |- (end);

            \draw [arr] (cond1) -- ($(cond1.east) + (0.9, 0)$) node[anchor = south] {否} -| (notFound.north);
            \draw [arr] (cond2) -- ($(cond2.west) + (-0.5, 0)$) node[anchor = east] {是} |- ($ (cond1.north) + (0, 0.3) $);

        \end{tikzpicture}
        \caption{查找HashTable中的值}
    \end{figure}
}

{
    \begin{figure}[H]
        \centering
        \begin{tikzpicture}
            \node (start) [startstop]  {开始};
            \node [io, left=1.0cm of start] (input) {输入值 \lstinline{value}};
            \node [process, below=.6cm of input] (hash) {算出 \lstinline{value} 的哈希值 \lstinline{hash}};
            \node [process, below=.6cm of hash] (get) {\lstinline{hash} 下标位置的节点指针 \lstinline{p}};
            \node [process, below=.6cm of get] (find) {\lstinline{value} 生成节点插入 \lstinline{p} 链表的首位置};
            \node [startstop, right=1.0cm of find] (end) {结束};

            \draw [arr] (start) -- (input);
            \draw [arr] (input) -- (hash);
            \draw [arr] (hash) -- (get);
            \draw [arr] (get) -- (find);
            \draw [arr] (find) -- (end);

        \end{tikzpicture}
        \caption{添加值到HashTable}
    \end{figure}
}

另外,\lstinline{HashTable} 的数组是自动分配大小的,因此采用了 \lstinline{Vector} 类的 \lstinline{resize} 函数。%
\lstinline{HashTable} 扩充内存的逻辑是:当含有元素个数大于当前数组的大小(单元个数)时,就将容量扩大大约一倍。实际上 \lstinline{HashTable} %
中的数组大小取值为一系列呈两倍递增的质数,因为模取质数获得的下标能尽可能保证均匀。

\subsection{代码实现}

下面展示\lstinline{HashTable}关键代码。(省略了范型参数)

\begin{itemize}
      \item \lstinline{ValueType &HashTable<..>::find(const Key &k)}\\
            找到 \lstinline{key} 的值。
\begin{lstlisting} [firstnumber=598]
template<..>
ValueType &
HashTable<..>::
find(const Key &k) {
    size_t n = findBucketKey(k, buckets.length());
    node *first = buckets[n];   // 索引头节点
    if (first == nullptr) throw HashTableException("Key Not Exist");
    for (node *cur = first; cur; cur = cur->next) {
        if (equals(getKey(cur->value), k)) {
            return cur->value;  // 找到
        }
    } // 没找到
    throw HashTableException("Key Not Exist");
}\end{lstlisting}
      \item \lstinline{bool HashTable<..>::insertUnique(const ValueType &obj);} \\
            插入一个值。
\begin{lstlisting}[escapechar=^, firstnumber=561]
template<..>
bool
HashTable<..>::
insertUnique(const ValueType &obj) {
    resize(num_elements + 1);   // 检查是否扩充
    const size_t pos = findBucket(obj);
    node *first = buckets[pos]; // 用哈希值索引

    for(node *cur = first; cur; cur = cur->next) {
        if (equals(getKey(obj), getKey(cur->value)))
            return false;       // 重复不添加
    }
    // 添加
    auto temp = new node(obj);
    temp->next = first;
    buckets[pos] = temp;
    ++num_elements;
    return true;
}\end{lstlisting}
\end{itemize}


\section{StringHashMap 类}

用 \lstinline{HashTable} 存放了 \lstinline{Pair<std::string, Type>} 类型的封装,可以用与存放键值对,通过字符串键索引值。%
因为都是调用 \lstinline{HashTable} 的方法,这里只展示类定义 API。


\subsection{Pair 类定义}
\lstset{morendkeywords={Pair}}
\begin{lstlisting}[firstnumber=699, caption=Pair 类定义]{}
template<class First, class Second>
struct Pair {
    typedef First firstType;
    typedef Second SecondType;

    First first;
    Second second;

    Pair() : first(First()), second(Second()) {}
    Pair(const First &t1, const Second &t2) : first(t1), second(t2) {}
    Pair(const Pair &p) : first(p.first), second(p.second) { }

    Pair &operator=(const Pair& p) { // 重载等号
        if (&p == this)
            return *this;
        first = p.first; second = p.second;
        return *this;
    }
};
\end{lstlisting}

\subsection{HashMap 类定义}
\lstset{morendkeywords={Pair}}
\begin{lstlisting}[firstnumber=734, caption=HashMap 类定义]{}
template <class Key, class Value>
class HashMap {
public:
    typedef Pair<Key, Value> MapPair;
    HashMap() : ht(50) { }
    size_t size() const { return ht.bucketCount(); }
    size_t maxSize() const { return ht.maxBucketNum(); }
    bool empty() const { return ht.empty(); }

    inline Value &operator[](const Key &key);
    MapPair &find(const Key &key) { return ht.find(key); }
    const MapPair &find(const Key &key) const { return ht.find(key); }
    bool hasKey(const Key &key) { return ht.hasKey(key); }
    size_t erase(const Key& key) { return ht.erase(key); }
    void resize(size_t hint) { ht.resize(hint); }
    inline void insert(const Key& s, const Value& val);
    inline void insert(const MapPair& pair);

private:// 用模板参数具体化 HashTable
    HashTable<
            MapPair,
            Key,
            Hash<Key>,
            GetPairKey<MapPair, Key>,
            Equals<Key> > ht;
};
\end{lstlisting}

}


\chapter{System 类}

一个处理逻辑和与外界交互的封装类,存放学生信息并用上述结构完成了题目要求的功能。

\subsection{PowerSystem 类定义}
\lstset{morendkeywords={Pair}}
\begin{lstlisting}[firstnumber=986, caption=System 类定义]{}
class PowerSystem {
    enum Command {
        CreatVertex, CreatEdge, BuildTree, ShowTree, ShowGrid, ShowMenu, Exit
    };
public:
    void run();

private:
    Graph<string> graph;
    Graph<string> minSpanTree;
    bool treeOK = false;

    bool parseCommand(const string &c);

    void createVertex();
    void createEdge();
    void buildTree();
    void showTree();
    void showGrid();
    static void showMenu();

    static void clearIStream(std::istream &is);
};
\end{lstlisting}

封装的类涉及了添加节点、添加边、生成树、展示树、展示图、展示目录等功能。将在下一部分展示目录和功能。