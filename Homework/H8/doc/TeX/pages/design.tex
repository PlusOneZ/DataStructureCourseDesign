
\chapter{背景分析}

20世纪以后,人们普遍认识到扩大电力系统的规模可以在能源开发、工业布局、负荷调整、系统安全与经济运行等方面带来明显的社会经济效益。

电网搭建是国家基础建设的一部分,电能供应不足或供电不可靠都会影响国民经济的发展,甚至造成严重的经济损失;%
发电和输、配电能力过剩又意味着电力投资效益降低,从而影响发电成本。因此,必须进行电力系统的全面规划,%
以提高发展电力系统的预见性和科学性。

因此如果能够依靠科学规划来减少送电途中的电能损失、减少电缆用材,可以使送电工程的成本降低,以便用相同的预算覆盖更大面积%
的居住区,造福百姓。如何减少电缆架设的成本是一个值得探讨的问题。

如果可以将电缆搭设的模型简化成下面一张图,每个点代表城市中的某个小区,相连的边上的数字表示搭设相连小区之间的电缆的开销,%
我们可以用图论相关的算法来求解这个问题。

\begin{figure}[H]
    \tikzstyle{house} = [triangle, fill=blue, minimum width=8mm]
    \tikzstyle{link} = [thick]
    \tikzset {
        triangle/.style={
    isosceles triangle,
    isosceles triangle apex angle=60,
    rotate=90,
    anchor=west,
  },
    }
    \centering
    \begin{tikzpicture}
        \node [house, label=A] at (-3, 3)  (A) { };
        \node [house, label=B] at (-2, 0)  (B) { };
        \node [house, label=C] at (-1, -2) (C) { };
        \node [house, label=D] at (1, 2)   (D) { };
        \node [house, label=below:E] at (2, -2)  (E) { };
        \node [house, label=right:F] at (3, 0)   (F) { };

        \draw [link ] (A) -- node[anchor=north east] {8} (B);
        \draw [link ] (B) -- node[anchor=north east] {7} (C);
        \draw [link ] (B) -- node[anchor=south] {1} (D);
        \draw [link ] (C) -- node[anchor=north east] {9} (E);
        \draw [link ] (C) -- node[anchor=south east] {20} (F);
        \draw [link ] (A) -- node[anchor=north east] {5} (E);
        \draw [link ] (B) -- node[anchor=south] {3} (F);
        \draw [link ] (E) -- node[anchor=north west] {15} (F);
        \draw [link ] (D) -- node[anchor=north east] {1} (F);

    \end{tikzpicture}
    \caption{电网连接花费模型}
    \label{net}
\end{figure}


\chapter{功能设计}

\section{题意转化}

题目描述为:
{
\begin{quote}
    \kaishu
假设一个城市有$n$个小区,要实现$n$个小区之间的电网都能够相互接通,构造这个城市$n$个小区之间的电网,使总工程造价最低。
请设计一个能够满足要求的造价方案。

项目功能要求:
在每个小区之间都可以设置一条电网线路,都要付出相应的经济代价。$n$个小区之间最多可以有%
$\frac{n(n-1)}{2}$条线路,选择其中的$n-1$条使总的耗费最少。
    
\end{quote}
}

依据题意,要计算产生这$n$个小区之间的{\kaishu 最小生成树}。最小生成树是一个图的、包含所有节点、边权重和最小的子图。%
图\ref{net}就可以求出一个最小生成树:

\begin{figure}[H]
    \tikzstyle{house} = [triangle, fill=blue, minimum width=8mm]
    \tikzstyle{link} = [thick]
    \tikzset {
        triangle/.style={
    isosceles triangle,
    isosceles triangle apex angle=60,
    rotate=90,
    anchor=west,
  },
    }
    \centering
    \begin{tikzpicture}
        \node [house, label=A] at (-3, 3)  (A) { };
        \node [house, label=B] at (-2, 0)  (B) { };
        \node [house, label=C] at (-1, -2) (C) { };
        \node [house, label=D] at (1, 2)   (D) { };
        \node [house, label=below:E] at (2, -2)  (E) { };
        \node [house, label=right:F] at (3, 0)   (F) { };

        \draw [link ] (A) -- node[anchor=north east] {8} (B); 
        \draw [link ] (B) -- node[anchor=north east] {7} (C); 
        \draw [link ] (B) -- node[anchor=south] {1} (D); 
        \draw [link ] (A) -- node[anchor=north east] {5} (E); 
        \draw [link ] (D) -- node[anchor=north east] {1} (F); 

    \end{tikzpicture}
    \caption{电网最低造价连接示意图}

\end{figure}

直观的看,这个图是图\ref{net}的连接方式去掉了一些造价较高的边且保留了造价低的边。但实际算法中用“一边删除边,一边查看是否%
保持图连通”的做法实现起来不太方便,问题在“删除的同时查看连通”这一个步骤。有两个最小生成树的算法:%
\emph{Kruskal 算法}和\emph{Prim 算法}。

此外,因为需要将输入的节点名称对应一个节点储存起来,就需要使用快速查找的结构,本题引入了{\kaishu 哈希Map} 作为查找的数据结构,%
哈希表是哈希 Map的底层数据结构:

哈希表的插入和删除时间复杂度均可看作 $O(1)$ 级别,其特点是空间复杂度为 $\Omega(n)$\footnote{$\Omega$ 为下界},其中 $n$ 为存储%
在表内的数据数量。本项目中,哈希表存放的是{\kaishu <顶点名, 顶点下标>}的键值对。

{
\begin{figure}[H]
    \tikzset{
        listnode/.style={
            rectangle split,rectangle split parts=2,draw,rectangle split horizontal,fill=blue!20
        },
        startnode/.style={
            draw,minimum width=1.5cm,minimum height=.75cm
        }
    }
    \centering
    \begin{tikzpicture}[scale=.2, >=stealth]
        \node[startnode] (t0) {$T[0]$};
        \node[startnode,below=0pt of t0] (t1) {$T[1]$};
        \node[startnode,below=0pt of t1] (t2) {$T[2]$};
        \node[startnode,below=0pt of t2] (t3) {$T[3]$};
        \node[startnode,below=0pt of t3] (t4) {$T[4]$};
        \node[startnode,below=0pt of t4] (t5) {$T[5]$};
        \node[listnode,right=of t0] (3) {3};
        \node[listnode,right=of t1] (1) {1};
        \node[listnode,right=of t2] (2) {2};
        \node[listnode,right=of t5] (13) {13};
        \node[listnode,right=.5cm of 2] (5) {5};
        \node[listnode,right=.5cm of 5] (8) {8};
        \node[right=.5cm of 3] (3x) {$\varnothing$};
        \node[right=.5cm of 1] (7x) {$\varnothing$};
        \node[right=.5cm of 8] (8x) {$\varnothing$};
        \node[right=1cm of t3] (3xx) {$\varnothing$};
        \node[right=1cm of t4] (4xx) {$\varnothing$};
        \node[right=.5cm of 13] (13x) {$\varnothing$};
        \draw[*->] let \p1 = (3.two), \p2 = (3.center) in (\x1,\y2) -- (3x);
        \draw[*->] let \p1 = (1.two), \p2 = (1.center) in (\x1,\y2) -- (7x);
        \draw[*->] let \p1 = (2.two), \p2 = (2.center) in (\x1,\y2) -- (5);
        \draw[*->] let \p1 = (5.two), \p2 = (5.center) in (\x1,\y2) -- (8);
        \draw[*->] let \p1 = (8.two), \p2 = (8.center) in (\x1,\y2) -- (8x);
        \draw[*->] let \p1 = (13.two), \p2 = (13.center) in (\x1,\y2) -- (13x);
        \draw[->] (t0) edge (3) (t1) edge (1) (t2) edge (2) (t3) edge (3xx) (t4) edge (4xx) (t5) edge (13);
    \end{tikzpicture}
        \caption{哈希表结构示意图\label{hash}}
\end{figure}
}

\tikzstyle{startstop} = [rectangle, rounded corners, minimum width = 2cm, 
        minimum height=1cm, text centered, draw = black, fill = red!40,
        font = {\bfseries}]
\tikzstyle{io} = [trapezium, trapezium left angle=70, trapezium right angle=110, 
    minimum width=2cm, minimum height=1cm, text centered, draw=black, fill = blue!40]
\tikzstyle{process} = [rectangle, minimum width=3cm, minimum height=1cm, text centered, draw=black, fill = yellow!50]
\tikzstyle{condition} = [diamond, aspect = 3, text centered, draw=black, fill = green!30]
\tikzstyle{arr} = [->, >=stealth, thick]

\section{逻辑功能}
以下是 {\kaishu Prim 算法} 的逻辑流程图:

\newpage
{
    \begin{figure}[H]
        \centering
        \begin{tikzpicture}
            \node (start) [startstop]  {开始};
            \node [io, below=.7cm of start] (input) {输入开始顶点 $V_0$};
            \node [process, below=.6cm of input] (inheap) {$V_0$ 的边入堆};
            \node [condition, below=1cm of inheap] (cond) {堆为空?};
            \node [process, below=.7cm of cond] (min) {取堆最小边 $E_{min}$};
            \node [condition, below=.6cm of min] (cond2) {$E_{min}$ 有一端不在树中};
            \node [process, below=.6cm of cond2] (inheap2) {$E_{min}$ 入树,邻边入堆};
            \node [condition, below=.6cm of inheap2] (cond3) {树中有 $n$个顶点?};
            \node [startstop, below=.6cm of cond3] (end) {结束};

            \draw [arr] (start) -- (input);
            \draw [arr] (input) -- (inheap);
            \draw [arr] (inheap) -- (cond);
            \draw [arr] (cond) --  node[anchor=west] {否} (min);
            \draw [arr] (min) -- (cond2);
            \draw [arr] (cond2) -- node[anchor=west] {是} (inheap2);
            \draw [arr] (inheap2) -- (cond3);
            \draw [arr] (cond3) -- node[anchor=east] {是} (end);

            \draw [arr] (cond) -- ($(cond.east) + (1.5, 0)$) node[anchor=south] {是} |- ($(end.north) + (0, 0.3)$);
            \draw [arr] (cond2) -- ($(cond2.west) + (-0.8, 0)$) node[anchor=north] {否} |- ($(cond.north) + (0, 0.3)$);
            \draw [arr] (cond3) -- ($(cond3.west) + (-1.5, 0)$) node[anchor=north] {否} |- ($(cond.north) + (0, 0.3)$);

        \end{tikzpicture}
        \caption{Prim 逻辑}
    \end{figure}
}