
\chapter{异常}
在输入时,不合法的字符会引起一些异常来指示当前程序的状态。通过处理这些异常可以改变程序的执行序列,在出错之前%
将程序的执行权交给处理错的代码。本例的数据结构和系统都有异常类来提供必要的信息。其中一个例子是:
\begin{lstlisting}[firstnumber=244, caption=HashTableException 异常类]
class HashTableException : public std::exception {
public:
    explicit HashTableException(const char* s) : content(s) { }
    const char *what() const _NOEXCEPT override {
        return content;
    };

    const char* content;
};
\end{lstlisting}

异常在合适的代码位置被抛出,但应该被处理,否则用户就会因为未处理的异常在使用程序时遇到软件终止的情况。%
异常的设计是:该代码段不应该处理该问题或是处理问题时会带来额外的副作用,就应该在此处抛出异常。本项目的%
异常基本都得到了处理。


\lstdefinestyle{console} {
    language=HTML,
    backgroundcolor=\color{CPPDark},   
    commentstyle=\color{CPPLight},
    keywordstyle=\color{red},
    ndkeywordstyle={},
    morekeywords={Unfinished, input, Invalid, character},
    numbers=none,
    stringstyle={},
    basicstyle=\fira\color{white}\footnotesize,
    breaklines=true,                                     
    keepspaces=true,                                     
    showspaces=false,                
    showstringspaces=false,
    showtabs=false,                  
    tabsize=2,
    escapechar=\%,
    identifierstyle={},
}

\chapter{数据测试}

\section{正常情况}
\begin{enumerate}
    \item 创建系统数据表
\begin{lstlisting} [style=console]
    Constructing Registering System!
    Enter amount of candidates: 4             -$
    Enter candidate's ID, name, sex, age and type of examination one by one: 
    #1: 1234 alice female 20 SSE              -$
    #2: 2345 Bob male 20 CS                   -$
    #3: 3456 zzy m 20 DataStructure           -$
    #4: 4567 miao pet 5 EatAndSleep           -$
    Unidentified sex pet modified to 'unknown'
    
    Total 4 students added
    No ID       Name        Sex     Age     Type of Examination
    1. 1234     alice       female  20      SSE
    2. 2345     Bob         male    20      CS
    3. 3456     zzy         male    20      DataStructure
    4. 4567     miao        unknown 5       EatAndSleep
\end{lstlisting}
可以看出,不合法的性别会被识别成 unknown。

    \item 目录
\begin{lstlisting}[style=console]
    |***** Menu of Commands *****|
    |  1: Insert      2: Delete  |
    |  3: Find        4: Edit    |
    |  5: Statistics  0: Exit    |
\end{lstlisting}

    \item 添加成员
\begin{lstlisting}[style=console]
    Enter your command, 6 for menu: 1       -$
    Enter the position of candidate: 1      -$
    Enter candidate's ID, name, sex, age and type of examination: 
    0123 Oscar m 21 Social                  -$
    No ID       Name        Sex      Age     Type of Examination
    1. 0123     Oscar       male     21      Social
    2. 1234     alice       female   20      SSE
    3. 2345     Bob         male     20      CS
    4. 3456     zzy         male     20      DataStructure
    5. 4567     miao        unknown  5       EatAndSleep
\end{lstlisting}

    \item 删除(按位置)
\begin{lstlisting}[style=console]
    Enter your command, 6 for menu: 2       -$
    Choose to remove by [1. Position, 2. Candidate ID]: 
    1                                       -$
    Enter candidate's position to remove: 4 -$
    The candidate removed is: 
    3456       zzy          male     20      DataStructure
    No ID      Name         Sex      Age     Type of Examination
    1. 0123    Oscar        male     21      Social
    2. 1234    alice        female   20      SSE
    3. 2345    Bob          male     20      CS
    4. 4567    miao         unknown  5       EatAndSleep
\end{lstlisting}

    \item 插入(按考号)
\begin{lstlisting}[style=console]
    Enter your command, 6 for menu: 2       -$
    Choose to remove by [1. Position, 2. Candidate ID]: 
    2                                       -$
    Enter candidate's ID to remove: 1234    -$
    The candidate removed is: 
    1234       alice        female   20      SSE
    No ID      Name         Sex      Age     Type of Examination
    1. 0123    Oscar        male     21      Social
    2. 2345    Bob          male     20      CS
    3. 4567    miao         unknown  5       EatAndSleep
\end{lstlisting}

    \item 用考号查找学生
\begin{lstlisting}[style=console]
    Enter your command, 6 for menu: 3       -$
    Enter candidate's ID to find: 2345      -$
    2345     Bob         male    20       CS
\end{lstlisting}

    \item 修改学生信息(按位置)
\begin{lstlisting}[style=console]
    Enter your command, 6 for menu: 4       -$
    Choose to edit by [1. Position, 2. Candidate ID]: 
    1                                       -$
    Enter candidate's position to edit: 3   -$
    Enter candidate's ID, name, sex, age and type of examination: 
    4567 miao m 16 EatAndWalk               -$
    No ID       Name        Sex      Age     Type of Examination
    1. 0123     Oscar       male     21      Social
    2. 2345     Bob         male     20      CS
    3. 4567     miao        male     16      EatAndWalk
\end{lstlisting}

    \item 统计
\begin{lstlisting}[style=console]
    Enter your command, 6 for menu: 5       -$
    Male: 3   Female: 0    Unknown: 0
    No ID       Name        Sex      Age     Type of Examination
    1. 0123     Oscar       male     21      Social
    2. 2345     Bob         male     20      CS
    3. 4567     miao        male     16      EatAndWalk
\end{lstlisting}

    \item 结束
\begin{lstlisting}[style=console]
    Enter your command, 6 for menu: 0       -$
    Bye~
\end{lstlisting}
\end{enumerate}

\vspace*{3cm}

\section{输入异常}

\begin{enumerate}
    \item 创建系统数据表时直接中断
\begin{lstlisting} [style=console]
    Constructing Registering System!
    Enter amount of candidates: 4           -$
    Enter candidate's ID, name, sex, age and type of examination one by one: 
    #1: 1234 alice female 20 SSE            -$
    #2: 2345 Bob male 20 CS                 -$
    #3:                                     -$ // 输入回车
    
    Total 2 students added
    No ID       Name        Sex       Age       Type of Examination
    1. 1234     alice       female    20        SSE
    2. 2345     Bob         male      20        CS
\end{lstlisting}
    用回车可以中断输入。

    \item 插入输入不存在的位置
\begin{lstlisting}[style=console]
    Enter your command, 6 for menu: 1       -$
    Enter the position of candidate: 6      -$
    Position should be between 1 to 5       -$
    Try again: // ...
\end{lstlisting}
    位置不存在会提醒重新输入。

    \item 删除:位置不存在
\begin{lstlisting}[style=console]
    Enter your command, 6 for menu: 2       -$
    Choose to remove by [1. Position, 2. Candidate ID]: 
    6                                       -$
    Try again, enter 1 / 2: 
    1                                       -$
    Enter candidate's position to remove: 6 -$
    Invalid Index
    Position 6 not found!

    Enter your command, 6 for menu:  // ...
\end{lstlisting}
    功能选择错误会提醒重新输入,位置不存在会有提示并退出。

\item 删除:考号不存在
\begin{lstlisting} [style=console]
    Enter your command, 6 for menu: 2       -$
    Choose to remove by [1. Position, 2. Candidate ID]: 
    2                                       -$
    Enter candidate's ID to remove: 0000    -$
    Candidate with ID 0000 not found!

    Enter your command, 6 for menu: // ...
\end{lstlisting}
    会给提示并退出。

    \item 查找:找不到考号
\begin{lstlisting} [style=console]
    Enter your command, 6 for menu: 3
    Enter candidate's ID to find: 
    0000
    Candidate with ID 0000 not found
\end{lstlisting}
    会给提示并退出。

    \item 修改学生信息:位置或考号不存在
\begin{lstlisting}[style=console]
    Enter your command, 6 for menu: 4       -$
    Choose to edit by [1. Position, 2. Candidate ID]: 
    1                                       -$
    Enter candidate's position to edit: 6   -$
    Invalid Index
    Position 6 not found!
    Enter your command, 6 for menu: 4       -$
    Choose to edit by [1. Position, 2. Candidate ID]: 
    2                                       -$
    Enter candidate's ID to edit: 0000      -$
    Candidate with ID 0000 not found!
\end{lstlisting}
    与删除类似。

\end{enumerate}