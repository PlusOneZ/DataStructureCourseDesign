
\chapter{背景分析}

考试报名工作给各高校报名工作带来了新的挑战,给教务管理部门增加了很大的工作量。随着信息时代的来临,手动输入管理考生的信息效率低下%
且容易出错,管理起来十分不便。如何快速地处理大规模的数据以及频繁的查询,是当今考试报名系统需要解决的关键问题。

使用使用计算机技术来进行考试信息的维护和管理,其具有管理容量上限高,管理快速,成本低等多个巨大优势,能极大提升考试报名管理的效率。%
本项目是对考试报名管理系统的简单模拟,系统主要有以下功能:输出考生信息、查询考生信息、添加考生信息、修改考生信息和删除考生信息。

系统需要维护一个内存中的报名表,表中包含的信息有准考证号,姓名,性别,年龄和报考类别等。并能够通过准考证号查询到考生,指定修改信息、%
删除等功能。




\chapter{功能设计}

\section{题意转化}
依据题意,需要使用一个数据结构将学生信息存储起来,并能做到{\kaishu 增、删、查、改}四个功能。有许多数据结构都能做到这项功能%
简单的有、{\kaishu 线性表},如数组、链表等结构,复杂的有哈希表,以及二叉查找树、平衡二叉树等非线性结构。%
本题使用了{\kaishu 双链表}作为主要的数据结构储存信息,双链表的数据组织如下图。

\newcommand{\data}{
      data \nodepart{second}
      \phantom{\texttt{NULL}}
    }

\begin{figure}[H]
    \tikzstyle{ptr}  = [draw, -{Stealth[scale=1.0]}, blue]
    \tikzstyle{head} = [rectangle, draw, text height=3mm, text width=3mm,
                        text centered, node distance=3cm, inner sep=0pt]
    \tikzstyle{data} = [rectangle split, rectangle split parts=3, draw, node distance=2.5cm, 
                        text centered, minimum height=3em]
    \centering
    \begin{tikzpicture} [node distance=2cm, auto]
    \node[data, node distance=2.5cm] (A) {data \nodepart{third} \footnotesize{nullptr}};
    \node[data, right of=A] (B) {\data};
    \node[data, right of=B] (C) {\data};
    \node[data, right of=C] (D) {\data};
    \node[data, right of=D] (E) {\data};
    \node[data, right of=E, node distance=2.5cm] (last) {data \nodepart{second} \footnotesize{nullptr}};
    
    \draw[fill] ($(A.south)!0.67!(A.text split)$) circle (0.05);
    \draw[ptr] ($(A.south)!0.67!(A.text split)$) --++(right:10mm) |- ($(B.south west)!0.67!(B.text split west)$);
    \draw[fill] ($(B.south)!0.67!(B.text split)$) circle (0.05);
    \draw[ptr] ($(B.south)!0.67!(B.text split)$) --++(right:10mm) |- ($(C.south west)!0.67!(C.text split west)$);
    \draw[fill] ($(C.south)!0.67!(C.text split)$) circle (0.05);
    \draw[ptr] ($(C.south)!0.67!(C.text split)$) --++(right:10mm) |- ($(D.south west)!0.67!(D.text split west)$);
    \draw[fill] ($(D.south)!0.67!(D.text split)$) circle (0.05);
    \draw[ptr] ($(D.south)!0.67!(D.text split)$) --++(right:10mm) |- ($(E.south west)!0.67!(E.text split west)$);
    \draw[fill] ($(E.south)!0.67!(E.text split)$) circle (0.05);
    \draw[ptr] ($(E.south)!0.67!(E.text split)$) --++(right:10mm) |- ($(last.south west)!0.67!(last.text split west)$);

    \draw[fill] ($(B.south)!0.3!(B.text split)$) circle (0.05);
    \draw[ptr] ($(B.south)!0.3!(B.text split)$) --++(left:10mm) |- ($(A.south east)!0.3!(A.text split east)$);
    \draw[fill] ($(C.south)!0.3!(C.text split)$) circle (0.05);
    \draw[ptr] ($(C.south)!0.3!(C.text split)$) --++(left:10mm) |- ($(B.south east)!0.3!(B.text split east)$);
    \draw[fill] ($(D.south)!0.3!(D.text split)$) circle (0.05);
    \draw[ptr] ($(D.south)!0.3!(D.text split)$) --++(left:10mm) |- ($(C.south east)!0.3!(C.text split east)$);
    \draw[fill] ($(E.south)!0.3!(E.text split)$) circle (0.05);
    \draw[ptr] ($(E.south)!0.3!(E.text split)$) --++(left:10mm) |- ($(D.south east)!0.3!(D.text split east)$);
    \draw[fill] ($(last.south)!0.3!(last.text split)$) circle (0.05);
    \draw[ptr] ($(last.south)!0.3!(last.text split)$) --++(left:10mm) |- ($(E.south east)!0.3!(E.text split east)$);
    
    
    \end{tikzpicture}
    \caption{双链表结构示意图}
\end{figure}

通过单向访问节点可以增加、修改、删除一个节点,这三个操作的时间复杂度均为 $O(1)$ 级别的,而查找某个节点的时间复杂度是 $O(n)$ 级别的。%
由于查找操作在一个考试系统中较为频繁的出现,所以在用链表储存有序的考生名单的基础上,我决定加用{\kaishu 哈希表}\emph{(Hash Table)}辅助进行查找,%
哈希表的插入和删除时间复杂度均可看作 $O(1)$ 级别,其特点是空间复杂度为 $\Omega(n)$\footnote{$\Omega$ 为下界},其中 $n$ 为存储%
在表内的数据数量。本项目中,哈希表存放的是{\kaishu <考号, 节点指针>}的键值对。

{
\begin{figure}[H]
    \tikzset{
        listnode/.style={
            rectangle split,rectangle split parts=2,draw,rectangle split horizontal,fill=blue!20
        },
        startnode/.style={
            draw,minimum width=1.5cm,minimum height=.75cm
        }
    }
    \centering
    \begin{tikzpicture}[scale=.2, >=stealth]
        \node[startnode] (t0) {$T[0]$};
        \node[startnode,below=0pt of t0] (t1) {$T[1]$};
        \node[startnode,below=0pt of t1] (t2) {$T[2]$};
        \node[startnode,below=0pt of t2] (t3) {$T[3]$};
        \node[startnode,below=0pt of t3] (t4) {$T[4]$};
        \node[startnode,below=0pt of t4] (t5) {$T[5]$};
        \node[listnode,right=of t0] (3) {3};
        \node[listnode,right=of t1] (1) {1};
        \node[listnode,right=of t2] (2) {2};
        \node[listnode,right=of t5] (13) {13};
        \node[listnode,right=.5cm of 2] (5) {5};
        \node[listnode,right=.5cm of 5] (8) {8};
        \node[right=.5cm of 3] (3x) {$\varnothing$};
        \node[right=.5cm of 1] (7x) {$\varnothing$};
        \node[right=.5cm of 8] (8x) {$\varnothing$};
        \node[right=1cm of t3] (3xx) {$\varnothing$};
        \node[right=1cm of t4] (4xx) {$\varnothing$};
        \node[right=.5cm of 13] (13x) {$\varnothing$};
        \draw[*->] let \p1 = (3.two), \p2 = (3.center) in (\x1,\y2) -- (3x);
        \draw[*->] let \p1 = (1.two), \p2 = (1.center) in (\x1,\y2) -- (7x);
        \draw[*->] let \p1 = (2.two), \p2 = (2.center) in (\x1,\y2) -- (5);
        \draw[*->] let \p1 = (5.two), \p2 = (5.center) in (\x1,\y2) -- (8);
        \draw[*->] let \p1 = (8.two), \p2 = (8.center) in (\x1,\y2) -- (8x);
        \draw[*->] let \p1 = (13.two), \p2 = (13.center) in (\x1,\y2) -- (13x);
        \draw[->] (t0) edge (3) (t1) edge (1) (t2) edge (2) (t3) edge (3xx) (t4) edge (4xx) (t5) edge (13);
    \end{tikzpicture}
        \caption{哈希表结构示意图\label{hash}}
\end{figure}
}

\vspace*{1.0cm}

\section{逻辑功能}
{\kaishu 增、删}功能对应于双链表结构中的{\kaishu 插入、删除}功能,哈希表优化后的{\kaishu 查找}功能通过%
哈希表的{\kaishu 查询(或索引)}功能实现。{\kaishu 修改}功能则是查找和赋值的结合。四种操作都十分简单。

但由于考试报名系统中维护了一个顺序的双链表和一个哈希表,在{\kaishu 增、删、改}这三个需要改变数据的操作中,%
在修改链表数据的{\kaishu 同时}一起修改哈希表当中的数据是十分必要的。因此,逻辑重点在于同时维护两套数据并使系统正常工作。

\section{维护两套数据}

\tikzstyle{startstop} = [rectangle, rounded corners, minimum width = 2cm, 
        minimum height=1cm, text centered, draw = black, fill = red!40,
        font = {\bfseries}]
    \tikzstyle{io} = [trapezium, trapezium left angle=70, trapezium right angle=110, 
    minimum width=2cm, minimum height=1cm, text centered, draw=black, fill = blue!40]
    \tikzstyle{process} = [rectangle, minimum width=3cm, minimum height=1cm, text centered, draw=black, fill = yellow!50]
    \tikzstyle{condition} = [diamond, aspect = 3, text centered, draw=black, fill = green!30]
    \tikzstyle{arr} = [->, >=stealth, thick]

\subsection{增加信息}
增加操作是最简单的操作,流程图如下:
{
    \begin{figure}[H]
        \centering
        \begin{tikzpicture}
            \node (start) [startstop]  {开始};
            \node [io, below=.7cm of start] (input) {读取到 \lstinline{S}};
            \node [condition, below=1cm of input] (cond) {\lstinline{S} 考号已经存在?};
            \node [process, below=.7cm of cond] (toList) {\lstinline{S} 加入链表};
            \node [process, below=.6cm of toList] (toMap) {\lstinline{S} 加入哈希表};
            \node [startstop, below=.6cm of toMap] (end) {结束};

            \draw [arr] (start) -- (input);
            \draw [arr] (input) -- (cond);
            \draw [arr] (cond) -- node[anchor=west] {否} (toList);
            \draw [arr] (toList) -- (toMap);
            \draw [arr] (toMap) -- (end);

            \draw [arr] (cond) -- ($(cond.east) + (0.5, 0)$) node[anchor = west] {是} |- ($ (cond.north) + (0, 0.5) $);

        \end{tikzpicture}
        \caption{添加考生信息逻辑图\label{add}}
    \end{figure}
}


\subsection{删除信息}
删除与增加类似,流程图如下:
{
    \begin{figure}[H]
        \centering
        \begin{tikzpicture}
            \node (start) [startstop]  {开始};
            \node [io, below=.7cm of start] (input) {输入考号 \lstinline{num}};
            \node [condition, below=.7cm of input] (cond) {\lstinline{num} 考号存在?};
            \node [process, left=1.3cm of cond] (toList) {\lstinline{S} 从链表删除};
            \node [process, below=.6cm of toList] (toMap) {\lstinline{S} 从哈希表抹去};
            \node [io, right=1.3cm of toMap] (out) {输出出考生信息};
            \node [startstop, below=1cm of out] (end) {结束};

            \draw [arr] (start) -- (input);
            \draw [arr] (input) -- (cond);
            \draw [arr] (cond) -- node[anchor=north] {是} (toList);
            \draw [arr] (toList) -- (toMap);
            \draw [arr] (toMap) -- (out);
            \draw [arr] (out) -- (end);

            \draw [arr] (cond) -- ($(cond.east) + (0.9, 0)$) node[anchor = west] {否} |- ($ (end.north) + (0, 0.5) $);

        \end{tikzpicture}
        \caption{删除考生信息逻辑图\label{del}}
    \end{figure}
}

\subsection{查找考生信息}
通过调用哈希表的查询方法可以很快的找到考生并输出信息,流程图如下:

{
    \begin{figure}[H]
        \centering
        \begin{tikzpicture}
            \node (start) [startstop]  {开始};
            \node [io, below=.7cm of start] (input) {输入考号 \lstinline{num}};
            \node [condition, below=.7cm of input] (cond) {\lstinline{S} 考号存在?};
            \node [process, below=.7cm of cond, xshift=-2.75cm] (find) {找到考生};
            \node [io, below=.7cm of find] (out) {输出考生信息};
            \node [io, below=.7cm of cond, xshift=2.75cm] (outErr) {输出错误提示};
            \node [startstop, below=.6cm of out, xshift=2.5cm] (end) {结束};

            \draw [arr] (start) -- (input);
            \draw [arr] (input) -- (cond);
            \draw [arr] (cond) -- ($(cond.west) + (-0.5, 0)$) node[anchor=east] {是} -- (find.north);
            \draw [arr] (find) -- (out);
            \draw [arr] (out) -- (end);
            \draw [arr] (outErr) -- (end);

            \draw [arr] (cond) -- ($(cond.east) + (0.5, 0)$) node[anchor = west] {否} -- (outErr.north);

        \end{tikzpicture}
        \caption{查询考生信息逻辑图\label{find}}
    \end{figure}
}

\subsection{修改考生信息}
修改考生信息除了要先找到考生之外,还有一些对哈希表操作的细节。流程图如下:

{
    \begin{figure}[H]
        \centering
        \begin{tikzpicture}
            \node (start) [startstop]  {开始};
            \node [io, below=.7cm of start] (input) {输入考号 \lstinline{num}};
            \node [process, below=.7cm of input] (find) {用 \lstinline{num} 查询考生 \lstinline{S}};
            \node [condition, below=.7cm of find] (cond) {\lstinline{S} 不为空?};
            \node [process, below=.8cm of cond] (erase) {\lstinline{S} 从哈希表抹去};
            \node [io, below=.7cm of erase] (input2) {读取到 \lstinline{S}};
            \node [condition, below=.6cm of input2] (cond2) {\lstinline{S} 考号已经存在?};
            \node [process, below=.8cm of cond2] (add) {\lstinline{S} 加入哈希表};
            \node [startstop, below=.8cm of add] (end) {结束};

            \draw [arr] (start) -- (input);
            \draw [arr] (input) -- (find);
            \draw [arr] (find) -- (cond);
            \draw [arr] (cond) -- node[anchor=east] {是} (erase);
            \draw [arr] (erase) -- (input2);
            \draw [arr] (input2) -- (cond2);
            \draw [arr] (cond2) --  node[anchor=east] {否} (add);
            \draw [arr] (add) -- (end);

            \draw [arr] (cond) -- ($(cond.east) + (0.9, 0)$) node[anchor = west] {否} |- ($ (end.north) + (0, 0.5) $);
            \draw [arr] (cond2) -- ($(cond2.west) + (-0.5, 0)$) node[anchor = east] {是} |- ($ (cond2.north) + (0, 0.3) $);

        \end{tikzpicture}
        \caption{修改考生信息逻辑图\label{mod}}
    \end{figure}
}

重点在于哈希表存放了{\kaishu <考号, 节点指针>}的键值对,所以在修改时如果要判断修改后考号是否重复,就要先将修改前考号的键值对抹去,这样在检查是否%
重复考号时的哈希表才是有效的。修改完成后再将该{\kaishu <考号, 节点指针>}对加入回哈希表,完成一次修改。