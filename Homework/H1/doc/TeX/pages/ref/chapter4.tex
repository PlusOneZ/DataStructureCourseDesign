% Chapter 4

\chapter{引用与链接}\label{chap1}
使用 \lstinline{cleveref} 宏包的 \lstinline{\cref} 命令进行引用,将会自动检测环境并添加相应的前缀。

\section{使用定理}
\begin{theorem}[(这是中文括号副标题)]\label{sec:the1}
	这是一条跟随\textsf{章}编号的定理,本质为计数器的功能,引用形式例如\ref{sec:the1}。
\end{theorem}
\begin{definition}[\space(This is the English bracket subtitle)]
	这是一个定义。
\end{definition}
\begin{lemma}
	这是一个引理。
\end{lemma}
\begin{corollary}
	这是一个推论。
\end{corollary}
\begin{proposition}
	这是一个性质。
\end{proposition}
\begin{example}
	这是一个例。
\end{example}
\begin{remark}
	这是一个注。
\end{remark}
\begin{proof}\pushQED{\qed}
	这是一个证明,末尾添加证明结束符。\popQED
\end{proof}

\section{脚注}
脚注应在注释与正文之间加细线分隔,使用 \lstinline{footmisc} 宏包设置,选项为悬挂缩进(\lstinline{hang})形式,也可设置为首行缩进的形式\footnote{脚注示例文字,若要设置为有缩进的形式将 \lstinline{hang} 选项及 \lstinline{footnotemargin} 参数去掉即可。\par 这是分段将会缩进两字符。}。

\section{引用章节}\label{sec1}
如引用\cref{chap1}、\cref{sec1}。

\section{引用参考文献}
这是一个参考文献引用的范例\cite{1979Prospect}。引用多个文献,将引用标号中的多个文献序号按升序排列,若其中有3个以上的连续序号,则改用范围序号,例如\cite{1979Prospect,2004Text,1979Intracellular}。

\section{索引}
使用 \lstinline{imakeidx} 生成索引,仅设置了索引题目并将索引加入目录,其他选项需另外扩展配置,可阅读宏包相关文档。