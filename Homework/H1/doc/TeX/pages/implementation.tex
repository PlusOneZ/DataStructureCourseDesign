
\chapter{类结构}
由于使用了哈希表,本项目的代码结构较为复杂,因此加以解释。\lstinline{HashTable} 类是仿照 \emph{C++ STL} 源码实现的,其中使用了%
\lstinline{Vector} 类,所以代码中出现在上面的类型(本应该被 \lstinline{#include} 进来)是下面类型的依赖,而不是更重要的意思。%
下面按照一定的逻辑顺序介绍这几个类:

\begin{itemize}
    \item \lstinline{LinkedList} 类\\
    链表类,存放顺序的学生名单结构。有 \lstinline{insert, remove, forEach} 方法用于插入、删除、遍历操作。
    \item \lstinline{HashTable} 类\\
    其中维护了一个 \lstinline{HashNode} 的 \lstinline{Vector},通过其哈希函数(\lstinline{hash function})完成对\lstinline{Vector}%
    数组的索引。主要通过\lstinline{find} 和 \lstinline{hasKey}索引和判断键的存在与否,通过 \lstinline{insert} 方法插入一个键唯一的对象。%
    通过 \lstinline{erase} 方法抹去一个值。
    \item \lstinline{StringHashMap} 类\\
    包装和特化之后的一个哈希键值对集合,通过字符串进行索引得到对应值,范型参数为值的类型。可以通过\lstinline{find} 和 %
    \lstinline{hasKey}索引和判断键的存在与否,通过 \lstinline{insert} 方法插入一个键唯一的对象。%
    通过 \lstinline{erase} 方法抹去一个值。并且重载了下标运算符 \lstinline{[]},将字符串作为下标可以索引对应的值。
    \item \lstinline{Vector} 类\\
    \lstinline{HashTable} 类用与存放\lstinline{HashNode}的数组,相同哈希值的对象会以链表的方式存放在这个数组的单元里。%
    在\lstinline{HashTable} 类的实现中主要用到了下标运算、\lstinline{reserve(n)} 方法:开辟 $n$ 长度的数组。\lstinline{resize(n)}%
    方法:重新分配 $n$ 长度的内存。
    \item \lstinline{Student, System} 类\\
    \lstinline{Student} 类是学生信息的封装,\lstinline{System} 类是交互系统实现,通过 \lstinline{run} 函数来开启,%
    内部封装了指令显示、读取以及指令实现的函数。
\end{itemize}

该部分主要详细介绍\lstinline{LinkedList} 链表类和 \lstinline{HashTable} 哈希表类的实现,其余部分简单概括被使用的接口。


\chapter{LinkedList 类}

\section{API}

\subsection{类定义}
以下是本题使用队列的 API,各函数的功能已用注释标出。
\begin{lstlisting}[morendkeywords={LinkedList}, firstnumber=637, caption=LinkedList 类定义]{}
template <class Type>
class LinkedList {
public:

    struct LinkNode {           // 嵌套节点类
        LinkNode *prev = nullptr;
        LinkNode *next = nullptr;
        Type data;

        LinkNode() = default;
        explicit LinkNode(const Type &);
    };

    typedef LinkNode* Link;           // 别名

    LinkedList();                     // 构造器
    LinkedList(LinkedList const &);
    ~LinkedList();

    int length() const;               // 大小
    bool isEmpty() const;             // 是否为空
    Link getHead() const;             // 获取首节点
    Link getTail() const;             // 获取尾节点
    int search(const Type& x) const;  //搜索和谓词搜索
    int searchByMethod(std::function<bool(Type&)> m);
    Link locate(int i) const;         // 取索引
    Link insert(int i, const Type& d);// 插入
    Link insert(int i, Link link);    
    Link insertTarget(Link tar, Link link);
    Link insertBack(const Type& d);   // 后端插入
    bool remove(int i, Type& d);      // 删除
    bool remove(int i, Link link = nullptr);
    bool removeTarget(Link link, Type& d);

    void forEach(std::function<void(Type&)>);
                                      // 对所有节点操作 

protected:
    Link head;
    Link tail;
    int elemNum = 0;

    bool makeNode(Link &) const;      // 分配新节点
    bool makeNode(Link &, const Type& x) const;
};
\end{lstlisting}
\vspace*{1cm}

{
\lstset{
    basicstyle=\ttfamily\color{CPPDark}, 
    numberstyle=\tiny\color{darkgray},
    keywordstyle=\color[RGB]{40,40,255},
    }
\subsection{公开成员}
\begin{enumerate}
    \item \lstinline{Link insertTarget(Link tar, Link link);}
          插入在 \lstinline{tar} 后面。返回插入节点的指针。该函数有两个托管函数,通过位置 $i$ 来索引节点。
    \item \lstinline{bool removeTarget(Link link, Type& d);}
          删除 \lstinline{tar} 节点。返回是否成功。该函数有两个托管函数,通过位置 $i$ 来索引节点。
    \item \lstinline{Link locate(int i) const;} \\
          从位置$i$取出一个节点,如果没有节点则报错并返回 \lstinline{nullptr}。
    \item \lstinline{int search(const Type& x) const;} \\
          遍历找到值为 \lstinline{x} 的节点的位置,找不到返回-1。
    \item \lstinline{int searchByMethod(std::function<bool(Type&)> method);} \\
          返回第一个 \lstinline{method} 函数返回值为 \lstinline{true} 的节点的位置,找不到返回-1.
    \item \lstinline{void forEach(std::function<void(Type&)> op);} \\
          对每个元素都调用操作 \lstinline{op},可以用于格式打印等操作。
\end{enumerate}

\subsection{私有成员}
\begin{enumerate}
    \item \lstinline{Link head; Link tail;} \\
          该链表的头节点和尾节点。
    \item \lstinline{int elemNum = 0;} \\
          链表中现有元素的个数。
    \item \lstinline{bool makeNode(Link &, const Type& x) const;} \\
          用 $x$ 的值来分配并初始化一个新的节点,失败返回 \lstinline{nullptr}。
\end{enumerate}
}

\section{代码逻辑与实现}

\subsection{逻辑描述}


双链表中,最主要改变其结构的动作是 {\kaishu 插入} 和 {\kaishu 删除},因此要说明这两个部分的内容。

插入节点时,先找到这个节点的前驱(后继向前走一格便是前驱),然后做按照顺序对目标节点操作:

\begin{itemize}
    \item 目标节点的后继指针指向前驱的后继,目标节点的前驱指针指向前驱;
    \item 前驱的后继指针指向目标;
    \item 后继的前驱指针指向目标。
\end{itemize}

删除节点时,先找到这个目标节点,然后按照顺序对目标节点操作:
\begin{itemize}
    \item 目标的前驱的后继指针指向目标的后继。
    \item 目标的后继的前驱指针指向目标的前驱。
    \item 释放目标的内存。
\end{itemize}


\subsection{代码实现}

{
\lstset{
      morendkeywords={LinkedList},
      morendkeywords = {head, tail, elemNum}
}
\begin{enumerate}
      \item 插入到目标后
\begin{lstlisting} [firstnumber = 792, caption={LinkedList::insertTarget}]
template<class Type>
typename LinkedList<Type>::Link
LinkedList<Type>::insertTarget(LinkedList::Link target, LinkedList::Link link) {
    if (target == nullptr) {
        return nullptr;
    }
    link->next = target->next;
    link->next->prev = link;
    target->next = link;
    link->prev = target;
    ++elemNum;
    return link;
} \end{lstlisting}
      \item 删除目标节点
\begin{lstlisting} [firstnumber = 829, caption={LinkedList::removeTarget}]
template<class Type>
bool LinkedList<Type>::removeTarget(Link link, Type& d) {
    d = link->data;
    link->prev->next = link->next;
    link->next->prev = link->prev;
    try {
        delete link;
    } catch (std::exception& e) {
        return false;
    }
    --elemNum;
    return true;
}\end{lstlisting}
      
\end{enumerate}
}

\chapter{HashTable 类}

仿照了 C++ STL 的哈希表实现。

\section{API}

\subsection{类定义}
{
\lstset{morendkeywords={HashTable}}
\begin{lstlisting}[firstnumber=283, caption=HashTable 类定义]{}
template<class ValueType, class Key, class HashFunction,
        class ExtractKey, class EqualKey>
class HashTable {
public:
    explicit HashTable(size_t n);

    inline size_t maxBucketNum() const;      // 最大容量
    inline size_t bucketCount() const;       // 当前容量
    bool insertUnique(const ValueType& obj); // 插入值
    void clear();                            // 清空
    ValueType & find(const Key&);            // 用键查找
    ValueType & findOrInsert(const ValueType& obj);     // 查找或插入值
    bool hasKey(const Key& k);          // 查键记录
    inline bool empty() const;          // 是否为空
    size_t erase(const Key& key);       // 抹去一个键值
    void resize(size_t hint);           // 重分配容量

private:
    typedef HashNode<ValueType> node;   // 节点
    typedef HashFunction hasher;        // 哈希函数类
    typedef EqualKey equal_key;         // 判断等值类

    hasher hash;            // 哈希函数对象
    ExtractKey getKey;      // 获取键函数对象
    equal_key equals;       // 判断等值函数对象

    Vector<node *> buckets; // 存值的容器
    size_t num_elements;    // 元素个数
    static constexpr int num_primes = 28;    // 见解释
    static const unsigned long prime_list[num_primes];
        // 找质数、调用哈希函数
    inline unsigned long next_prime(unsigned long n);
    inline size_t findBucketKey(const Key &key, size_t size) const;
    inline size_t findBucket(const ValueType &obj) const;
    inline size_t findBucket(const ValueType &obj, size_t size) const;
};
\end{lstlisting}


\lstinline{HashTable} 有五个模板参数:\lstinline{ValueType} 为存放值类型,\lstinline{Key} 为键类型\footnote{这里的键、值关系并非%
键值对的关系,\lstinline{HashTable} 的值通过\lstinline{ExtractKey}类可以取出键,他们是包含关系而不是关联。},%
\lstinline{HashFunction} 可算出 \lstinline{Key}对象对应的哈希值, \lstinline{ExtractKey} 取出 \lstinline{ValueType} 中的键,%
\lstinline{EqualKey} 可以比较两键是否相等。

\section{代码逻辑与实现}

\subsection{逻辑描述}

哈希表是一个查找和添加元素平均开销均为 $O(1)$ 的数据结构,其特点是数据的存储方式为将数据分散在数组里,并以数组的下标来表征元素的特征。元素的值%
与下标之间的关系满足函数关系,也就是对于任意的元素取值都能找到哈希表中的唯一下标与之对应。由于数据的分散特征,哈希表又称{散列表}。在哈希表中查找一个%
数据可称作“索引”,数据一定要有{\kaishu 可哈希性(可散列性)},即该数据的{\kaishu 键(Key)}可以通过一个函数获得一个唯一个值,%
相同的键的函数值也相同。由于一般的哈希函数算出的值都是非负整数,很难保证不同键的哈希值不相同(如果这样就能构成一一对应关系),%
因此要实现哈希表还要解决{\kaishu 冲突}问题.

C++ STL 的哈希表采用了“开链”方式组织解决冲突。“开链”之意为在表的单元中存放链表的指针,并将应该放在该表中的元素用链表延伸出去。如图 \ref{hash} 所示,%
一个单元中有元素就存放首节点的指针,以单链表延伸;如果没有元素,就要存放空指针。这样,一个哈希表所有键冲突元素都可以存在一个单元中,并且查找某个哈希值对应的%
键值是否存在表中时,只需要单向遍历这个单元的链表就行。理想情况下,键与哈希值是均匀分布的,每个单元只存储一个键;最坏条件下,哈希表退化成链表,查找开销为 $O(n)$。

以下是流程图:

{
    \begin{figure}[H]
        \centering
        \begin{tikzpicture}
            \node (start) [startstop]  {开始};
            \node [io, below=.6cm of start] (input) {输入键 \lstinline{key}};
            \node [process, below=.6cm of input] (hash) {算出 \lstinline{key} 的哈希值 \lstinline{hash}};
            \node [process, below=.6cm of hash] (get) {\lstinline{hash} 下标位置的节点指针 \lstinline{p}};
            \node [condition, below=.6cm of get] (cond1) {\lstinline{p} 不为空};
            \node [condition, below=.6cm of cond1] (cond2) {\lstinline{p->key} != \lstinline{key}};
            \node [process, below=.6cm of cond2] (find) {返回\lstinline{p}节点的值};
            \node [process, below=.6cm of cond2, xshift = 4cm] (notFound) {返回空值};
            \node [startstop, below=.6cm of find] (end) {结束};

            \draw [arr] (start) -- (input);
            \draw [arr] (input) -- (hash);
            \draw [arr] (hash) -- (get);
            \draw [arr] (get) -- (cond1);
            \draw [arr] (cond1) -- node[anchor=east] {是} (cond2);
            \draw [arr] (cond2) -- node[anchor=east] {否} (find);
            \draw [arr] (find) -- (end);
            \draw [arr] (notFound.south) |- (end);

            \draw [arr] (cond1) -- ($(cond1.east) + (0.9, 0)$) node[anchor = south] {否} -| (notFound.north);
            \draw [arr] (cond2) -- ($(cond2.west) + (-0.5, 0)$) node[anchor = east] {是} |- ($ (cond1.north) + (0, 0.3) $);

        \end{tikzpicture}
        \caption{查找HashTable中的值}
    \end{figure}
}

{
    \begin{figure}[H]
        \centering
        \begin{tikzpicture}
            \node (start) [startstop]  {开始};
            \node [io, left=1.0cm of start] (input) {输入值 \lstinline{value}};
            \node [process, below=.6cm of input] (hash) {算出 \lstinline{value} 的哈希值 \lstinline{hash}};
            \node [process, below=.6cm of hash] (get) {\lstinline{hash} 下标位置的节点指针 \lstinline{p}};
            \node [process, below=.6cm of get] (find) {\lstinline{value} 生成节点插入 \lstinline{p} 链表的首位置};
            \node [startstop, right=1.0cm of find] (end) {结束};

            \draw [arr] (start) -- (input);
            \draw [arr] (input) -- (hash);
            \draw [arr] (hash) -- (get);
            \draw [arr] (get) -- (find);
            \draw [arr] (find) -- (end);

        \end{tikzpicture}
        \caption{添加值到HashTable}
    \end{figure}
}

另外,\lstinline{HashTable} 的数组是自动分配大小的,因此采用了 \lstinline{Vector} 类的 \lstinline{resize} 函数。%
\lstinline{HashTable} 扩充内存的逻辑是:当含有元素个数大于当前数组的大小(单元个数)时,就将容量扩大大约一倍。实际上 \lstinline{HashTable} %
中的数组大小取值为一系列呈两倍递增的质数,因为模取质数获得的下标能尽可能保证均匀。

\subsection{代码实现}

下面展示\lstinline{HashTable}关键代码。(省略了范型参数)

\begin{itemize}
      \item \lstinline{ValueType &HashTable<..>::find(const Key &k)}\\
            找到 \lstinline{key} 的值。
\begin{lstlisting} [firstnumber=446]
template<..>
ValueType &
HashTable<..>::
find(const Key &k) {
    size_t n = findBucketKey(k, buckets.length());
    node *first = buckets[n];   // 索引头节点
    if (first == nullptr) throw HashTableException("Key Not Exist");
    for (node *cur = first; cur; cur = cur->next) {
        if (equals(getKey(cur->value), k)) {
            return cur->value;  // 找到
        }
    } // 没找到
    throw HashTableException("Key Not Exist");
}\end{lstlisting}
      \item \lstinline{bool HashTable<..>::insertUnique(const ValueType &obj);} \\
            插入一个值。
\begin{lstlisting}[escapechar=^, firstnumber=409]
template<..>
bool
HashTable<..>::
insertUnique(const ValueType &obj) {
    resize(num_elements + 1);   // 检查是否扩充
    const size_t pos = findBucket(obj);
    node *first = buckets[pos]; // 用哈希值索引

    for(node *cur = first; cur; cur = cur->next) {
        if (equals(getKey(obj), getKey(cur->value)))
            return false;       // 重复不添加
    }
    // 添加
    auto temp = new node(obj);
    temp->next = first;
    buckets[pos] = temp;
    ++num_elements;
    return true;
}\end{lstlisting}
\end{itemize}


\section{StringHashMap 类}

用 \lstinline{HashTable} 存放了 \lstinline{Pair<std::string, Type>} 类型的封装,可以用与存放键值对,通过字符串键索引值。%
因为都是调用 \lstinline{HashTable} 的方法,这里只展示类定义 API。


\subsection{Pair 类定义}
\lstset{morendkeywords={Pair}}
\begin{lstlisting}[firstnumber=551, caption=Pair 类定义]{}
template<class First, class Second>
struct Pair {
    typedef First firstType;
    typedef Second SecondType;

    First first;
    Second second;

    Pair() : first(First()), second(Second()) {}
    Pair(const First &t1, const Second &t2) : first(t1), second(t2) {}
    Pair(const Pair &p) : first(p.first), second(p.second) { }

    Pair &operator=(const Pair& p) { // 重载等号
        if (&p == this)
            return *this;
        first = p.first; second = p.second;
        return *this;
    }
};
\end{lstlisting}

\subsection{StringHashMap 类定义}
\lstset{morendkeywords={Pair}}
\begin{lstlisting}[firstnumber=586, caption=StringHashMap 类定义]{}
template<class Value>
class StringHashMap {
public:
    typedef Pair<string, Value> MapPair; // 别名

    StringHashMap() : ht(50) { }
    // 委托调用
    size_t size() const { return ht.bucketCount(); }
    size_t maxSize() const { return ht.maxBucketNum(); }
    bool empty() const { return ht.empty(); }
    
    inline Value &operator[](const string &s);
    MapPair &find(const string &s) { return ht.find(s); }
    const MapPair &find(const string &s) const { return ht.find(s); }
    bool hasKey(const string &s) { return ht.hasKey(s); }
    size_t erase(const string& key) { return ht.erase(key); }
    void resize(size_t hint) { ht.resize(hint); }
    inline void insert(const string& s, const Value& val);
    inline void insert(const MapPair& pair);


private:    // 用模板参数具体化 HashTable
    HashTable<
            MapPair,
            string,
            Hash<string>,
            GetPairKey<MapPair, string>,
            Equals<string> > ht;
};
\end{lstlisting}

}


\chapter{System 类}

一个处理逻辑和与外界交互的封装类,存放学生信息并用上述结构完成了题目要求的功能。

\subsection{System 类定义}
\lstset{morendkeywords={Pair}}
\begin{lstlisting}[firstnumber=986, caption=System 类定义]{}
class System {
    typedef LinkedList<Student>::Link Link;
public:
    enum Command { kQuit, kInsert, kDelete, kFind, kModify, kStatistic, kMenu };
    System();
    void run();

protected:
    static void menu(); // 展示目录
    int readCommand();  // 读指令
    void insert();      // 插入
    void remove();      // 删除
    void find();        // 查找
    void modify();      // 修改
    void statistic();   // 统计
    void show(int option = 0);//显示
    void statAdd(const Student &s);
    void statSub(const Student &s);
                        // 统计用
    LinkedList<Student> list;
    StringHashMap<Link> hashMap;
    long long maleCnt = 0;
    long long femaleCnt = 0;
};
\end{lstlisting}

封装的具体代码涉及到许多繁琐的输入输出操作,这里不予展示,可以参照图\ref{add}、\ref{del}、\ref{find}、\ref{mod}的内容%
理解其工作逻辑。