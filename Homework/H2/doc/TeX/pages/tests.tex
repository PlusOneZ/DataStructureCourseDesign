
\chapter{异常}
在输入时,不合法的字符会引起一些异常来指示当前程序的状态。通过处理这些异常可以改变程序的执行序列,在出错之前%
将程序的执行权交给处理错的代码。本例的数据结构和系统都有异常类来提供必要的信息。其中一个例子是非数字异常:
\begin{lstlisting}[firstnumber=244, caption=NaNException 异常类]
class NaNException : public std::exception {
public:
    const char *what() const noexcept override {
        return "Not a Number!";
    }
};
\end{lstlisting}

异常在合适的代码位置被抛出,但应该被处理,否则用户就会因为未处理的异常在使用程序时遇到软件终止的情况。%
异常的设计是:该代码段不应该处理该问题或是处理问题时会带来额外的副作用,就应该在此处抛出异常。本项目的%
异常基本都得到了处理。


\lstdefinestyle{console} {
    language=HTML,
    backgroundcolor=\color{CPPDark},   
    commentstyle=\color{CPPLight},
    keywordstyle=\color{red},
    ndkeywordstyle={},
    morekeywords={Unfinished, input, Invalid, character},
    numbers=none,
    stringstyle={},
    basicstyle=\fira\color{white}\footnotesize,
    breaklines=true,                                     
    keepspaces=true,                                     
    showspaces=false,                
    showstringspaces=false,
    showtabs=false,                  
    tabsize=2,
    escapechar=\%,
    identifierstyle={},
}

\chapter{数据测试}

\section{正常情况}
\begin{enumerate}
    \item 一般情况
\begin{lstlisting} [style=console]
    Enter the sorted sequence A end with -1: 
    1 2 5 -1
    Enter the sorted sequence B end with -1: 
    2 4 5 8 10 -1
    Intersection is: 2 5 
    Continue? [Y/n]:  // ...
\end{lstlisting}

    \item 交集为空
\begin{lstlisting} [style=console]
    Enter the sorted sequence A end with -1: 
    1 3 5 -1
    Enter the sorted sequence B end with -1: 
    2 4 6 8 10 -1
    Empty set.
\end{lstlisting}    

    \item 完全相交
\begin{lstlisting} [style=console]
    Enter the sorted sequence A end with -1: 
    1 2 3 4 5 -1
    Enter the sorted sequence B end with -1: 
    1 2 3 4 5 -1
    Intersection is: 1 2 3 4 5 
\end{lstlisting}    

    \item 其中一个序列完全属于交集
\begin{lstlisting} [style=console]
    Enter the sorted sequence A end with -1: 
    3 5 7 -1
    Enter the sorted sequence B end with -1: 
    2 3 4 5 6 7 8 -1
    Intersection is: 3 5 7 
\end{lstlisting}

    \item 其中一个序列为空
\begin{lstlisting} [style=console]
    Enter the sorted sequence A end with -1: 
    -1
    Enter the sorted sequence B end with -1: 
    10 100 1000 -1
    Empty set.
\end{lstlisting}

\end{enumerate}

\vspace*{3cm}

\section{输入异常}

\begin{enumerate}

    \item 输入小于 $-1$ 的负数
\begin{lstlisting} [style=console]
    Enter the sorted sequence A end with -1: 
    1 2 3 4 -1
    Enter the sorted sequence B end with -1: 
    -2 -1 0 1
    NaN or negative number except for -1 in sequence, abandoning...
    
    Try again: 
    4 5 6 -1
    Intersection is: 4 
\end{lstlisting}
    会有警告并要求再次尝试,之前该序列的输入全部废弃。

    \item 输入非数字字符
\begin{lstlisting} [style=console]
    Enter the sorted sequence A end with -1: 
    a b c -1
    NaN or negative number except for -1 in sequence, abandoning...
    
    Try again: 
    1 2 3 -1
    Enter the sorted sequence B end with -1: 
    1+3+4 9 10 -1
    NaN or negative number except for -1 in sequence, abandoning...
    
    Try again: 
    3 4 5 -1
    Intersection is: 3 
\end{lstlisting}
会有警告并要求再次尝试,之前该序列的输入全部废弃。

    \item 输入序列非增序
\begin{lstlisting} [style=console]
    Enter the sorted sequence A end with -1: 
    10 9 8 7 -1
    Not increasing sequence, abandoning...
    Try again: 
    5 6 7 -1
    Enter the sorted sequence B end with -1: 
    3 4 5 -1
    Intersection is: 5 
\end{lstlisting}
会有警告并要求再次尝试,之前该序列的输入全部废弃。

\item 重复数字
\begin{lstlisting} [style=console]
    Continue? [Y/n]: Y
    Enter the sorted sequence A end with -1: 
    1 1 1 1 -1
    Enter the sorted sequence B end with -1: 
    1 1 2 2 -1
    Intersection is: 1 1 
\end{lstlisting}
    只有两个序列中都重复的数字才会在交集中重复相应次数。

\end{enumerate}