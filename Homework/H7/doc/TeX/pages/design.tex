
\chapter{背景分析}

农夫要修理牧场的一段栅栏,他测量了栅栏,发现需要$N$块木头,每块木头长度为整数$L_i$个%
长度单位,于是他购买了一个很长的,能锯成$N$块的木头,即该木头的长度是$L_i$的总和。

但是农夫自己没有锯子,请人锯木的酬金跟这段木头的长度成正比。为简单起见,不妨就%
设酬金等于所锯木头的长度。例如,要将长度为$20$的木头锯成长度为$8$,$7$和$5$的三段,第一次%
锯木头将木头锯成$12$和$8$,花费$20$;第二次锯木头将长度为$12$的木头锯成$7$和$5$花费$12$,%
总花费$32$元。如果第一次将木头锯成$15$和$5$,则第二次将木头锯成$7$和$8$,那么总的花费是$35$(大于$32$).


\chapter{功能设计}

\section{题意转化}

切割一根木头,每个目标木头长度的开销都至少计算一次,而有些木头的长度可能被计算多次。%
所以总体上来看需要避免对长的木头进行多次操作。而尽可能地通过把短的木头锯多次来达到目的。%

自底向上看,怎么让短木头被锯多次而长木头锯少次,重点在于怎么将目标木头两两组合成更长的目标木头%
作为中间产物计算。每次将最短的两根木头进行结合,将结合后的木头算入目标中,再取最短两根,这样的做法%
可以使越短的木头使用次数越多,而长的木头则会更迟结合,更少被使用。

所以项目需要一个可以维持元素顺序的数据结构,满足每次取出元素时都能取出最小的元素。{\kaishu 最小堆}可以满足这种特性,%
因此本项目就是用了最小堆这一数据结构。

\forestset{ 
  default preamble={ 
    for tree={circle,draw, minimum width=0.8cm} 
  } 
} 

\begin{figure}[H]
    \centering
    \begin{forest}
        [9 [17 [23 [53] [31]] [45] ] [65 [78] [87]]]
    \end{forest}
    \caption{最小堆示意}
\end{figure}

最小堆的结构特点是:所含有的节点元素大小\textbf{小于}自己的子女节点。

这个结构特点通过小于关系的传递性传递到根节点,于是根节点是整棵树中元素值最小的节点。%
文档将在实现部分描述堆的存储和实现。


\section{逻辑功能}

切木头问题的解法十分简单:将所有木头长度加入最小堆,每次取出两个最小元素相加得到长度 $l$,总花费 $cost += l$,%
并将 $l$ 压入堆中,再重复取出两个元素和之后的动作,直到堆中只有一个元素,$cost$ 就是需要的总花费。

