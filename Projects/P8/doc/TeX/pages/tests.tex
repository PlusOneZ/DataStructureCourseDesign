
\chapter{异常}
在输入时,不合法的字符会引起一些异常来指示当前程序的状态。通过处理这些异常可以改变程序的执行序列,在出错之前%
将程序的执行权交给处理错的代码。本例的数据结构和系统都有异常类来提供必要的信息。其中一个例子是:
\begin{lstlisting}[firstnumber=244, caption=HashTableException 异常类]
class HashTableException : public std::exception {
public:
    explicit HashTableException(const char* s) : content(s) { }
    const char *what() const _NOEXCEPT override {
        return content;
    };

    const char* content;
};
\end{lstlisting}

异常在合适的代码位置被抛出,但应该被处理,否则用户就会因为未处理的异常在使用程序时遇到软件终止的情况。%
异常的设计是:该代码段不应该处理该问题或是处理问题时会带来额外的副作用,就应该在此处抛出异常。本项目的%
异常基本都得到了处理。


\lstdefinestyle{console} {
    language=HTML,
    backgroundcolor=\color{CPPDark},   
    commentstyle=\color{CPPLight},
    keywordstyle=\color{red},
    ndkeywordstyle={},
    morekeywords={Unfinished, input, Invalid, character},
    numbers=none,
    stringstyle={},
    basicstyle=\fira\color{white}\footnotesize,
    breaklines=true,                                     
    keepspaces=true,                                     
    showspaces=false,                
    showstringspaces=false,
    showtabs=false,                  
    tabsize=2,
    escapechar=\%,
    identifierstyle={},
}

\chapter{数据测试}

\section{正常情况}
\begin{enumerate}
    \item 开始目录
\begin{lstlisting} [style=console]
    Welcome to Power Network Cost Simulating System, this is menu:
    ***************************************************
    **        Power Network Cost Simulating System   **
    **-----------------------------------------------**
    **         A.   Add vertexes to the net          **
    **         B.     Add edge to the net            **
    **         C. Generate minimum spanning tree     **
    **         D.     Show solution tree             **
    **         E.        Show Network                **
    **         F.          Show menu                 **
    **         G.            EXIT                    **
    ***************************************************
    
    Enter Your first command: 
\end{lstlisting}

    \item 第一条指令:增加节点
\begin{lstlisting}[style=console]
    Enter Your first command: A
    Enter amount of vertexes to add: 5
    Enter Names of Vertexes separated by LINEBREAKS.
    Empty line to quit.
    #1: a
    #2: b
    #3: c
    #4: d
    #5:          // 空行
    4 vertexes added successfully!
\end{lstlisting}
    可以看出,可以用空行打断输入。

    \item 添加边
\begin{lstlisting}[style=console]
    Enter edges in the format below, end with LINEBREAKS, empty line to quit.
    FORMAT: <vertex from> <vertex to> <weight> #LINEBREAK#
    #1: a b 3
    #2: b c 2
    #3: c d 5
    #4: a d 1000
    #5:          // 空行
    4 edges added successfully!
\end{lstlisting}
    同样也是用空行打断输入。

    \item 生成最小树
\begin{lstlisting}[style=console]
    Enter your next command: C

    Enter starting point of tree, end with a LINEBREAK: d
    d -> c
    c -> b
    b -> a
    Tree successfully built, use command D to show.
    Enter your next command: 
\end{lstlisting}
    可以选择开始节点,生成的过程会显示在输出中。

    \item 查看图
\begin{lstlisting}[style=console]
    Enter your next command: E

    Vertex a: [a] -(3)-> [b]		[a] -(1000)-> [d]		
    Vertex b: [b] -(3)-> [a]		[b] -(2)-> [c]		
    Vertex c: [c] -(2)-> [b]		[c] -(5)-> [d]		
    Vertex d: [d] -(5)-> [c]		[d] -(1000)-> [a]		
\end{lstlisting}
    原始的图会按照存储结构展示

    \item 推出
\begin{lstlisting}[style=console]
    Enter your next command: G

    Bye~
\end{lstlisting}

    
\end{enumerate}


\section{输入异常}

\begin{enumerate}
    \item 输入相同名字节点
\begin{lstlisting} [style=console]
    Enter Your first command: A
    Enter amount of vertexes to add: 4
    Enter Names of Vertexes separated by LINEBREAKS.
    Empty line to quit.
    #1: a
    #2: a
    Name a already exist, try again: 
    #2: b
    #3: c
    #4: d
    4 vertexes added successfully!
    Enter your next command: 
\end{lstlisting}
    会提示且不允许输入。

    \item 插入输入不存在的位置
\begin{lstlisting}[style=console]
    Enter your command, 6 for menu: 1       -$
    Enter the position of candidate: 6      -$
    Position should be between 1 to 5       -$
    Try again: // ...
\end{lstlisting}
    位置不存在会提醒重新输入。

    \item 添加端点为同一顶点的边
\begin{lstlisting}[style=console]
    Enter edges in the format below, end with LINEBREAKS, empty line to quit.
    FORMAT: <vertex from> <vertex to> <weight> #LINEBREAK#
    #1: a a 1
    Cannot create self link!
    #1: a b 1
    #2: 
    1 edges added successfully!
\end{lstlisting}
    不允许这么做

\item 给不完整的树生成最小树
\begin{lstlisting} [style=console]
    Enter your next command: E

    Vertex a: [a] -(1)-> [b]		
    Vertex b: [b] -(1)-> [a]		
    Vertex c: Vertex dangling!
    Vertex d: Vertex dangling!
    
    Enter your next command: C
    
    Enter starting point of tree, end with a LINEBREAK: a
    a -> b
    Tree not connected, building failed.
    Enter your next command: 
\end{lstlisting}
    显示树的时候会提示没有连接的悬挂顶点,生成失败给出信息。

    \item 生成失败或是没有生成树时打印结果
\begin{lstlisting} [style=console]
    Enter your next command: D

    Tree not ready, try build or complete the net first.
\end{lstlisting}
    提示需要先生成。

\end{enumerate}