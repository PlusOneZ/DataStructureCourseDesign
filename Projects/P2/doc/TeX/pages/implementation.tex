
\chapter{类结构}
本项目使用了{\kaishu 单链表(\lstinline{LinkedList})}和{\kaishu 大整型(\lstinline{BigInteger})}两个类型,%
求交集的部分由函数完成,显示逻辑在 \lstinline{main} 函数中。

\begin{itemize}
    \item \lstinline{LinkedList} 类\\
    单链表类,存放大整形变量,主要应用{\kaishu 尾后插入(\lstinline{insertBack})}和{\kaishu 遍历(由 \lstinline{Cursor} 内部类实现)}功能。
    \item \lstinline{BigInteger} 类\\
    其中维护了一个字符串来存放里面的数据,初始化时用 \lstinline{check} 函数自动检查且在不合法时抛出异常。重载了大于、小于、相等、赋值、输出运算符。
\end{itemize}


\chapter{LinkedList 类}

\section{API}

\subsection{类定义}
以下是本题使用链表的 API,各函数的功能已用注释标出。
\begin{lstlisting}[morendkeywords={LinkedList}, firstnumber=637, caption=LinkedList 类定义]{}
template<class Type>
class LinkedList {
public: 
    // 内部节点类
    struct LinkNode {   
        LinkNode *next = nullptr;
        Type data;

        LinkNode() = default;
        explicit LinkNode(const Type &x);
    };

    typedef LinkNode *Link;
    // 内部指针类
    class Cursor {
    protected:
        LinkedList *_list;
        Link _current;

    public:
        explicit Cursor(LinkedList *l);
        void reset();            // 置零
        void next();             // 下一节点
        Link current();          // 返回当前节点
        const Type &getElement();// 返回值
    };

    LinkedList();
    ~LinkedList();

    inline int length() const;  // 当前元素
    inline bool isEmpty() const;// 是否为空
    void clear();               // 清空
    Link locate(int i) const;   // 定位
    bool insert(int i, const Type &d);
    bool insert(int i, Link link);
    bool insertBack(const Type &d);
    void forEach(std::function<void(Type &)>);
    void show();                // 打印
    Cursor getCursor();

protected:
    Link head;
    int size = 0;

    bool makeNode(Link &) const;
    bool makeNode(Link &, const Type &x) const;
    bool insertTarget(Link tar, Link link);
};
\end{lstlisting}

{
\lstset{
    basicstyle=\ttfamily\color{CPPDark}, 
    numberstyle=\tiny\color{darkgray},
    keywordstyle=\color[RGB]{40,40,255},
    }
\subsection{公开成员}
\begin{enumerate}
    \item \lstinline{bool insertBack(const Type &d);}\\
          插入到整个链表最后,本题使用最多的函数。内部调用了\lstinline{insertTarget}。
    \item \lstinline{Link locate(int i) const;} \\
          从位置$i$取出一个节点,如果没有节点则报错并返回 \lstinline{nullptr}。
    \item \lstinline{void forEach(std::function<void(Type&)> op);} \\
          对每个元素都调用操作 \lstinline{op},可以用于格式打印等操作。
    \item \lstinline{void clear();} \\
          清空链表,只留下头节点。
    \item \lstinline{Cursor getCursor();} \\
	  获取一个指向第一个数据节点的光标对象。
\end{enumerate}

\subsection{私有成员}
\begin{enumerate}
    \item \lstinline{Link insertTarget(Link tar, Link link);}\\
          插入在 \lstinline{tar} 后面。返回插入节点的指针。该函数有两个托管函数,通过位置 $i$ 来索引节点。
    \item \lstinline{Link head;} \\
          该链表的头节点。
    \item \lstinline{int zise = 0;} \\
          链表中现有元素的个数。
    \item \lstinline{bool makeNode(Link &, const Type& x) const;} \\
          用 $x$ 的值来分配并初始化一个新的节点,失败返回 \lstinline{nullptr}。
\end{enumerate}
}

\section{代码逻辑与实现}

\subsection{逻辑描述}


本项目中,最主要改变单链表结构的动作是 {\kaishu 插入},因此要说明这个部分的内容。

插入节点时,先找到这个节点的前驱,然后做按照顺序对目标节点操作:

\begin{itemize}
    \item 目标节点的后继指针指向前驱的后继;
    \item 前驱的后继指针指向目标;
\end{itemize}


\subsection{代码实现}

{
\lstset{
      morendkeywords={LinkedList},
      morendkeywords = {head, tail, size}
}
\begin{enumerate}
      \item 插入到目标后
\begin{lstlisting} [firstnumber = 117, caption={LinkedList::insertTarget}]
template<class Type>
bool
LinkedList<Type>::insertTarget(Link target, Link link) {
    if (target == nullptr) {
        return false;
    }
    link->next = target->next;
    target->next = link;
    ++size;
    return true;
} \end{lstlisting}
      \item 插入到最后
\begin{lstlisting} [firstnumber = 148, caption={LinkedList::insertBack}]
template<class Type>
bool LinkedList<Type>::insertBack(const Type &d) {
    Link target = head, p = target->next;
    while (p != nullptr) {
        target = p;
        p = p->next;
    }
    Link newNode;
    if (!makeNode(newNode, d)) {
        std::cerr << "Error Distributing Memory in insert()" << std::endl;
        return false;
    }
    insertTarget(target, newNode);
    return false;
} \end{lstlisting}
      
\end{enumerate}
}

\chapter{BigInteger 类}

参考了 Java BigInteger 类的接口。

\section{API}

以下是 \lstinline{BigInteger} 类的类定义。
{
\lstset{morendkeywords={BigInteger}}
\begin{lstlisting}[firstnumber=251, caption=BigInteger 类定义]{}
class BigInteger {
    friend std::ostream &operator <<(std::ostream &, const BigInteger &);
public:
    explicit BigInteger(const std::string &i = "") :
            integer(check(i)) { }

    BigInteger &operator=(std::string s);
    bool operator>(const BigInteger &other) const;
    bool operator<(const BigInteger &other) const;
    bool operator==(const BigInteger &other) const;

    static std::string check(const std::string &s);

protected:
    std::string integer = "";
};
\end{lstlisting}


\section{实现}

\lstinline{BigInteger} 类其中包含了一个字符串,字符串的每一位都存放一个{\kaishu 数位(digit)},%
因此接口都是对字符串的操作。

这个项目中使用的 \lstinline{BigInteger} 主要实现了比较、赋值和打印功能,这里着重展示比较的实现。

\begin{enumerate}
    \item 大于运算:\lstinline{bool operator>(const BigInteger &other) const;}\\
\begin{lstlisting}[firstnumber=273]{}
bool BigInteger::operator>(const BigInteger &other) const {
    if (integer.length() > other.integer.length()) return true;
    if (integer.length() < other.integer.length()) return false;
    for (int i = 0; i < (int)integer.length(); ++i) {
        if (integer[i] > other.integer[i]) return true;
        if (integer[i] < other.integer[i]) return false;
    }
    return false;
}
\end{lstlisting}
小于号与大于相似。
    \item 等于运算:\lstinline{bool operator==(const BigInteger &other) const;}\\
\begin{lstlisting}[firstnumber=293]{}
bool BigInteger::operator==(const BigInteger &other) const {
    if (integer.length() != other.integer.length()) return false;
    for (int i = 0; i < (int)integer.length(); ++i) {
        if (integer[i] != other.integer[i]) return false;
    }
    return true;
}
\end{lstlisting}
\end{enumerate}

此外,因为用字符串存放数值,就不能让其他字符进入大整型的数据中,因此需要实现一个检查函数,也就是 \lstinline{stringcheck (const string &s)}。

\begin{lstlisting}[firstnumber=301, caption=\lstinline{check} 函数实现]
std::string BigInteger::check(const std::string &s) {
    if (s == "-1") return s;
    int headZeros = 0;
    while (headZeros < (int)s.length() && s[headZeros] == '0') {
        ++headZeros;
    }
    for (int i = headZeros; i < (int)s.length(); ++i) {
        if (s[i] < '0' || s[i] > '9') {
            throw NaNException();
        }
    }
    return std::string(s.begin() + headZeros, s.end());
}
\end{lstlisting}
}