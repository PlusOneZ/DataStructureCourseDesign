
\chapter{背景分析}

表达式求值是一台计算器的基础功能。在计算机中,快速准确地计算表达式是一个 CPU 的基础功能。但是限于计算机 CPU 结构的设计,%
其原生能够计算数据类型有限(整形和浮点型),所含有的计算运算也不多(逻辑运算、简单算术运算、左移右移)。如果能通过扩展的%
数据结构实现拓展类型的更多运算和复合表达式求值,并确保效率,将是一件有意义的事。

本项目可以对输入的复合表达式进行求值,可含有的类型是大整型(即没有上限的整型数),可用的运算符是加减,乘除取余,乘方和括号。%
如果表达式有语法错误,还会给出相应的提示。

\chapter{功能设计}

\section{题意转化}

我们一般的表达式是{\kaishu 中缀形式}的,也就是将左算子写在算符左边,右算子写在算符右边。这样的表达式在普通人看来最简单,但是其运算顺序并不显然。我们需要通过%
一系列规则来判断子表达式的求值顺序。下面是一个中缀表达式:
\begin{equation}
    1 + 2 * 3 + (5 - 4) / 7^8 \% 10 \label{1}
\end{equation}

按照常识,我们先算 $(5-4)$ 得出结果 $1$,在算 $7^8$ 的部分得出 $5764801$ ,再算 $2*3$ 得 $6$,接着是 $1/5764801\%10$ 得 $0$,最后 $1+6+0 $得 $7$。%
那么不同的表达式表示序列有什么影响呢,这是该表达式的{\kaishu 后缀表达式}:

\begin{center}
\lstinline{1 2 3 *  +  5 4 -  7 8 ^  /  10 %  + }
\end{center}

看起来平平无奇,但是可以发现,如果按照从左到右的顺序,将数字如一个栈,并且遇到运算符就弹出两个数字作为算符得两个算子,计算后压入栈中,这样算得的结果与%
前述按照规则计算中缀表达式是一样的。这种表达式叫做{\kaishu 逆波兰表达式}或{\kaishu 后缀表达式}。同样的有前缀表达式:

\begin{center}
\lstinline{+ + 1 * 2 3 % / - 5 4 ^ 7 8 10}
\end{center}

需要能够计算复合运算表达式,就要能判断算符的优先级,以及判断运算的先后顺序。波兰表达式是一种不需要括号就能表示运算优先顺序的表达式,%
它是普通中缀表达式的前缀或者后缀形式。通过符号栈和优先级表可以实现一个表达式向运算栈和波兰表达式的转化,但是本项目使用了另一个方法:\textbf{构建表达式树}。

前述的前、中、后缀的遍历形式来自于一颗二叉树的遍历方式,一个不含有三元运算以上算符\footnote{一元运算符可以直接结合到算子里}的表达式可以转化为一颗表达式树,%
其中每一个叶子结点放最简的算子(数字),其余有子女的节点存放运算符。越往上的运算符执行的优先级就越低,上述式子\ref{1}形成的表达式树就可以为:

\begin{figure}[H]
\centering
\forestset{ 
  default preamble={ 
    for tree={circle,draw, minimum width=0.8cm} 
  } 
} 
\begin{forest} 
    [$+$ [$+$ [1] [$\times$ [2] [3]]] [$\%$ [$/$ [$-$ [5] [4]] [$\hat{}$ [7] [8]]]  [10]]]
\end{forest}
\caption{一颗表达式树}
\end{figure}

因此本题的另一种思路是从输入开始就构建这样一颗表达式树,它的叶子节点是数字,其余节点表示二元运算符。在求值时,二元运算符将两边的值为算子做运算,数字叶子结点的值%
就是本身。这样自底向上可以完成整棵树的求值。该实现可以简单地通过递归给出。

\section{逻辑功能}


本题最主要的逻辑就是将表达式翻译成一颗表达式树,并将这棵树求值。树的求值只需通过遍历一次获得,所以需要考虑的是翻译(parse,分析)的过程。

\subsection{检查}
表达式是逐字符解析的,首先是预处理,要将不合法的表达式排除在外。此外,还要排除一些不合法的表达式。预处理中会对以下情况直接抛出异常:

\begin{itemize}
    \item 出现数字和五个运算符($+$、$-$、$\times$、$/$、$\hat{}$)和左右括号以外的符号。(符号异常)
    \item 两个数字相连,中间只出现空白字符。(语法错误)
    \item 在后括号后直接出现数字。(语法错误)
    \item 两个连续的运算符,不管中间有没有空格。(语法错误)
\end{itemize}

另外,出现简写乘法的情况,表达式中会自动补上乘号。一般我们写:

$$ 2 + 8(9+3)$$

是想写:

$$ 2 + 8 \times (9 + 3)$$

这不是语法错误,因此在求值时不能略去该情况。

\subsection{翻译表达式}

表达式需要通过处理才能形成一棵表达式二叉树。首先说明优先级和结合律。

\subsubsection{优先级、结合律}
\begin{table}[H]
    \centering
    \caption{符号优先级和结合律}
    \begin{tabular}{@{}|lcc|@{}}
    \toprule
    \rowcolor[HTML]{C0C0C0} 
    符号                 & 优先级 & 结合律 \\ \midrule
   $+$, $-$(一元)             & 4   & 无   \\ \midrule
    $($, $)$               & 4   & 无   \\ \midrule
    \textasciicircum{} & 3   & 右   \\ \midrule
    $\times$, $\div$, $\%$           & 2   & 左   \\ \midrule
    $+$, $-$               & 1   & 左   \\ \bottomrule
    \end{tabular}
\end{table}

{\kaishu 优先级}很好理解,在没有括号的表达式中谁先计算谁的优先级高。在本项目中,一元运算符的优先级最高,%
二元运算符中乘方比乘法系列高,加法系列最低。{\kaishu 结合律}比较难理解,运算符分为 {\kaishu 左结合}和{\kaishu 右结合}。%
对于可交换的运算来说左右结合没有差别,但是对于不能交换的运算,不同的结合方式会造成不同的结果:减法是不可交换的左结合运算,%
因此,在计算$$5 - 3 - 2$$时,产生了图 \ref{associa} 左边的树,而不是右边的树。

\begin{figure} [H]
\centering
\forestset{ 
  default preamble={ 
    for tree={circle,draw, minimum width=0.8cm} 
  } 
} 
    \centering 
    \begin{forest}
        [$-$ [$-$ [5] [3]] [2]] 
    \end{forest}
    \hspace*{1cm}
    \begin{forest}
        [$-$ [5] [$-$ [3] [2]]] 
    \end{forest}
    \caption{结合律(左:左结合,右:右结合)} 
    \label{associa} 
\end{figure} 

与结果对应的, $5 - 3 - 2$ 的值应该是 $0$ 而不是 $4$。

在项目中,只有乘方运算是右结合的。在计算$$3^{2^{2^2}}$$时\footnote{还有一种方式说明乘方运算是右结合的,在 Python 脚本中运行 \lstinline{3**2**2**2} 结果是 43046721,而不是6561。},产生了树:

\begin{figure}[H]
    \centering
    \forestset{ 
      default preamble={ 
        for tree={circle,draw, minimum width=0.8cm} 
      } 
    } 
    \begin{forest} 
        [$\hat{}$ [3] [$\hat{}$ [2] [$\hat{}$ [2] [2]]]]
    \end{forest}
    \caption{$3^{2^{2^2}}$的树}
\end{figure}

所以在产生乘方树时需要用不同的顺序生成节点。

接下来说明翻译的逻辑。

\subsubsection{翻译}
翻译表达式有两个步骤:{\kaishu 产生节点}、把{\kaishu 两个节点接到一个节点下}。所以有两个函数互相递归调用的话,实现会稍微简单。其中,%
产生节点的函数处理的主要是一元运算符,处理括号、一元正负符号、数字。连接节点的函数则是处理二元运算符的,他产生两个左右节点,用当前运算符当作%
父节点连起来。函数中使用了一个指向字符串某位置的指针来记录翻译进程和保存当前字符信息,以下是两个函数的逻辑:


\[
\text{一元运算符}
\begin{cases}
    \text{(:将括号内容当作一个新的表达式,返回后检查后括号是否存在。}\\
    \text{-:将后面当作一个一元表达式,返回后将值取相反数。}\\
    \text{+:将后面当作一个一元表达式,直接返回}\\
    \text{数字:将该内容开始的所有数字存到一个 BigInteger 对象中}\\
\end{cases}
\]\\

以下 $n$ 代表优先级。

\[
\text{二元运算符}
\begin{cases}
    n \ge 4 \text{:将括号内的内容当作一元的,调用一元表达式翻译。}\\
    \text{左算子等于} n+1 \text{子表达式}
    \begin{cases}
        \text{算式结束:返回左算子。} \\
        \text{右结合:返回左表算子与右边 }\\
        \quad \quad \quad n \text{ 优先级表达式的新节点。}\\
        \text{左结合:右算子等于 } n+1 \text{}{ 子表达式,}\\
        \quad \quad \quad \text{左算子更新为左右算子为子女的新节点。}\\
    \end{cases}
\end{cases}
\]


\subsubsection{求值}
经过检查和翻译,现在表达式已经构成了一棵只有叶子结点是数字,其余节点都是操作符的表达式二叉树。最后一步是求值:

\tikzstyle{startstop} = [rectangle, rounded corners, minimum width = 2cm, 
        minimum height=1cm, text centered, draw = black, fill = red!40,
        font = {\bfseries}]
    \tikzstyle{io} = [trapezium, trapezium left angle=70, trapezium right angle=110, 
    minimum width=2cm, minimum height=1cm, text centered, draw=black, fill = blue!40]
    \tikzstyle{process} = [rectangle, minimum width=3cm, minimum height=1cm, text centered, draw=black, fill = yellow!50]
    \tikzstyle{condition} = [diamond, aspect = 3, text centered, draw=black, fill = green!30]
    \tikzstyle{arr} = [->, >=stealth, thick]

\begin{figure}[H]
    \begin{tikzpicture}
        \node [startstop] (start) {开始};
        \node [condition, below=0.6cm of start] (cond) {节点为符号?};
        \node [process, below=0.6cm of cond, xshift=-4cm] (op) {计算左右子表达式(递归),返回左右求值};
        \node [process, below=0.6cm of cond, xshift=3cm] (num) {返回节点数字};
        \node [io, below=0.6cm of op, xshift=4cm] (result) {输出结果};
        \node [startstop, below=0.6cm of result] (end) {结束};

        \draw [arr] (start) -- (cond);
        \draw [arr] (cond) -| node[anchor=south] {是} (op);
        \draw [arr] (cond) -| node[anchor=south] {否} (num);
        \draw [arr] (op) |- (result);
        \draw [arr] (num) |- (result);
        \draw [arr] (result) -- (end);
    \end{tikzpicture}
    \caption{表达式树求值过程}
\end{figure}