
\chapter{类结构}
本项目由大整型(\lstinline{BigInteger}),表达式节点(\lstinline{ExpressionNode})和两个子类:数字节点(\lstinline{NumberNode})、%
符号节点(\lstinline{OperationNode}),表达式翻译器(\lstinline{ExpressionParser}),测试用例(\lstinline{TestCases})类构成。

\begin{itemize}
    \item \lstinline{BigInteger} 类\\
    大整型类,使表达式求值支持很大的整数类型。
    \item \lstinline{ExpressionNode} 类\\
    表达式树的节点,是一个抽象类,其子类数字节点(\lstinline{NumberNode})用于存放数字,符号节点(\lstinline{OperationNode})用于存放符号,%
    节点各自实现了求值和前、后续遍历函数。
    \item \lstinline{ExpressionParser} 类\\
    最重要的类,封装了翻译解析整个表达式内容的逻辑代码。
    \item \lstinline{TestCases} 类\\
    包含了多种可能出现的极端情况测试,并与答案比对。还有异常输入与报错的测试。
\end{itemize}

按照逻辑顺序,本部分将先介绍节点与翻译器,再简单介绍大整型的运算符实现逻辑,最后介绍测试类。

\chapter{ExpressionNode 类与子类}

\lstinline{ExpressionNode} 类是一个提供抽象接口的类,没有实际实现。
\begin{lstlisting}[morendkeywords={ExpressionNode}, firstnumber=383, caption=ExpressionNode 类定义]{}
class ExpressionNode {
public:
    ExpressionNode() = default;
    virtual ~ExpressionNode() = default;

    // 计算
    virtual BigInteger evaluate() = 0;
    // 遍历
    virtual void prefixVisit(std::ostream &os) = 0;
    virtual void postfixVisit(std::ostream &os) = 0;
};
\end{lstlisting}

实现在两个子类里,其中\lstinline{NumberNode} 含有一个私有成员:

\begin{lstlisting}[morendkeywords={ExpressionNode}, firstnumber=406]{}
private:
    BigInteger num;
\end{lstlisting}

其接口的实现很简单:求值时返回 \lstinline{num},遍历时输出 \lstinline{num}。

\lstinline{OperationNode} 含有三个私有成员:

\begin{lstlisting}[morendkeywords={ExpressionNode}, firstnumber=434]{}
private:
    char op;
    ExpressionNode *leftChild;
    ExpressionNode *rightChild;
\end{lstlisting}

其求值用的是 \lstinline{switch} 语句,对每一种运算符做对应的运算;遍历时,前序遍历就先输出 \lstinline{op},%
再对左右节点调用前序遍历,后序就是先对左右节点调用后序遍历后再输出自己的 \lstinline{op}。


\chapter{ExpressionParser 类}

\section{API}
\begin{lstlisting}[morendkeywords={ExpressionParser}, firstnumber=482, caption=ExpressionParser 类定义]{}
class ExpressionParser {
    class StringHandler {...}; // handler for string
\end{lstlisting}
\begin{lstlisting} [morendkeywords={ExpressionParser}, firstnumber=508]
public:
    explicit ExpressionParser(string expr) 
            : handler(std::move(expr)) {};
    ~ExpressionParser() = default;

    // parse
    ExpressionNode *parse(int precedence);
    ExpressionNode *unaryParse();

    // result
    void calculate();
    const BigInteger &getResult() { return result; }

    // visit
    void prefixTraverse(std::ostream &os);
    void postfixTraverse(std::ostream &os);

private:
    enum Precedence {
        AddPrecedence = 1,          // + -
        MulPrecedence = 2,          // * / %
        PowerPrecedence = 3,        // ^
        UnaryPrecedence = 4         // + - ()
    };
    enum Associativity {
        Left, Right
    };

    StringHandler handler;
    BigInteger result;
    ExpressionNode *tree = nullptr;

    // operation rules
    static Precedence precedenceOf(char op);
    static Associativity associativityOf(char op);
};
\end{lstlisting}

其中 \lstinline{StringHandler} 是处理字符信息的内部类型,其接口和实现如下:

\begin{lstlisting} [morendkeywords={StringHandler}, firstnumber=485]
class StringHandler { // handler for string
public:
    explicit StringHandler(string expr)
            : expression(std::move(expr)), curPos(0),
              curOp('\0') { next(); };
    // 预处理
    void preprocess();
    void next(); // 下一字符
    bool isOp()  { return status == Operator; }
    bool isNum() { return status == Number; }
    bool isEnd() { return status == End; }

public:
    string expression;  // 表达式
    unsigned int curPos;// 标记状态
    char curOp;
    BigInteger curNum;
    enum Status {
        Number, Operator, End
    } status = Number;

private:
    unsigned parseInteger();// 解析整数
};
\end{lstlisting}


\section{代码逻辑与实现}

\subsection{逻辑描述}

逻辑其实在设计已经介绍了大部分,这里主要展示代码。

\subsection{代码实现}

{
\lstset{
      morendkeywords={ExpressionParser, StringHandler},
      morendkeywords = {Left, Right,handler, result, tree, AddPrecedence, MulPrecedence, PowerPrecedence, UnaryPrecedence}
}
\begin{enumerate}
      \item 一元运算符
\begin{lstlisting} [firstnumber = 674, caption={ExpressionParser::unaryParse}]
ExpressionNode *ExpressionParser::unaryParse() {
    ExpressionNode *ret = nullptr;
    if (handler.isOp() && handler.curOp == '(') {
        handler.next();
        ret = parse(AddPrecedence);
        if (handler.status != StringHandler::Operator || handler.curOp != ')') {
            throw ParenthesesNotMatch();
        }
        handler.next();
    } else {
        if (handler.isOp() && handler.curOp == '-') {
            handler.next();
            auto temp = unaryParse();
            ret = new NumberNode(-temp->evaluate());
            delete temp;
        } else if (handler.isOp() && handler.curOp == '+') {
            handler.next();
            ret = unaryParse();
        } else if (handler.isNum()) {
            ret = new NumberNode(handler.curNum);
            handler.next();
        }
    }
    return ret;
} \end{lstlisting}
      \item 二元运算符
\begin{lstlisting} [firstnumber = 652, caption={ExpressionParser::parse}]
ExpressionNode *ExpressionParser::parse(int precedence) {
    if (precedence >= UnaryPrecedence) {
        return unaryParse();
    }
    auto leftOperand = parse(precedence + 1);
    if (handler.isEnd())
        return leftOperand;
    char op = handler.curOp;
    if (associativityOf(op) == Right) {
        handler.next();
        return new OperationNode(op, leftOperand, parse(precedence));
    } else {
        while (precedenceOf(op) == precedence && !handler.isEnd()) {
            handler.next();
            auto rightOperand = parse(precedence + 1);
            leftOperand = new OperationNode(op, leftOperand, rightOperand);
            op = handler.curOp;
        }
        return leftOperand;
    }
}\end{lstlisting}
      
\end{enumerate}
}

\chapter{BigInteger 类}

实现了加、减、乘、除、乘方、模除运算。

\section{API}

{
\lstset{morendkeywords={HashTable}}
\begin{lstlisting}[firstnumber=22, caption=BigInteger 类定义]{}
class BigInteger {
public:

    BigInteger() : value("0"), negative(false) {}
    explicit BigInteger(int i);
    explicit BigInteger(const string &s);
    BigInteger(const BigInteger &bi);
    ~BigInteger() = default;

    // 一元运算符
    BigInteger operator-() const;
    BigInteger operator+() const;
    BigInteger absolute() const;

    std::ostream &show(std::ostream &os) const;
    int compare(const BigInteger &bi) const; // 比较

    // 二元运算符
    BigInteger &operator=(const BigInteger &bi);
    BigInteger operator+(const BigInteger &bi) const;
    BigInteger operator-(const BigInteger &bi) const;
    BigInteger operator*(const BigInteger &bi) const;
    BigInteger operator/(const BigInteger &bi) const;
    BigInteger operator%(const BigInteger &bi) const;
    BigInteger operator^(const BigInteger &bi) const;
    bool operator==(const BigInteger &bi) const;

protected:
    string value;
    bool negative = false;

    int compareAbs(const BigInteger &bi) const;
};
\end{lstlisting}


\section{代码逻辑与实现}


大整型运算实际上是在模拟小学列竖式计算时的过程:加减法每一位对齐,对位相加减并考虑进借位;乘法则与每位相乘后相加;%
除法从最高位除数乘某数来试;模除则是 被除数 - 除数 * 除法结果。

由于代码过长所以具体实现不在文档中展示,可以在源码中查看。

\chapter{TestCases 类}

因为有许多情况难以完全考虑,在测试本项目时我使用了一个测试用例,其覆盖了能想到的许多极端情况并且有异常输入测试,测试类大致代码结构为:

\begin{lstlisting}[firstnumber=722, caption= TestCases 类定义]{}
class TestCases {
public:
    typedef BigInteger BI;      // 别名
    typedef ExpressionParser EP;
    void run();

    void compareResults();
    void exceptionTests();

private:
    static constexpr int testNum = 11;
    static constexpr int exceptionNum = 6;
    EP testCases[testNum] = {
            EP("((((1))))="),     // 括号
            EP("1+1+1+1-1-1"),    // 加减
            EP("10*60/89*45"),    // 乘除
            EP("10293^253"),      // 乘方
            EP("8764 % 675"),     // 模除
            EP("5+8*8-98/9"),     // 优先级
            EP("87*43/3^2+98%3"), // 优先级
            EP("(8+6*(5+4)^3)%31"),// 优先级
            EP("3^3^3"),          // 右结合
            EP("8+-5--4++73"),    // 一元运算
            EP("847(8374)"),      // 简写
    };
    BI answers[testNum] = {
            BI("1"),
            BI("2"),
            BI("270"),
            BI(...), // 太长
            BI("664"),
            BI("59"),
            BI("417"),
            BI("11"),
            BI("7625597484987"),
            BI("80"),
            BI("7092778")
    };
    EP exceptionCases[exceptionNum] = {
            EP("2837&4$5=1"),
            EP("4563%(1-1)"),
            EP("(837465+8384^5465"),
            EP("1+"),
            EP(""),
            EP("1 2 3"),
    };
    string expectedException[exceptionNum] = {
            "UnidentifiedToken",
            "DivideByZero",
            "ParenthesesNotMatch",
            "SyntaxError",
            "EmptyExpression",
            "SyntaxError"
    };
};
\end{lstlisting}

在代码中,有一行:

\begin{lstlisting}[firstnumber=834, caption=启用TestCases]{}
// #define CASE_TEST
\end{lstlisting}  

解除注释就能编译出使用测试用例的程序。

运行结果在下一部分展示。