
\chapter{类结构}

\section{概述}
本题将所有操作都封装在了 \lstinline{SortingSystem} 类中,并且为了完成题目要求的计数,类内用了几个内联函数%
来承载比较和交换的操作,这些函数有一个 \lstinline{enum} 型的参数 \lstinline{Sortings s},会对类成员计数数组%
的值进行相应的增加。

此外,这个 \lstinline{SortingSystem} 类是模板类,因此所有的排序函数都能在不同的类型上使用\footnote{是的,基数排序也可以,不过我添加了类型检查}%
(前提是重载了比较、赋值运算),随机数生成方面使用了 C++ \emph{<random>} 头文件产生的随机数生成器,当然,可以痛哟模板参数来%
修改默认的随机数生成器(如果打算给浮点数排序,就需要使用浮点数随机数生成器)。下面介绍一些有关设计,之后展示算法内容。

\section{SortingSystem 类定义}

\begin{lstlisting}[morendkeywords={SortingSystem}, firstnumber=8, caption=SortingSystem 类定义]{}
template <class Type, class RandomMachine = std::uniform_int_distribution<Type>>
class SortingSystem {
    enum Sortings {
        Bubble, Selection, Insert, Shell,
        Quick, Heap, Merge, Radix, NoSort
    };
    // 大于
    static bool greater(Type t1, Type t2, Sortings s) {..};
    // 小于
    static bool less(Type t1, Type t2, Sortings s) {..};
    // 等于
    static bool equal(Type t1, Type t2, Sortings s) {..}
    // 交换
    static void swap(Type &t1, Type &t2, Sortings s) {..}
    // 二者最小
    template<typename T>
     static const T& min(const T &a, const T &b, Sortings s = NoSort) {..}

public:
    SortingSystem() = default;
    // 排序
    static void bubbleSort(Type arr[], size_t n);
    static void selectionSort(Type arr[], size_t n);
    static void insertSort(Type arr[], size_t n);
    static void shellSort(Type arr[], size_t n);
    static void quickSort(Type arr[], size_t n);
    static void heapSort(Type arr[], size_t n);
    static void mergeSort(Type arr[], size_t n);
    static void radixSort(Type arr[], size_t n);
    // 功能逻辑
    void resize(int n);
    clock_t testAndClock(Sortings s);
    bool parseCommand(const std::string &cmd);
    void redesignateSize();
    void showTestResultOf(Sortings s, const std::string& name);
    void run();

private:
    // system
    void genRandom(size_t n);
    static void showMenu();
    static void copy(Type *dest, Type *source, size_t size);
    static void clearCounters(Sortings s);
    static bool checkResult(Type *arr, size_t n);

    // quick sort
    static void quickSort(Type arr[], Type *l, Type *r);
    static Type * partition(Type *first, Type *last, Type pivot);
    static const Type& medium(const Type &a, const Type &b, const Type &c);
    // heap sort
    static void minHeapify(Type arr[], int start, int end);
    // radix sort
    static Type maxDigits(Type *arr, size_t n, const Type &radix);
    static void radixSort(Type arr[], size_t n, const Type& radix);

private:
    Type *array = nullptr;
    size_t curSize = 0;
    static unsigned long long compareCount[9];
    static unsigned long long swapCount[9];
    static unsigned long long assignCount[9];
};

template <class T, class R>
unsigned long long SortingSystem<T, R>::compareCount[9];
template <class T, class R>
unsigned long long SortingSystem<T, R>::swapCount[9];
template <class T, class R>
unsigned long long SortingSystem<T, R>::assignCount[9];
\end{lstlisting}


\chapter{辅助设计}

本项目使用了以下设计来使排序计数更加合理:

\begin{enumerate}
    \item 使用了三个辅助数组来完成计数功能。题目要求给比较和交换计数,但是有些算法是没有严格意义上的交换的,因此我加入了给赋值计数的数组。
          数组可以用规定的\lstinline{enum}标签\lstinline{Sortings}来索引。
    \item 使用了四个辅助函数来完成比较、交换的过程。因为每次比较交换赋值都会导致数组内的数值增加,所以需要将这一小段代码封装起来%
          方便复用。这些函数是写在类内的,因此是明确内联的。
    \item 用函数 \lstinline{clock_t testAndClock(Sortings s);} 完成了测试和计时的封装,返回值是 \lstinline{clock_t} 的数,%
          表示函数排序所用的时间。
    \item 产生的序列存储在 \lstinline{Type *array} 里,每次排序的数组是它的一个拷贝,这样可以保证每一个排序函数使用的数组内容都是相同的。%
    \item 想要刷新数组或是重新分配大小,可以调用 \lstinline{resize} 函数。
    \item 每次排序后会正序遍历一遍整个序列以检查排序结果是否有序。
\end{enumerate}

\chapter{排序实现}

以下代码省略了模板参数。

\section{冒泡排序}
\begin{lstlisting}[firstnumber= 97, caption=bubbleSort]
void SortingSystem::bubbleSort(Type *arr, size_t n) {
    Type temp;
    for (int i = 0; i < n - 1; ++i) {
        for (int j = 0; j < n - i - 1; ++j) {
            if (greater(arr[j], arr[j+1], Bubble)) {
                swap(arr[j], arr[j+1], Bubble);
            }
        }
    }
}
\end{lstlisting}

\section{选择排序}
\begin{lstlisting}[firstnumber=109 , caption=selectionSort]
void SortingSystem::selectionSort(Type *arr, size_t n) {
    for (int i = 0; i < n; ++i) {
        int minIndex = i;
        for (int j = i; j < n; ++j) {
            if (greater(arr[minIndex], arr[j], Selection))
                minIndex = j;
        }
        if (minIndex != i) {
            swap(arr[minIndex], arr[i], Selection);
        }
    }
}
\end{lstlisting}

\section{直接插入排序}
\begin{lstlisting}[firstnumber=120 , caption=insertSort]
void SortingSystem::insertSort(Type *arr, size_t n) {
    int j;
    for (int i = 0; i < n; ++i) {
        Type temp = arr[i];
        for (j = i; j > 0 && less(temp, arr[j - 1], Insert); --j) {
            arr[j] = arr[j - 1];
            ++assignCount[Insert];
        }
        arr[j] = temp;
        ++assignCount[Insert];
    }
}
\end{lstlisting}

\section{希尔排序}
\begin{lstlisting}[firstnumber=134 , caption=shellSort]
void SortingSystem::shellSort(Type *arr, size_t n) {
    size_t gap = 1;
    while (gap < n/3) gap = 3*gap + 1;
    while (gap >= 1) {
        for (int i = gap; i < n; i++) {
            for (int j = i; j >= gap && less(arr[j], arr[j - gap], Shell);
                 j -= gap) {
                swap(arr[j], arr[j - gap], Shell);
            }
        }
        gap = gap / 3;
    }
}
\end{lstlisting}

\section{快速排序}
\begin{lstlisting}[firstnumber=149 , caption=quickSort]
void SortingSystem::quickSort(Type *arr, size_t n) {
    quickSort(arr, arr, arr + n - 1);
}
\end{lstlisting}

\begin{lstlisting}[firstnumber=154]
void SortingSystem::quickSort(Type *arr, Type *l, Type *r) {
    if (l >= r)
        return;
    auto mid = l + (r - l) / 2;
    auto &pivot = medium(*l, *mid, *r);// 取中间
    ++assignCount[Quick];
    auto cut = partition(l, r, pivot);
    quickSort(arr, l, cut);
    quickSort(arr, cut + 1, r);
}
\end{lstlisting}

\begin{lstlisting}[firstnumber=166]
Type * SortingSystem::partition(Type *first, Type *last, Type pivot) {
    while (true) {
        while (less(*first, pivot, Quick)) ++first;
        while (less(pivot, *last, Quick))  --last;
        if (first >= last) return last;
        swap(*first, *last, Quick);
        ++first; --last;
    }
}
\end{lstlisting}

\section{堆排序}
\begin{lstlisting}[firstnumber=189 , caption=heapSort]
void SortingSystem::heapSort(Type *arr, size_t n) {
    for (long long i = n/2 - 1; i >= 0; i--)
        maxHeapify(arr, i, n - 1);
    for (long long i = n-1; i > 0; i--) {
        swap(arr[0], arr[i], Heap);
        maxHeapify(arr, 0, i - 1);
    }
}
\end{lstlisting}

\begin{lstlisting}[firstnumber=199]
void SortingSystem::maxHeapify(Type *arr, int start, int end) {
    int parent = start;
    int child = parent * 2 + 1;
    while (child <= end) {
        if (child + 1 <= end && less(arr[child], arr[child + 1], Heap)) {
            child++;
        }
        if (greater(arr[parent], arr[child], Heap)) {
            return;
        } else {
            swap(arr[parent], arr[child], Heap);
            parent = child;
            child = parent * 2 + 1;
        }
    }
}
\end{lstlisting}

\section{归并排序}
\begin{lstlisting}[firstnumber=217 , caption=mergeSort]
void SortingSystem::mergeSort(Type *arr, size_t n) {
    Type *copy = arr;
    Type *cache = new Type[n];
    for (size_t step = 1; step < n; step += step) {
        for (size_t start = 0; start < n; start += 2 * step) {
            size_t low = start;
            size_t mid = min(start + step, n), 
                high = min(start + 2 * step, n);
            size_t k = low;
            size_t leftStart = low, leftEnd = mid;
            size_t rightStart = mid, rightEnd = high;
            // Merge
            while (leftStart < leftEnd && rightStart < rightEnd) {
                cache[k++] = less(copy[leftStart], copy[rightStart], Merge)
                        ? copy[leftStart++] : copy[rightStart++];
                ++assignCount[Merge];
            }
            // dealing with left parts
            while (leftStart < leftEnd) {
                cache[k++] = copy[leftStart++];
                ++assignCount[Merge];
            }
            while (rightStart < rightEnd) {
                cache[k++] = copy[rightStart++];
                ++assignCount[Merge];
            }
        }
        // Swap merged array with unmerged one,
        Type *temp = copy;
        copy = cache;
        cache = temp;
    }
    if (copy != arr) {
        // make sure arr is the result.
        for (int i = 0; i < n; i++) {
            cache[i] = copy[i];
            ++assignCount[Merge];
        }
        cache = copy;
    }
    delete[] cache;
}
\end{lstlisting}

\section{基数排序}
\begin{lstlisting}[firstnumber= 261, caption=radixSort]
void SortingSystem::radixSort(Type *arr, size_t n) {
    if (!(typeid(Type) == typeid(unsigned) ||
          typeid(Type) == typeid(unsigned long) ||
          typeid(Type) == typeid(unsigned long long)
    )) { // 类型检查
        cerr << "Radix sort is only for positive numeric value." << endl;
        return;
    }
    radixSort(arr, n, static_cast<Type>(10));
}
\end{lstlisting}

\begin{lstlisting}[firstnumber= 291]
void SortingSystem<Type, RandomMachine>::radixSort(Type *arr, size_t n, const Type &radix) {
    if (radix <= 0) {
        std::cerr << "Invalid radix" << std::endl;
        return;
    }
    Type *s = arr;
    Type d = maxDigits(arr, n, radix);
    Type *tmp = new Type[n];
    Type *count = new Type[radix];
    long long i, j, k;
    Type curRadix = (Type)(1); ++assignCount[Radix];
    for(i = 1; i <= d; i++) {
        for (j = 0; j < radix; j++) {
            count[j] = (Type)(0);
            ++assignCount[Radix];
        }
        // count
        for (j = 0; j < n; j++) {
            k = (arr[j] / curRadix) % radix;
            ++count[k];
        }
        // make room
        for (j = 1; j < radix; j++) {
            count[j] = count[j - 1] + count[j];
        }
        // move
        for (j = n - 1; j >= 0; j--) {
            k = (arr[j] / curRadix) % radix;
            tmp[count[k] - 1] = arr[j];
            ++assignCount[Radix];
            --count[k];
        }
        auto p = tmp;
        tmp = arr;
        arr = p;
        curRadix = curRadix * radix;
    }
    if (arr != s) {
        for (int l = 0; l < n; ++l) {
            tmp[l] = arr[l];
            ++assignCount[Radix];
        }
        tmp = arr;
    }
    delete []tmp;
    delete []count;
}
\end{lstlisting}
