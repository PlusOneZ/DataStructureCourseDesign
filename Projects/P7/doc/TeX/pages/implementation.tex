\lstset {
    morendkeywords={MinHeap, Vector, BigInteger, WoodCutter}
}

\chapter{类结构}

本项目最重要的数据结构是 {\kaishu 最小堆},为了实现堆,我引入了 Vector 类,此外为了处理木头长度太长超出整型表达范围%
的问题,本题引入了大整型(BigInteger) 类。最后,项目将输入输出和计算的逻辑代码封装在WoodCutter类中。另外,本项目%
提供了自带测试用例 TestCases 类。

\begin{itemize}
    \item \lstinline{MinHeap} 类\\
    最小堆,用于存放所有的数据并且每次弹出最小值。
    \item \lstinline{WoodCutter} 类\\
    封装输入、砍树逻辑、输出的代码。
    \item \lstinline{Vector} 类\\
    \lstinline{MinHeap} 类用于存放元素的数组。
    在\lstinline{HashTable} 类的实现中主要用到了下标运算、\lstinline{reserve(n)} 方法:开辟 $n$ 长度的数组。\lstinline{resize(n)}%
    方法:重新分配 $n$ 长度的内存。
    \item \lstinline{BigInteger} 类\\
    防止数字太大溢出而使用的大整型类,重载了加法和比较运算。
    \item \lstinline{TestCases} 类\\
    含有测试用例,可以自动进行测试。
\end{itemize}

该部分主要详细介绍 \lstinline{Vector} 类和\lstinline{MinHeap} 类的实现,其余部分简单概括被使用的接口。


\chapter{Vector 类}

\lstinline{Vector} 类在之前出现过很多次,都是一笔带过,这次用到了 \lstinline{Vector} 的较多功能,因此在这里介绍这个类。

\lstinline{Vector}类的实现和 API 参考了 \emph{C++ STL} 中的\lstinline{vector} 类,后者是标准库成员之一。

\section{API}

\subsection{类定义}
以下是本题使用\lstinline{Vector}的 API,各函数的功能已用注释标出。
\begin{lstlisting}[morendkeywords={Vector}, firstnumber=637, caption=LinkedList 类定义]{}
template<class ElemType>
class Vector {
public:
    Vector();
    Vector(size_t num, const ElemType& d);
    Vector(const Vector<ElemType> &vec);
    ~Vector();

    void resize(size_t new_size);// 重分配
    void reserve(size_t n);      // 保留空间
    const ElemType &pop() const; // 返回元素
    void popBack();              // 删除元素
    void pushBack(const ElemType &e);
                                 // 插入
    size_t length() const;  // 长度
    bool isEmpty() const;   // 是否为空
    void clear();           // 清零
    void swap(Vector<ElemType> &another);
                            // 交换
    ElemType &operator[](int index);// 下标运算
    const ElemType &operator[](int index) const;
    Vector &operator=(Vector<ElemType> const &another);

    std::ostream &show(std::ostream& os) const;

protected:
    size_t size = 0;    // 当前有效元素
    int capacity = 10;  // 当前容量
    ElemType *data;     // 数组
    static constexpr int LowerBound = 10;

    void extend();      // 扩张
    void tryShrink();   // 缩减
    void reallocate(size_t new_size);
};
\end{lstlisting}
\vspace*{1cm}

{
\lstset{
    basicstyle=\ttfamily\color{CPPDark}, 
    numberstyle=\tiny\color{darkgray},
    keywordstyle=\color[RGB]{40,40,255},
    }
\subsection{公开成员}
\begin{enumerate}
    \item \lstinline{bool isEmpty() const;} \\
          查看是否为空队,是的话返回 \lstinline{true},否返回 \lstinline{false}。
    \item \lstinline{int length() const;} \\
          返回当前队列中的元素个数。
    \item \lstinline{void clear();} \\
          清空队列。清空后调用 \lstinline{length()} 应返回 0,调用 \lstinline{isEmpty()} 返回 \lstinline{true}。
    \item \lstinline{std::ostream &show(std::ostream &os) const;} \\
          从队头到队尾依次输出元素到 \lstinline{os}。
    \item \lstinline{int resize(int new_size);} \\
          重新调整 \lstinline{data} 有效元素个数。如果\lstinline{new_size} 比当前 \lstinline{capacity} 大的话就%
          重新分配数组。
    \item \lstinline{void reserve(size_t n);} \\
          如果\lstinline{n + size} 比当前 \lstinline{capacity} 大的话就重新分配数组。否则什么也不做。%
          这个函数保证 \lstinline{data} 内至少有 \lstinline{n} 个空位。
\end{enumerate}

\subsection{私有成员}
\begin{enumerate}
    \item \lstinline{Type * data;} \\
          存放数据的数组,\lstinline{Type} 是范型的型别。
    \item \lstinline{int size; int capacity; int front; int rear;} \\
          分别是 \lstinline{data} 数组有效元素的个数、数组大小,队列之头的位置,队列之尾的位置。\\
    \item \lstinline{void tryExtend();} \\
          封装了“检查数组是否已满,若已满就扩大到两倍”的动作。
    \item \lstinline{void tryShrink();} \\
          封装了“检查数组是否过空($1/4$ 容量),若过空就缩小一半”的动作。
\end{enumerate}
}

\section{代码逻辑与实现}

\subsection{逻辑描述}

\lstinline{Vector} 有两个表示大小的数据:\lstinline{size} 和 \lstinline{capacity},两者分别为当前有效元素(可以被下标索引的),
和内部数组的总空间。\lstinline{Vactor} 内部数据可能长这样:

\begin{figure}[H]
    \centering
    \tikzstyle{c} = [minimum width = 8mm, minimum height = 10mm]
    \tikzstyle{token} = [draw = white, node distance = 1.2cm]
    \tikzstyle{arr} = [thick, ->]
\begin{tikzpicture}[every node/.style={draw}]
    \matrix (m1) [draw=white]
    {
          \node [c](1){1}; & \node [c](2){2}; & \node [c](3){3}; & \node [c](4){ };\\
    };
    \node [token, below=0.3cm of 2] {capacity=4};
    
    \node (r1)[token, above of = 3]{size};
    \draw [arr] (r1) -- (3);
    
    \matrix (m2) [draw=white, below=2cm of m1]
    {
          \node [c](21){1}; & \node [c](22){2}; & \node [c](23){3}; & \node [c](24){4};\\
    };
    \node [token, below=0.3cm of 22] {capacity=4};

    \node (r2)[token, above of = 24]{size};
    \draw [arr] (r2) -- (24);
    
    \matrix (m3) [draw=white, below=2cm of m2.east]
    {
          \node [c](31){1}; & \node [c](32){2}; & \node [c](33){3}; & \node [c](34){4}; & 
          \node [c](35){5}; & \node [c](36){ }; & \node [c](37){ }; & \node [c](38){ };\\
    };
    \node [token, below=0.3cm of 32] {capacity=4};
    
    \node (r3)[token, above of = 35]{size};
    \draw [arr] (r3) -- (35);

\end{tikzpicture}
\caption{Vector 内部数据示意图}
\end{figure}

可以看出,到了一定边界条件,\lstinline{Vactor}会自动伸长来保证可以装下更多元素。

\subsection{代码实现}

{
\lstset{
      morendkeywords={LinkedList},
      morendkeywords = {head, tail, elemNum}
}
\begin{enumerate}
      \item 插入元素到最后
\begin{lstlisting} [firstnumber = 164, caption={Vector::pushBack}]
template<class ElemType>
void Vector<ElemType>::pushBack(const ElemType &e) {
    if (size == capacity) {
        extend();
    }
    data[size++] = e;
} \end{lstlisting}
      \item 删除最后元素
\begin{lstlisting} [firstnumber = 154, caption={Vector::popBack}]
template<class ElemType>
void Vector<ElemType>::popBack() {
    if (!isEmpty()) {
        size--;
        tryShrink();
    } else {
        std::cerr << "Popping Back an empty vector!";
    }
}\end{lstlisting}

    \item 扩张数组(私有)
\begin{lstlisting} [firstnumber = 138, caption={Vector::extend}]
template<class ElemType>
void Vector<ElemType>::extend() {
    if (capacity == 0) {
        capacity = LowerBound;
        data = new ElemType[capacity];
    } else {
        reallocate(2 * capacity);
    }
}\end{lstlisting}
      

    \item \lstinline{reallocate} 内部实现
\begin{lstlisting} [firstnumber = 113, caption={Vector::reallocate}]
template<class ElemType>
void Vector<ElemType>::reallocate(size_t new_size) {
    if (new_size < 0) {
        std::cerr << "Cannot resize to negative or zero!" << std::endl;
        return;
    }
    if (new_size == capacity)
        return;
    auto new_data = new ElemType[new_size];
    if (new_size > capacity) {
        if (!isEmpty())
            memcpy(new_data, data, sizeof(ElemType) * size);
    } else {
        if (new_size < size) {
            size = new_size;
        }
        memcpy(new_data, data, sizeof(ElemType) * size);
    }
    delete[] data;
    data = new_data;
    capacity = new_size;
}\end{lstlisting}
\end{enumerate}
}

重分配的逻辑是:如果当前元素已满,则开辟一个数组大小为两倍原大小的数组,并将当前元素全部拷贝到新数组。%
当元素不足 $\frac{1}{4}$ 数组大小时,将数组大小减半。


\chapter{MinHeap 类}

保持堆顶的元素为最小元素。

\section{API}

\subsection{类定义}
{
\lstset{morendkeywords={MinHeap}}
\begin{lstlisting}[firstnumber=270, caption=MinHeap 类定义]{}
template<class ElemType, class Compare = Less<ElemType> >
class MinHeap {
public:
    MinHeap() : data() { }
    explicit MinHeap(const Vector<ElemType> &v);

    void sink(int root);
    void swim(int child);
    void push(const ElemType &x);
    void pop();
    void clear();

    ElemType &peek();
    size_t size();
    bool isEmpty();

private:
    Vector<ElemType> data;
};
\end{lstlisting}

两个模板参数 \lstinline{ElemType}、\lstinline{Compare}分别是堆中存放元素的类型和比较堆中元素的方式。\lstinline{Compare}类的%
对象重载了调用运算 \lstinline{bool operator()(const T &left, const T &right)},在函数中实例化 \lstinline{Compare} 并对对象使用%
函数调用可以获得一个布尔值\footnote{结果应该满足偏序关系},最小堆通过该布尔值实现排序。

\section{代码逻辑与实现}

\subsection{逻辑描述}

在二叉堆中,要保持元素有序,有两个重要的动作:下沉和上浮。其原理是:对于入堆的元素,原堆已经有序,将新元素放在一个叶子结点的位置,%
让该节点与其父节点比较并在非有序情况下交换他和父节点的位置,一次操作完成后就能得到新的有序堆;对于弹出的堆顶元素,取堆的叶子结点%
放在堆顶的位置,并让该元素与其子女节点比较,如果无序就交换两者,一次操作后也能达到有序。

\begin{figure}[H]
    \centering
    \begin{forest}
        [2 [3 [5] ] [4] ]
    \end{forest}
    \hspace*{1cm}
    \begin{forest}
        [2 [3 [5] [1, red]] [4]]
    \end{forest}
    \hspace*{1cm}
    \begin{forest}
        [2 [1, red [5] [3, red]] [4]]
    \end{forest}
    \hspace*{1cm}
    \begin{forest}
        [1, red [2, red [5] [3]] [4]]
    \end{forest}
    \caption{堆的元素上浮过程}
\end{figure}

红色的节点表示参与交换的节点,可以看出堆如何组织元素变的有序。

\begin{figure}[H]
    \centering
    \begin{forest}
        [1 [2 [5] [3]] [4]]
    \end{forest}
    \hspace*{1cm}
    \begin{forest}
        [3, blue  [2 [5] [{}, blue]] [4]]
    \end{forest}
    \hspace*{1cm}
    \begin{forest}
        [2, blue [3, blue [5] ] [4]]
    \end{forest}
    \hspace*{1cm}
    \begin{forest}
        [2 [3 [5] ] [4]]
    \end{forest}
    \caption{堆的元素下沉过程}
\end{figure}

蓝色的节点表示参与交换的节点,可以看出当堆顶元素1被弹出之后,堆用最后的元素补齐堆顶位置,然后再向下调整的过程。

在实现中,堆元素被存储在一个 \lstinline{Vector} 里,并使用二叉树的线性表存储方式来组织其中的元素位置。位置满足:

\begin{itemize}
    \item 根节点在 0 位置。
    \item n位置元素的父节点在 $\frac{n - 1}{2}$ 位置。
    \item n位置元素的左子女在 $2n + 1$ 位置。
    \item n位置元素的右子女在 $2n + 2$ 位置。
\end{itemize}

通过这些规律我们可以实现一个堆。

\subsection{代码实现}
以下是堆的代码实现:

\begin{itemize}
      \item 下沉
\begin{lstlisting} [firstnumber=301, caption = MinHeap::sink]
template<class ElemType, class Compare>
void MinHeap<ElemType, Compare>::sink(int root) {
    Compare com;
    int parent = root;
    size_t child = parent * 2 + 1; // 找到孩子
    while (child < data.length()) {
        if (child + 1 < data.length() &&
            com(data[child + 1], data[child])) {
            ++child; //这一层内的孩子
        }

        if (com(data[child], data[parent])) {
            swap(data[child], data[parent]);
            parent = child;// 交换
            child = parent * 2 + 1;
        } else {
            break;
        }
    }
}\end{lstlisting}
      \item 上浮
\begin{lstlisting}[firstnumber=322, caption=MinHeap::swim]
template<class ElemType, class Compare>
void MinHeap<ElemType, Compare>::swim(int child) {
    Compare com;
    int parent = (child - 1) / 2;
    while (parent >= 0) {
        if (com(data[child], data[parent])) {
            swap(data[parent], data[child]);
            child = parent;
            parent = (child - 1) / 2;
        } else {
            break;
        }
    }
}\end{lstlisting}

    \item 插入
\begin{lstlisting}[firstnumber=337]
template<class ElemType, class Compare>
void MinHeap<ElemType, Compare>::push(const ElemType &x) {
    data.pushBack(x);
    swim(data.length() - 1);
}\end{lstlisting}

    \item 弹出最小
\begin{lstlisting}[firstnumber=343]
template<class ElemType, class Compare>
void MinHeap<ElemType, Compare>::pop() {
    if (data.isEmpty())
        throw MinHeapException("Popping from empty heap.");
    swap(data[0], data[data.length() - 1]);
    data.popBack();
    sink(0);
}\end{lstlisting}
\end{itemize}


\chapter{BigInteger 类}

实现了加法和比较的大整型类。

\subsection{StringHashMap 类定义}
\lstset{morendkeywords={Pair}}
\begin{lstlisting}[firstnumber=386, caption=BigInteger 类定义]{}
class BigInteger {
    friend std::ostream &operator <<(std::ostream &, const BigInteger &);
public:
    explicit BigInteger(const string& i = "") :
            integer(check(i)) { }
    BigInteger &operator=(string s);
    ~BigInteger() = default;
    // 重载运算
    bool operator>(const BigInteger &other) const;
    bool operator<(const BigInteger &other) const;
    BigInteger &operator=(const BigInteger& other);
    bool operator==(const BigInteger &other) const;
    BigInteger operator+(const BigInteger &bi) const;

    static string check(const string &s);

protected:
    string integer = "";
};
\end{lstlisting}

}

大整型比较的方式是先比较两个数的位数,位数大的数值就大;如果位数相同,就从大到小逐位比较,先出现不同的数字,大者值更大。

加法则是用列竖式的方式计算的,将两个数个位对齐,从个位开始加,最后处理进位。


\chapter{System 类}

一个处理逻辑和与外界交互的封装类,存放学生信息并用上述结构完成了题目要求的功能。

\section{WoodCutter 类定义}

完成了图\ref{logic}的逻辑业务,并处理输入和输出。

\begin{lstlisting}[firstnumber=521, caption=System 类定义]{}
class WoodCutter {
public:
    explicit WoodCutter(const Vector<BigInteger> &vec);
    WoodCutter(BigInteger arr[], int n);
    WoodCutter() = default;
    BigInteger calculateCost(); // 计算花费
    void readAndRun(std::istream& is, int n);
                                // 输入输出
protected:
    MinHeap<BigInteger> heap;
    BigInteger result;
    bool hasResult = false;
};
\end{lstlisting}

在此不对封装做过多介绍。

\chapter{TestCases 类}

\lstinline{TestCases} 类中包含了几种极端情况和错误输入的测试。

\section{TestCases 类定义}

\begin{lstlisting}[firstnumber=609, caption=TestCases 类定义]{}
class TestCases {
public:
    void run();
private:
    static constexpr int caseNum = 10;
    string tests[caseNum] = {
            "1 9",
            "10 7 3 5 13 9 11 45 97 57 31",
            "10 12 42 4 6 86 46 80 78 54 2",
            "20 1 2 3 4 5 6 7 8 9 10 11 12 13 14 15 16 17 18 19 20",
            "3 273646485859594887373722312323 27364648585959488737372231232 7",
            "0 ",
            "8 2 1 3 9 4 11 13 15",
            "8 4 5 1 2 1 3 1 1",
            "1 .",
            "19 1"
    };
};
\end{lstlisting}

解除对应行注释可以使用测试用例:

\begin{lstlisting}[firstnumber=648, caption=使用测试用例]{}
//#define TEST_CASES
\end{lstlisting}

测试结果在下一部分展示。