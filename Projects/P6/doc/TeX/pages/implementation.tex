
\chapter{类结构}
由于使用了哈希表,本项目的代码结构较为复杂,因此加以解释。\lstinline{HashTable} 类是仿照 \emph{C++ STL} 源码实现的,其中使用了%
\lstinline{Vector} 类,所以代码中出现在上面的类型(本应该被 \lstinline{#include} 进来)是下面类型的依赖,而不是更重要的意思。%
下面按照一定的逻辑顺序介绍这几个类:

\begin{itemize}
    \item \lstinline{MultiTree} 类\\
    由二叉树存储的多叉树类,遵循“左长子右兄弟”的规则进行存放。每个节点含有三个指针域,分别指向长子、兄弟和父母。
    \item \lstinline{HashTable} 类\\
    其中维护了一个 \lstinline{HashNode} 的 \lstinline{Vector},通过其哈希函数(\lstinline{hash function})完成对\lstinline{Vector}%
    数组的索引。主要通过\lstinline{find} 和 \lstinline{hasKey}索引和判断键的存在与否,通过 \lstinline{insert} 方法插入一个键唯一的对象。%
    通过 \lstinline{erase} 方法抹去一个值。
    \item \lstinline{StringHashMap} 类\\
    包装和特化之后的一个哈希键值对集合,通过字符串进行索引得到对应值,范型参数为值的类型。可以通过\lstinline{find} 和 %
    \lstinline{hasKey}索引和判断键的存在与否,通过 \lstinline{insert} 方法插入一个键唯一的对象。%
    通过 \lstinline{erase} 方法抹去一个值。并且重载了下标运算符 \lstinline{[]},将字符串作为下标可以索引对应的值。
    \item \lstinline{Vector} 类\\
    \lstinline{HashTable} 类用与存放\lstinline{HashNode}的数组,相同哈希值的对象会以链表的方式存放在这个数组的单元里。%
    在\lstinline{HashTable} 类的实现中主要用到了下标运算、\lstinline{reserve(n)} 方法:开辟 $n$ 长度的数组。\lstinline{resize(n)}%
    方法:重新分配 $n$ 长度的内存。
    \item \lstinline{GenealogyTreeSystem} 类\\
    家谱管理系统封装类,使用了一个多叉树和一个哈希map来管理家谱的结构,并实现了前述功能。
\end{itemize}

该部分主要详细介绍\lstinline{MultiTree} 类和 \lstinline{HashTable} 哈希表类的实现。


\chapter{MultiTree 类}

\section{API}

\subsection{类定义}
以下是本题使用队列的 API,各函数的功能已用注释标出。
\begin{lstlisting}[morendkeywords={MultiTree}, firstnumber=254, caption=MultiTree 类定义]{}
template <class Type>
class MultiTree {
public:
    struct Node {  
        Node(const Type& d, Node* par,
             Node * child = nullptr, Node * sib = nullptr) :
                data(d), firstChild(child), 
                sibling(sib), parent(par) { }
        Type data;
        Node * firstChild;
        Node * sibling;
        Node * parent;      // 添加子女、兄弟
        Node * insertChild(const Type &d);
        Node * insertSibling(const Type &d);
        Node *leftMostOf();
        Node *rightMostOf();
    };

public:
    explicit MultiTree(const Type &d) {
        root = new Node(d, nullptr);
    }
    ~MultiTree();
    static Node *insertChild(Node *node, const Type &d);
    static Node *insertSibling(Node *node, const Type &d);
    Node *trivialFind(const Type& d);
    std::ostream &show(std::ostream &os);
    int dismissSubTree(Node *&node);    // 删除子树,保存姓名
    int dismissSubTree(Node *&node, Vector<Type> &valList);
    Node *getRoot() const;              // 打印树
    void showNode(struct Node* node, int indent);

private:
    Node * root;
    Vector<int> indentVec;
    static bool deleteLeave(Node *&node, Node *&parent);
    static Node *recursiveFind(Node *, const Type &d);
    static void showNode(Node *, std::ostream & os);
};
\end{lstlisting}

{
\lstset{
    basicstyle=\ttfamily\color{CPPDark}, 
    numberstyle=\tiny\color{darkgray},
    keywordstyle=\color[RGB]{40,40,255},
    }
\subsection{公开成员}
\begin{enumerate}
    \item \lstinline{Node *insertChild(Node *node, const Type &d);}\\
          对一个节点增加子女的操作,向子女方向走一次,走到节点的长子节点,再沿着兄弟方向走到底,在空指针处添加这个新节点。%
          该函数委托了节点的 \lstinline{insertChild} 方法。
    \item \lstinline{Node *insertSibling(Node *node, const Type &d);}\\
          对一个节点的增加兄弟操作,节点向兄弟方向走到底,在空指针的位置创建新的节点。%
          该函数委托了节点的 \lstinline{insertSibling} 方法。
    \item \lstinline{int dismissSubTree(Node *&node, Vector<Type> &valList);} \\
          解散 \lstinline{node} 的家庭,具体做法在\ref{dismiss} 中描述。
    \item \lstinline{Node *trivialFind(const Type& d);} \\
          遍历找到值为 \lstinline{x} 的节点的位置,找不到返回 \lstinline{nullptr}。该方法没有被使用,因为有哈希map。
\end{enumerate}

\subsection{私有成员}
\begin{enumerate}
    \item \lstinline{Node * root;} \\
          根节点。
    \item \lstinline{static bool deleteLeave(Node *&node, Node *&parent);} \\
          删除叶子结点的函数,在 \lstinline{dismissSubTree} 中调用。
\end{enumerate}
}

\section{代码逻辑与实现}

逻辑已经在设计中概括,本部分展现实现功能的代码。

{
\lstset{
      morendkeywords={MultiTree, Node},
      morendkeywords = {firstChild, root, }
}
\begin{enumerate}
      \item 给一个节点添加一个子女(静态成员)
\begin{lstlisting} [firstnumber = 365, caption={MultiTree::insertChild}]
template<class Type>
typename MultiTree<Type>::Node *
MultiTree<Type>::insertChild(Node *node, const Type &d) {
    auto child = node->firstChild;
    Node* ret;
    if (!child) {
        ret = node->insertChild(d);
    } else {
        child = child->rightMostOf(); // 右到底
        ret = child->insertSibling(d);
    }
    if (ret == nullptr)
        throw MultiTreeException("Cannot insert child");
    return ret;
} \end{lstlisting}
      \item 给节点添加一个兄弟(静态成员)
\begin{lstlisting} [firstnumber = 381, caption={MultiTree::insertSibling}]
template<class Type>
typename MultiTree<Type>::Node *
MultiTree<Type>::insertSibling(Node *node, const Type &d) {
    auto sib = node->rightMostOf(); // 右到底
    auto ret = sib->insertSibling(d);
    if (ret == nullptr)
        throw MultiTreeException("Cannot insert sibling");
    return ret;
}\end{lstlisting}
    \item 非递归解散节点下的家庭,返回值为删除节点的个数, \lstinline{valList} 存放删除的节点存放的值。
\begin{lstlisting} [firstnumber = 293, caption={MultiTree::dismissSubTree}]
template<class Type>
int
MultiTree<Type>::dismissSubTree(Node *&node, Vector<Type> &valList) {
    if (node == nullptr)
        return 0;
    if (node == root && root->sibling == nullptr) {
        throw MultiTreeException("Cannot delete only root");
    }
    auto *child = node->firstChild;
    auto *p = child;
    Node *temp;
    int count = 0;
    if (child != nullptr) {
        while (child->sibling || child->firstChild) {
            if (p->sibling) {
                p = p->sibling;
                continue;
            } else if (p->firstChild) {
                p = p->firstChild;
                continue;
            } else {
                Node* parent;
                valList.pushBack(p->data);
                if (!deleteLeave(p, parent)) {
                    throw MultiTreeException("Fail deleting leave.");
                }
                ++count;
                p = parent;
            }
        }
        valList.pushBack(child->data);
        deleteLeave(child, temp);
        ++count;
    }
    if (node->sibling) {
        if (node->parent)
            node->parent->sibling = node->sibling;
        else
            root = node->sibling;
        node->sibling = nullptr;
    } else {
        node->parent->sibling = nullptr;
    }
    valList.pushBack(node->data);
    node->parent = nullptr;
    deleteLeave(node, temp);
    ++count;
    node = nullptr;
    return count;
}\end{lstlisting}

\end{enumerate}
}

\chapter{HashTable 类}

仿照了 C++ STL 的哈希表实现。

\section{API}

\subsection{类定义}
{
\lstset{morendkeywords={HashTable}}
\begin{lstlisting}[firstnumber=550, caption=HashTable 类定义]{}
template<class ValueType, class Key, class HashFunction,
        class ExtractKey, class EqualKey>
class HashTable {
public:
    explicit HashTable(size_t n);

    inline size_t maxBucketNum() const;      // 最大容量
    inline size_t bucketCount() const;       // 当前容量
    bool insertUnique(const ValueType& obj); // 插入值
    void clear();                            // 清空
    ValueType & find(const Key&);            // 用键查找
    ValueType & findOrInsert(const ValueType& obj);     // 查找或插入值
    bool hasKey(const Key& k);          // 查键记录
    inline bool empty() const;          // 是否为空
    size_t erase(const Key& key);       // 抹去一个键值
    void resize(size_t hint);           // 重分配容量

private:
    typedef HashNode<ValueType> node;   // 节点
    typedef HashFunction hasher;        // 哈希函数类
    typedef EqualKey equal_key;         // 判断等值类

    hasher hash;            // 哈希函数对象
    ExtractKey getKey;      // 获取键函数对象
    equal_key equals;       // 判断等值函数对象

    Vector<node *> buckets; // 存值的容器
    size_t num_elements;    // 元素个数
    static constexpr int num_primes = 28;    // 见解释
    static const unsigned long prime_list[num_primes];
        // 找质数、调用哈希函数
    inline unsigned long next_prime(unsigned long n);
    inline size_t findBucketKey(const Key &key, size_t size) const;
    inline size_t findBucket(const ValueType &obj) const;
    inline size_t findBucket(const ValueType &obj, size_t size) const;
};
\end{lstlisting}


\lstinline{HashTable} 有五个模板参数:\lstinline{ValueType} 为存放值类型,\lstinline{Key} 为键类型\footnote{这里的键、值关系并非%
键值对的关系,\lstinline{HashTable} 的值通过\lstinline{ExtractKey}类可以取出键,他们是包含关系而不是关联。},%
\lstinline{HashFunction} 可算出 \lstinline{Key}对象对应的哈希值, \lstinline{ExtractKey} 取出 \lstinline{ValueType} 中的键,%
\lstinline{EqualKey} 可以比较两键是否相等。

\section{代码逻辑与实现}

\subsection{逻辑描述}

哈希表是一个查找和添加元素平均开销均为 $O(1)$ 的数据结构,其特点是数据的存储方式为将数据分散在数组里,并以数组的下标来表征元素的特征。元素的值%
与下标之间的关系满足函数关系,也就是对于任意的元素取值都能找到哈希表中的唯一下标与之对应。由于数据的分散特征,哈希表又称{散列表}。在哈希表中查找一个%
数据可称作“索引”,数据一定要有{\kaishu 可哈希性(可散列性)},即该数据的{\kaishu 键(Key)}可以通过一个函数获得一个唯一个值,%
相同的键的函数值也相同。由于一般的哈希函数算出的值都是非负整数,很难保证不同键的哈希值不相同(如果这样就能构成一一对应关系),%
因此要实现哈希表还要解决{\kaishu 冲突}问题.

C++ STL 的哈希表采用了“开链”方式组织解决冲突。“开链”之意为在表的单元中存放链表的指针,并将应该放在该表中的元素用链表延伸出去。如图 \ref{hash} 所示,%
一个单元中有元素就存放首节点的指针,以单链表延伸;如果没有元素,就要存放空指针。这样,一个哈希表所有键冲突元素都可以存在一个单元中,并且查找某个哈希值对应的%
键值是否存在表中时,只需要单向遍历这个单元的链表就行。理想情况下,键与哈希值是均匀分布的,每个单元只存储一个键;最坏条件下,哈希表退化成链表,查找开销为 $O(n)$。

以下是流程图:

\newpage

{
    \begin{figure}[H]
        \centering
        \begin{tikzpicture}
            \node (start) [startstop]  {开始};
            \node [io, below=.6cm of start] (input) {输入键 \lstinline{key}};
            \node [process, below=.6cm of input] (hash) {算出 \lstinline{key} 的哈希值 \lstinline{hash}};
            \node [process, below=.6cm of hash] (get) {\lstinline{hash} 下标位置的节点指针 \lstinline{p}};
            \node [condition, below=.6cm of get] (cond1) {\lstinline{p} 不为空};
            \node [condition, below=.6cm of cond1] (cond2) {\lstinline{p->key} != \lstinline{key}};
            \node [process, below=.6cm of cond2] (find) {返回\lstinline{p}节点的值};
            \node [process, below=.6cm of cond2, xshift = 4cm] (notFound) {返回空值};
            \node [startstop, below=.6cm of find] (end) {结束};

            \draw [arr] (start) -- (input);
            \draw [arr] (input) -- (hash);
            \draw [arr] (hash) -- (get);
            \draw [arr] (get) -- (cond1);
            \draw [arr] (cond1) -- node[anchor=east] {是} (cond2);
            \draw [arr] (cond2) -- node[anchor=east] {否} (find);
            \draw [arr] (find) -- (end);
            \draw [arr] (notFound.south) |- (end);

            \draw [arr] (cond1) -- ($(cond1.east) + (0.9, 0)$) node[anchor = south] {否} -| (notFound.north);
            \draw [arr] (cond2) -- ($(cond2.west) + (-0.5, 0)$) node[anchor = east] {是} |- ($ (cond1.north) + (0, 0.3) $);

        \end{tikzpicture}
        \caption{查找HashTable中的值}
    \end{figure}
}

{
    \begin{figure}[H]
        \centering
        \begin{tikzpicture}
            \node (start) [startstop]  {开始};
            \node [io, left=1.0cm of start] (input) {输入值 \lstinline{value}};
            \node [process, below=.6cm of input] (hash) {算出 \lstinline{value} 的哈希值 \lstinline{hash}};
            \node [process, below=.6cm of hash] (get) {\lstinline{hash} 下标位置的节点指针 \lstinline{p}};
            \node [process, below=.6cm of get] (find) {\lstinline{value} 生成节点插入 \lstinline{p} 链表的首位置};
            \node [startstop, right=1.0cm of find] (end) {结束};

            \draw [arr] (start) -- (input);
            \draw [arr] (input) -- (hash);
            \draw [arr] (hash) -- (get);
            \draw [arr] (get) -- (find);
            \draw [arr] (find) -- (end);

        \end{tikzpicture}
        \caption{添加值到HashTable}
    \end{figure}
}

另外,\lstinline{HashTable} 的数组是自动分配大小的,因此采用了 \lstinline{Vector} 类的 \lstinline{resize} 函数。%
\lstinline{HashTable} 扩充内存的逻辑是:当含有元素个数大于当前数组的大小(单元个数)时,就将容量扩大大约一倍。实际上 \lstinline{HashTable} %
中的数组大小取值为一系列呈两倍递增的质数,因为模取质数获得的下标能尽可能保证均匀。

\subsection{代码实现}

下面展示\lstinline{HashTable}关键代码。(省略了范型参数)

\begin{itemize}
      \item \lstinline{ValueType &HashTable<..>::find(const Key &k)}\\
            找到 \lstinline{key} 的值。
\begin{lstlisting} [firstnumber=713]
template<..>
ValueType &
HashTable<..>::
find(const Key &k) {
    size_t n = findBucketKey(k, buckets.length());
    node *first = buckets[n];   // 索引头节点
    if (first == nullptr) throw HashTableException("Key Not Exist");
    for (node *cur = first; cur; cur = cur->next) {
        if (equals(getKey(cur->value), k)) {
            return cur->value;  // 找到
        }
    } // 没找到
    throw HashTableException("Key Not Exist");
}\end{lstlisting}
      \item \lstinline{bool HashTable<..>::insertUnique(const ValueType &obj);} \\
            插入一个值。
\begin{lstlisting}[escapechar=^, firstnumber=679]
template<..>
bool
HashTable<..>::
insertUnique(const ValueType &obj) {
    resize(num_elements + 1);   // 检查是否扩充
    const size_t pos = findBucket(obj);
    node *first = buckets[pos]; // 用哈希值索引

    for(node *cur = first; cur; cur = cur->next) {
        if (equals(getKey(obj), getKey(cur->value)))
            return false;       // 重复不添加
    }
    // 添加
    auto temp = new node(obj);
    temp->next = first;
    buckets[pos] = temp;
    ++num_elements;
    return true;
}\end{lstlisting}
\end{itemize}


\section{StringHashMap 类}

用 \lstinline{HashTable} 存放了 \lstinline{Pair<std::string, Type>} 类型的封装,可以用与存放键值对,通过字符串键索引值。%
因为都是调用 \lstinline{HashTable} 的方法,这里只展示类定义 API。


\subsection{Pair 类定义}
\lstset{morendkeywords={Pair}}
\begin{lstlisting}[firstnumber=817, caption=Pair 类定义]{}
template<class First, class Second>
struct Pair {
    typedef First firstType;
    typedef Second SecondType;

    First first;
    Second second;

    Pair() : first(First()), second(Second()) {}
    Pair(const First &t1, const Second &t2) : first(t1), second(t2) {}
    Pair(const Pair &p) : first(p.first), second(p.second) { }

    Pair &operator=(const Pair& p) { // 重载等号
        if (&p == this)
            return *this;
        first = p.first; second = p.second;
        return *this;
    }
};
\end{lstlisting}

\subsection{StringHashMap 类定义}
\lstset{morendkeywords={Pair}}
\begin{lstlisting}[firstnumber=853, caption=StringHashMap 类定义]{}
template<class Value>
class StringHashMap {
public:
    typedef Pair<string, Value> MapPair; // 别名

    StringHashMap() : ht(50) { }
    // 委托调用
    size_t size() const { return ht.bucketCount(); }
    size_t maxSize() const { return ht.maxBucketNum(); }
    bool empty() const { return ht.empty(); }
    
    inline Value &operator[](const string &s);
    MapPair &find(const string &s) { return ht.find(s); }
    const MapPair &find(const string &s) const { return ht.find(s); }
    bool hasKey(const string &s) { return ht.hasKey(s); }
    size_t erase(const string& key) { return ht.erase(key); }
    void resize(size_t hint) { ht.resize(hint); }
    inline void insert(const string& s, const Value& val);
    inline void insert(const MapPair& pair);

private:    // 用模板参数具体化 HashTable
    HashTable<
            MapPair,
            string,
            Hash<string>,
            GetPairKey<MapPair, string>,
            Equals<string> > ht;
};
\end{lstlisting}

}


\chapter{GenealogyTreeSystem 类}

一个处理逻辑和与外界交互的封装类,存放家谱信息结构并用上述结构完成了题目要求的功能。

\section{GenealogyTreeSystem 类定义}
\lstset{morendkeywords={Pair, GenealogyTreeSystem}}
\begin{lstlisting}[firstnumber=911, caption=GenealogyTreeSystem 类定义]{}
class GenealogyTreeSystem {
public:

    GenealogyTreeSystem() : tree("") { }

    void run();
    bool parseCommand(char c);
    void addAncestor();     // 加祖先
    void addChild();        // 加子女
    void dismissSubFamily();// 解散家庭
    void rename();          // 改名
    static void showMenu();

private:
    typedef MultiTree<string> TreeType;
    typedef TreeType::Node NodeType;
    TreeType tree;
    StringHashMap<NodeType *> regMap;
    StringHashMap<int> collisionCount;

private:
    static std::istream &clearInput(std::istream &is);
    static void notValidCommand(const string &s);
    string collisionHandle(const string &name);
};
\end{lstlisting}

其余部分前面都有提及,这里介绍姓名冲突下的处理:

\section{重名处理}

家谱树存放且仅存放一个姓名信息,由于真实家谱中也存在重名问题,因此要求用户重新输入重复的名字不够人性化。%
姓名不能当作唯一识别的标志,保存相同的姓名又回使得搜索产生多义,使结构变得复杂。因此我用哈希map的功能%
设计了一个姓名冲突处理功能,其工作逻辑是:

{
    \begin{figure}[H]
        \centering
        \begin{tikzpicture}
            \node (start) [startstop] {开始};
            \node [io, below=.6cm of start] (input) {输入姓名 \lstinline{n}};
            \node [condition, below=.6cm of input] (cond) {\lstinline{n} 已存在?};
            \node [condition, below=.6cm of cond] (cond2) {重名表存在姓名 \lstinline{n}?};
            \node [process, left=.7cm of cond2] (insert) {重名表插入 \lstinline{<n, 1>}};
            \node [process, below=.6 of insert] (one) { \lstinline{m = 1}};
            \node [process, below=.6cm of cond2] (get) {取出重名表中的次数 m};
            \node [process, below=.6 of get] (add) {\lstinline{m = m + 1}};
            \node [condition, below=1cm of add] (cond3) {\lstinline{n + m} 存在?};
            \node [startstop, below=.6cm of cond3] (end) {返回\lstinline{n + m}};


            \draw [arr] (start) -- (input);
            \draw [arr] (input) -- (cond);
            % \draw [arr] (find) -- (cond);
            \draw [arr] (cond) -- node[anchor=east] {是} (cond2);
            \draw [arr] (cond2) -- node[anchor=east] {是} (get);
            \draw [arr] (get) -- (add);
            \draw [arr] (add) -- (cond3);
            \draw [arr] (cond3) --  node[anchor=west] {否} (end);
            % \draw [arr] (add) -- (end);
            \draw [arr] (cond2) -- node[anchor=south] {否} (insert);
            \draw [arr] (insert) -- (one);
            \draw [arr] (one) |- ($(cond3.north) + (0, 0.5)$);
            \draw [arr] (cond3) -- ($(cond3.east) + (0.9, 0)$) node[anchor=north] {是}  |- ($(add.north) + (0, 0.3)$);

        \end{tikzpicture}
        \caption{重名处理逻辑}
    \end{figure}
}

处理的代码如下:

\begin{lstlisting}[firstnumber=1096, caption=GenealogyTreeSystem::collisionHandle]{}
string GenealogyTreeSystem::collisionHandle(const string &name) {
    int n;
    if (collisionCount.hasKey(name)) {
        n = ++collisionCount[name];
        while (regMap.hasKey(name + std::to_string(n))) {
            n = ++collisionCount[name];
        }
    } else {
        collisionCount.insert(name, 1);
        n = 1;
        while (regMap.hasKey(name + std::to_string(n))) {
            n = ++collisionCount[name];
        }
    }
    string newName = name + std::to_string(n);
    cout << "Same name! " << name
         << " renamed to " << newName << endl;
    return newName;
}
\end{lstlisting}