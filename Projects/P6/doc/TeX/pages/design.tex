
\chapter{背景分析}


家谱:又称族谱、宗谱等。是一种以表谱形式,记载一个家族的世系繁衍及重要人物事迹的书。家谱属珍贵的人文资料,%
对于历史学、民俗学、人口学、社会学和经济学的深入研究,均有其不可替代的独特功能。

随着电子时代的来临,传统纸媒介记载逐渐被替代,如果能用计算机存储这些家谱数据,便利的同时历史的遗产也将被保留,%
可谓一举多得。

家中最重要的信息是家庭的{\kaishu 层级关系(Hierarchy)},即谁是谁的子女,谁是谁的父母。一个家长(父母)和子女的关系%
是一对多的:一个家长可以有多个子女,但是一个人只能有一父一母。因此一个家谱层级关系可以映射到一颗多叉树上,每个节点存放一个人。

本项目要求的功能与 {\kaishu 考试报名管理系统} 相似,要实现增、删、查、改四个功能。下面是一个家谱树示意:

\forestset{ 
  default preamble={ 
    for tree={circle,draw, minimum width=0.8cm} 
  } 
} 

\begin{figure} [H]
    \centering
    \begin{forest}
        [卓负一 [卓正一 [卓斜一] [卓斜二] ] [卓正二 [卓斜三] ] [卓正三 [卓斜四] [卓斜五] [卓斜六]] ]
    \end{forest}
    \caption{家谱树示意图}
\end{figure}


\chapter{功能设计}

\section{题意转化}
该题的目的明确,要用一个能够模拟多叉树的数据结构将家谱内的层级信息储存起来,并能够做到{\kaishu 增、删、查、改}四个功能。%
在实现上,能够模拟多叉树结构的数据结构,常用的有两种:一个是在节点内存放一个能够存储其所有子节点的线性结构(一般是 Vector),%
这样的多叉树在逻辑结构上与存储的真实多叉树一致,因为其节点有真是的多叉结构。另一种做法是用二叉树存放多叉树,一个节点的一个指针域%
存放与自己同级的节点,另一个域存放自己的子女级节点,这样任何多叉树到二叉树都存在唯一的映射;这种存储结构被称为“左长子右兄弟”方式。

用两幅图可以看出这两者的对应关系:

图\ref{multi}是一棵多叉树,也是直接存储多叉树法的数据内部逻辑结构。图\ref{binary}是\ref{multi}对应的二叉树表示方式。本项目选用%
“左长子右兄弟”法来储存家谱的层级结构。


\begin{figure} [H]
    \centering
    \begin{forest}
        [卓负一 [卓正一 [卓斜一] [卓斜二] ] [卓正二 [卓斜三] ] [卓正三 [卓斜四]] ]
    \end{forest}
    \caption{多叉树存储逻辑 - 直接模拟法}
    \label{multi}
\end{figure}

\newpage

\begin{figure} [H]
    \centering
    \begin{forest}
        [卓负一 [卓正一 [卓斜一 [卓斜二] [NULL] ] [卓正二 [卓斜三] [卓正三 [卓斜四] [NULL] ] ] ] [NULL] ]
    \end{forest}
    \caption{多叉树存储逻辑 - “左长子右兄弟”法}
    \label{binary}
\end{figure}


在二叉树中,可以通过递归的方式遍历查找某个节点,遍历每次的时间复杂度 $O(n)$。本项目中几乎所有操作都需要查找一个已经存在的%
家庭成员,$O(n)$的时间复杂度难以接受,我决定加用{\kaishu 哈希表}\emph{(Hash Table)}辅助进行查找,%
哈希表的插入、删除和搜索时间复杂度均可看作 $O(1)$ 级别,其特点是空间复杂度为 $\Omega(n)$\footnote{$\Omega$ 为下界},其中 $n$ 为存储%
在表内的数据数量。本项目中,用字符串索引的哈希map(\lstinline{StringHashMap})用于保存已经存在家谱树中的姓名和所在节点的键值对。下图是哈希表的储存结构:

{
\begin{figure}[H]
    \tikzset{
        listnode/.style={
            rectangle split,rectangle split parts=2,draw,rectangle split horizontal,fill=blue!20
        },
        startnode/.style={
            draw,minimum width=1.5cm,minimum height=.75cm
        }
    }
    \centering
    \begin{tikzpicture}[scale=.2, >=stealth]
        \node[startnode] (t0) {$T[0]$};
        \node[startnode,below=0pt of t0] (t1) {$T[1]$};
        \node[startnode,below=0pt of t1] (t2) {$T[2]$};
        \node[startnode,below=0pt of t2] (t3) {$T[3]$};
        \node[startnode,below=0pt of t3] (t4) {$T[4]$};
        \node[startnode,below=0pt of t4] (t5) {$T[5]$};
        \node[listnode,right=of t0] (3) {3};
        \node[listnode,right=of t1] (1) {1};
        \node[listnode,right=of t2] (2) {2};
        \node[listnode,right=of t5] (13) {13};
        \node[listnode,right=.5cm of 2] (5) {5};
        \node[listnode,right=.5cm of 5] (8) {8};
        \node[right=.5cm of 3] (3x) {$\varnothing$};
        \node[right=.5cm of 1] (7x) {$\varnothing$};
        \node[right=.5cm of 8] (8x) {$\varnothing$};
        \node[right=1cm of t3] (3xx) {$\varnothing$};
        \node[right=1cm of t4] (4xx) {$\varnothing$};
        \node[right=.5cm of 13] (13x) {$\varnothing$};
        \draw[*->] let \p1 = (3.two), \p2 = (3.center) in (\x1,\y2) -- (3x);
        \draw[*->] let \p1 = (1.two), \p2 = (1.center) in (\x1,\y2) -- (7x);
        \draw[*->] let \p1 = (2.two), \p2 = (2.center) in (\x1,\y2) -- (5);
        \draw[*->] let \p1 = (5.two), \p2 = (5.center) in (\x1,\y2) -- (8);
        \draw[*->] let \p1 = (8.two), \p2 = (8.center) in (\x1,\y2) -- (8x);
        \draw[*->] let \p1 = (13.two), \p2 = (13.center) in (\x1,\y2) -- (13x);
        \draw[->] (t0) edge (3) (t1) edge (1) (t2) edge (2) (t3) edge (3xx) (t4) edge (4xx) (t5) edge (13);
    \end{tikzpicture}
        \caption{哈希表结构示意图\label{hash}}
\end{figure}
}


\section{逻辑功能}

为了实现{\kaishu 增、删、查、改}四个功能,我们要维护两套数据。一份放在多叉树里,另一份放在哈希map里。%


需要实现的功能有以下四个:

\begin{itemize}
    \item 增加祖先。
    \item 给指定成员添加子女。
    \item 解散以指定成员为根节点的家庭。
    \item 更改给定成员的姓名。
\end{itemize}

增加祖先的操作十分直接:在存储二叉树的根节点向兄弟方向走到底,在空指针的位置创建新的节点就是一个祖先。%
创建的新节点需要加入哈希 map 中。

添加子女和删除是需要查找的,下面给出他们的功能流程图。

\tikzstyle{startstop} = [rectangle, rounded corners, minimum width = 2cm, 
        minimum height=1cm, text centered, draw = black, fill = red!40,
        font = {\bfseries}]
    \tikzstyle{io} = [trapezium, trapezium left angle=70, trapezium right angle=110, 
    minimum width=2cm, minimum height=1cm, text centered, draw=black, fill = blue!40]
    \tikzstyle{process} = [rectangle, minimum width=3cm, minimum height=1cm, text centered, draw=black, fill = yellow!50]
    \tikzstyle{condition} = [diamond, aspect = 3, text centered, draw=black, fill = green!30]
    \tikzstyle{arr} = [->, >=stealth, thick]

\subsection{添加子女}

对于一个存储二叉树节点,给它添加子女的操作是:向子女方向走一次,走到节点的长子节点,再沿着兄弟方向走到底,在空指针处添加这个新节点。
有了这个操作,添加子女的流程图为:

\newpage

{
    \begin{figure}[H]
        \centering
        \begin{tikzpicture}
            \node (start) [startstop]  {开始};
            \node [io, below=.6cm of start] (input) {读取到 \lstinline{n}};
            \node [condition, below=.6cm of input] (cond) {\lstinline{n} 姓名已存在?};
            \node [io, below=.6cm of cond] (input2) {读取子女 \lstinline{c}};
            \node [condition, below=.6cm of input2] (cond2) {\lstinline{c} 姓名已存在?};
            \node [process, below=.6cm of cond2] (toList) {添加子女 \lstinline{c}};
            \node [process, below=.6cm of toList] (toMap) {\lstinline{c} 加入哈希表};
            \node [startstop, below=.6cm of toMap] (end) {结束};
            \node [process, left=.8cm of cond] (conf) {重名处理};
            \node [process, right=.8cm of cond2] (conf2) {重名处理};

            \draw [arr] (start) -- (input);
            \draw [arr] (input) -- (cond);
            \draw [arr] (cond) -- node[anchor=west] {否} (input2);
            \draw [arr] (cond) -- node[anchor=south] {是} (conf);
            \draw [arr] (conf) |- ($(input2.north) + (0, 0.3)$);
            \draw [arr] (input2) -- (cond2);
            \draw [arr] (cond2) -- node[anchor=east] {否} (toList);
            \draw [arr] (cond2) -- node[anchor=south] {是} (conf2);
            \draw [arr] (conf2) |- ($(toList.north) + (0, 0.3)$);
            \draw [arr] (toList) -- (toMap);
            \draw [arr] (toMap) -- (end);

        \end{tikzpicture}
        \caption{增加子女操作流程图}
    \end{figure}
}

\subsection{查找考生信息}

在存储二叉树中删除子树需要从底向上删除,具体做法是:当一个节点没有兄弟和长子节点的时候(两者为空0,可以删除,并将当前节点置为这个节点的父母节点;%
如果有存在子女或兄弟,则要进入存在的下级节点;当返回到要删除子树的根节点时说明删除完成。但是要注意,在家谱中解散 A 的局部家庭,我们不希望将%
A 的兄弟的局部家庭也解散。所以我们要在 A 的长子节点上做遍历删除,删除后移除长子,并将 A 的父母指针接向 A 的下一个兄弟节点。\label{dismiss}

有了这个操作,我们可以简单画出删除部分的流程:

\newpage

{
    \begin{figure}[H]
        \centering
        \begin{tikzpicture}
            \node (start) [startstop]  {开始};
            \node [io, below=.7cm of start] (input) {输入姓名 \lstinline{n}};
            \node [condition, below=.7cm of input] (cond) {\lstinline{n} 姓名};
            \node [process, below=.7cm of cond, xshift=-2.75cm] (find) {找到对应节点};
            \node [process, below=.7cm of find] (out) {删除子树};
            \node [io, below=.7cm of cond, xshift=2.75cm] (outErr) {输出错误提示};
            \node [startstop, below=.6cm of out, xshift=2.5cm] (end) {结束};

            \draw [arr] (start) -- (input);
            \draw [arr] (input) -- (cond);
            \draw [arr] (cond) -|  node[anchor=east] {是}  (find.north);
            \draw [arr] (find) -- (out);
            \draw [arr] (out) -- (end);
            \draw [arr] (outErr) -- (end);

            \draw [arr] (cond) -|  node[anchor = west] {否}  (outErr.north);

        \end{tikzpicture}
        \caption{解散局部家庭操作流程图}
    \end{figure}
}

\subsection{修改姓名}
修改考生信息除了要先找到考生之外,还有一些对哈希表操作的细节。流程图如下:

重点在于哈希表存放了{\kaishu <姓名, 节点指针>}的键值对,修改后原姓名应该不能找到任何节点,所以要先删除,修改完成后再将该{\kaishu <新姓名, 节点指针>}对加入回哈希表,完成一次修改。%

\newpage

{
    \begin{figure}[H]
        \centering
        \begin{tikzpicture}
            \node (start) [startstop] {开始};
            \node [io, below=.6cm of start] (input) {输入姓名 \lstinline{n}};
            \node [process, below=.6cm of input] (find) {用 \lstinline{n} 查询成员 \lstinline{m}};
            \node [condition, below=.6cm of find] (cond) {\lstinline{m} 不为空?};
            \node [process, below=.6cm of cond] (erase) {\lstinline{m} 从哈希表抹去};
            \node [io, below=.7cm of erase] (input2) {读取到 \lstinline{m}};
            \node [condition, below=.6cm of input2] (cond2) {\lstinline{m} 姓名已经存在?};
            \node [process, below=.6cm of cond2] (add) {\lstinline{S} 加入哈希表};
            \node [startstop, below=.6cm of add] (end) {结束};
            \node [process, left=.6cm of cond2] (conf2) {重名处理};

            \draw [arr] (start) -- (input);
            \draw [arr] (input) -- (find);
            \draw [arr] (find) -- (cond);
            \draw [arr] (cond) -- node[anchor=east] {是} (erase);
            \draw [arr] (erase) -- (input2);
            \draw [arr] (input2) -- (cond2);
            \draw [arr] (cond2) --  node[anchor=west] {否} (add);
            \draw [arr] (add) -- (end);

            \draw [arr] (cond) -- ($(cond.east) + (0.9, 0)$) node[anchor = west] {否} |- ($ (end.north) + (0, 0.3) $);
            \draw [arr] (cond2) -- node[anchor=south] {是} (conf2);
            \draw [arr] (conf2) |- ($(add.north) + (0, 0.3)$);

        \end{tikzpicture}
        \caption{修改现有成员名字流程图}
    \end{figure}
}

