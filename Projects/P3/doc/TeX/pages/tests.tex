
\chapter{异常}
在输入时,不合法的字符会引起一些异常来指示当前程序的状态。通过处理这些异常可以改变程序的执行序列,在出错之前%
将程序的执行权交给处理错的代码。本例的数据结构和系统都有异常类来提供必要的信息。其中一个例子是非查看栈顶异常:
\begin{lstlisting}[firstnumber=244, caption=NaNException 异常类]
class PeekingEmptyError : public std::exception {
    const char *what() const _NOEXCEPT override {
        return "Cannot peek empty stack";
    };
};
\end{lstlisting}

异常在合适的代码位置被抛出,但应该被处理,否则用户就会因为未处理的异常在使用程序时遇到软件终止的情况。%
异常的设计是:该代码段不应该处理该问题或是处理问题时会带来额外的副作用,就应该在此处抛出异常。本项目的%
异常基本都得到了处理。


\lstdefinestyle{console} {
    language=HTML,
    backgroundcolor=\color{CPPDark},   
    commentstyle=\color{CPPLight},
    keywordstyle=\color{red},
    ndkeywordstyle={},
    morekeywords={Unfinished, input, Invalid, character},
    numbers=none,
    stringstyle={},
    basicstyle=\fira\color{white}\footnotesize,
    breaklines=true,                                     
    keepspaces=true,                                     
    showspaces=false,                
    showstringspaces=false,
    showtabs=false,                  
    tabsize=2,
    escapechar=\%,
    identifierstyle={},
}

\chapter{数据测试}

\section{正常情况}
羡慕的输出分为三个部分:打印原始迷宫、打印路径、打印含有路径的迷宫。
\begin{enumerate}
    \item 输出原始迷宫
\begin{lstlisting} [style=console]
    ####################
    #1#..#.#..#.#..#0..#
    #.#.#...#...#.####.#
    #...###.##.#..##.#.#
    ###...#....##....#.#
    #...######..#...#..#
    #.###......#..##..##
    #..##.#####..##..###
    ##..#..####.##..####
    ###...####.....##..#
    #.##.####..###..##.#
    #.##....##..##.##..#
    #..####....###..#..#
    ##......#######...##
    ####################    
\end{lstlisting}

    \item 输出路径
\begin{lstlisting} [style=console]
    The route is: 
    <1,1> --> <2,1> --> <3,1> --> <3,2> --> <3,3> --> 
    <4,3> --> <5,3> --> <5,2> --> <5,1> --> <6,1> --> 
    <7,1> --> <7,2> --> <8,2> --> <8,3> --> <9,3> --> 
    <9,4> --> <10,4> --> <11,4> --> <11,5> --> <11,6> --> 
    <11,7> --> <12,7> --> <12,8> --> <12,9> --> <12,10> --> 
    <11,10> --> <10,10> --> <9,10> --> <9,11> --> <9,12> --> 
    <9,13> --> <9,14> --> <8,14> --> <8,15> --> <7,15> --> 
    <7,16> --> <6,16> --> <6,17> --> <5,17> --> <5,18> --> 
    <4,18> --> <3,18> --> <2,18> --> <1,18> --> <1,17> --> 
    <1,16>
\end{lstlisting}    

    \item 含有路径的迷宫
\begin{lstlisting} [style=console]
    ####################
    # #..#.#..#.#..#   #
    # #.#...#...#.#### #
    #   ###.##.#..##.# #
    ### ..#....##....# #
    #   ######..#...#  #
    # ###......#..##  ##
    #  ##.#####..##  ###
    ##  #..####.##  ####
    ###  .####     ##..#
    #.## ####. ###..##.#
    #.##    ## .##.##..#
    #..####    ###..#..#
    ##......#######...##
    ####################
\end{lstlisting}    

\end{enumerate}


\section{输入异常}
将路径封死查看结果。

\begin{enumerate}

    \item 原始迷宫
\begin{lstlisting} [style=console]
    ####################
    #1#..#.#..#.#..#0..#
    #.#.#...#...#.####.#
    #...###.##.#..##.#.#
    ###...#....##....### // 在这里封死
    #...######..#...#..#
    #.###......#..##..##
    #..##.#####..##..###
    ##..#..####.##..####
    ###...####.....##..#
    #.##.####..###..##.#
    #.##....##..##.##..#
    #..####....###..#..#
    ##......#######...##
    ####################
\end{lstlisting}


    \item 输出结果
\begin{lstlisting} [style=console]
    No way out!
\end{lstlisting}
    有一句警告,没有其他输出。

\end{enumerate}