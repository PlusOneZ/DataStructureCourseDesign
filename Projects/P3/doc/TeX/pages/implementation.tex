
\chapter{类结构}
本项目使用了{\kaishu 链式栈(\lstinline{Stack})}和{\kaishu 迷宫(\lstinline{Maze})}两个类型。%
在迷宫类型中封装了深度优先搜索寻路的实现、迷宫和路径;其中路径的存储用的是链式栈。

\begin{itemize}
    \item \lstinline{Stack} 类\\
    链式栈,可以在尾后插入、删除。
    \item \lstinline{Maze} 类\\
    维护迷宫数组、路径、访问状态数组,实现了打印、寻路、交互功能。
\end{itemize}


\chapter{Stack 类}

\section{API}

\subsection{类定义}
以下是本题使用链式栈的 API,各函数的功能已用注释标出。
\begin{lstlisting}[morendkeywords={LinkedList}, firstnumber=637, caption=LinkedList 类定义]{}
template<class Type>
class Stack {
    struct Node {  // 内部节点类
        Node *next = nullptr;
        Type data;

        explicit Node(const Type &d) : data(d) {}
        Node() = default;
    };

public:
    Stack();
    ~Stack();

    const Type &peek() const; // 栈顶元素
    bool pop();               // 弹出
    void push(const Type &d); // 压入

    void clear();             // 清除
    bool isEmpty() const;     // 判空

    bool show(std::ostream &os, const std::string &sep) const;

protected:
    Node *head = nullptr;
};
\end{lstlisting}

{
\lstset{
    basicstyle=\ttfamily\color{CPPDark}, 
    numberstyle=\tiny\color{darkgray},
    keywordstyle=\color[RGB]{40,40,255},
    }
\subsection{公开成员}
\begin{enumerate}
    \item \lstinline{const Type &peek() const;}\\
          返回栈顶的元素,但不删除,只是{\kaishu 看一眼}。
    \item \lstinline{bool pop();} \\
          弹出栈顶元素,返回值表示成功或失败。
    \item \lstinline{void push(const Type &d);} \\
          用 $d$ 创建一个新节点并压入栈中。
    \item \lstinline{void clear();} \\
          清空栈内所有元素。
    \item \lstinline{bool show(ostream &os, const string &sep) const;} \\
          打印栈中所有元素(按入栈顺序),并用 \lstinline{sep} 隔开。
\end{enumerate}

}

\section{代码逻辑与实现}


{
\lstset{
      morendkeywords={Stack},
      morendkeywords = {head}
}
\begin{enumerate}
      \item 压入栈中
\begin{lstlisting} [firstnumber = 93, caption={Stack::push}]
template<class Type>
void Stack<Type>::push(const Type &d) {
    auto nNode = new Node(d);
    nNode->next = head->next;
    head->next = nNode;
} \end{lstlisting}
      \item 弹出
\begin{lstlisting} [firstnumber = 82, caption={Stack::pop}]
template<class Type>
bool Stack<Type>::pop() {
    if (isEmpty()) {
        return false;
    }
    Node *p = head->next;
    head->next = p->next;
    delete p;
    return true;
}\end{lstlisting}
       
\end{enumerate}
}

\chapter{Maze 类}

功能封装。

\section{API}

以下是 \lstinline{Maze} 类的类定义。
{
\lstset{morendkeywords={Maze}}
\begin{lstlisting}[firstnumber=121, caption=Maze 类定义]{}
class Maze {
public:
    class Position {
        int coordinate[2] = {0, 0};
    public:
        Position(int x, int y) {...}
        Position() = default;

        std::ostream &show(std::ostream &os) {...}
        int x() const { return coordinate[0]; }
        int y() const { return coordinate[1]; }
    };

    Maze() : start(1, 16) { }

    void run() {...}

protected:
    static constexpr int width = 100;
    static constexpr int height = 100;

    char maze[height][width] = {
            "####################",
            "#1#..#.#..#.#..#0..#",
            "#.#.#...#...#.####.#",
            "#...###.##.#..##.#.#",
            "###...#....##....#.#",
            "#...######..#...#..#",
            "#.###......#..##..##",
            "#..##.#####..##..###",
            "##..#..####.##..####",
            "###...####.....##..#",
            "#.##.####..###..##.#",
            "#.##....##..##.##..#",
            "#..####....###..#..#",
            "##......#######...##",
            "####################",
    };
    enum Token { Unvisited, ToBeMarked, BadWay };
    Token traversed[height][width] = { Unvisited };

    Position start;
    Stack<Position> route;
    Stack<int> routeSequence;
    int directions[4][2] = {{-1, 0},
                            {1,  0},
                            {0,  1},
                            {0,  -1}
    };

    void showMaze() {...}
    void modifyMaze() {...}
    char &getPos(int x, int y) {...}
    char &getPos(const Position &pos) {...}
    void clearAllTraversed() {...}
    bool checkValid(const Position &pos) {...}
    bool isDestination(const Position &pos) {...}
    bool DeepFirstSearch(const Position &position) {...}
};  
\end{lstlisting}


\section{实现}

从API中可以卡出,本项目的迷宫写在了代码里,修改迷宫的内容可以直接进入代码并编译。但是本题的接口并不仅限于代码中给出的迷宫,%
修改迷宫可以改变代码输出,且如果没有路径会给出相应提示。

迷宫类的实现重点在于 \lstinline{DeepFirstSearch} 函数的实现:

\begin{lstlisting} [firstnumber = 247]
bool DeepFirstSearch(const Position &position) {
    Position curPos = start;
    if (checkValid(curPos)) {
        route.push(curPos);
    } else {
        return false;
    }
    bool canForward;    // 记录是否前进
    while (true) {
        if (isDestination(curPos)) {
            break;
        }
        canForward = false;
        for (const auto &d : directions) {
            int x_new = curPos.x() + d[0];
            int y_new = curPos.y() + d[1];
            // 前进
            if (checkValid(Position(x_new, y_new))) {
                canForward = true;
                traversed[curPos.x()][curPos.y()] = ToBeMarked;
                curPos = Position(x_new, y_new);
                route.push(curPos);
                break;
            }
        }
        if (!canForward) { // 标为死路后退
            traversed[curPos.x()][curPos.y()] = BadWay;
            route.pop();
            if (route.isEmpty()) {
                return false;
            }
            curPos = route.peek();
        }
    }
    return !route.isEmpty();
}
\end{lstlisting}